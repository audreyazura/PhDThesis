\documentclass[a4paper,12pt,nofrench]{thujf}

\usepackage[utf8]{inputenc}
\usepackage[T1]{fontenc}
\usepackage{array}
\usepackage{amsmath}
\usepackage[english]{babel}
\usepackage{bm}
\usepackage{graphicx}
\usepackage[a4paper]{geometry}
\usepackage[colorlinks=true,urlcolor=blue,linkcolor=blue]{hyperref}
\usepackage{url}
\usepackage[nottoc,numbib]{tocbibind}
\usepackage{color}
\usepackage{epstopdf}
\usepackage{xcolor}
\usepackage[backend=biber,style=phys]{biblatex}
\usepackage{upgreek}
\usepackage[capbesideposition={right,center}]{floatrow}
\usepackage[ampersand]{easylist}
\usepackage{lipsum}
\usepackage{pdfpages}
\usepackage[english]{minitoc}

%%%%%%%%%%%%%%%%%%%% Bibliography %%%%%%%%%%%%%%%%%%%%

\addbibresource{../Bibliography.bib}



%%%%%%%%%%%%%%%%%%%% Chapter numbering %%%%%%%%%%%%%%%%%%%%

\makeatletter
	\renewcommand{\thechapter}{\Roman{chapter}}
\makeatother



%%%%%%%%%%%%%%%%%%%% Special command for tables %%%%%%%%%%%%%%%%%%%%

\newcolumntype{M}[1]{>{\centering\arraybackslash}m{#1}}

\floatsetup[table]{style=plaintop}



%%%%%%%%%%%%%%%%%%%% Header %%%%%%%%%%%%%%%%%%%%

\newcommand\upun[1]{\uppercase{\underline{\underline{#1}}}}
\FormatHeadingsWith\upun

\newcommand\itheadings[1]{\textit{#1}}
\FormatHeadingsWith{\itheadings}

%%for a line under the header:
\setlength{\HeadRuleWidth}{0.4pt}



%%%%%%%%%%%%%%%%%%%% Start of the document %%%%%%%%%%%%%%%%%%%%

\begin{document}

%%%%%%%%%%%%%%%%%%%% Header and footer positions and style %%%%%%%%%%%%%%%%%%%%

\OddHead={{\leftmark\rightmark}{\hfil\slshape\rightmark}}
\EvenHead={{\leftmark}{{\slshape\leftmark}\hfil}}
\OddFoot={\hfil\thepage}
\EvenFoot={\thepage\hfil}
\pagestyle{ThesisHeadingsII}


%%%%%%%%%%%%%%%%%%%% Style of the Table of Content %%%%%%%%%%%%%%%%%%%%

\FrameChaptersInToc

%%small ToC for each chapter:
\dominitoc


%%%%%%%%%%%%%%%%%%%% Start of the thesis %%%%%%%%%%%%%%%%%%%%

\includepdf{Pictures/THES_MOD_02_UGA_couverture_these_cotutelle.pdf}

%\shipout\null

\clearpage
\thispagestyle{empty}

\chapter*{Introduction}
\pagenumbering{roman}

aa

	Constructing a quantum computer is one of the challenges of this century. The core component of this computer is the \emph{qubit}, the quantum bits. Instead of regular bits, that can take the states (values) $|0\rangle$ and $|1\rangle$, the \emph{qubits}, being quantum devices, can also be in a superposition of states, $\alpha |0\rangle + \beta |1\rangle$. As system able to store the quantum information is therefore needed. The two main criteria for this system is its characteristic time, that must be long enough to do the operation and stock the results, and the ease of preparation in a given state, determining the speed of each operation.
%	
%	Miniaturisation is one of the challenge of this century: electronic system keep getting smaller. On the same time, our need in informatics power keeps growing. In order to keep up, memories have to became smaller and the physical system storing the information too. We propose in this thesis, to study a system that reach the ultimate limit of information storing: using a single atom as a bit of information.
%	
%	In a computer, the operation are done on the time scale of a few tens of picosecond. The system stocking the information must last longer than the characteristic time of those operations.
	
	One promising system for the realization of a \emph{qubit} is the quantum dots (QDs), nanometer sized devices designed to confine carrier in all three dimensions. This confinement leads to a quantization of the carriers energy, akin to the energy level of the electrons in an isolated atom. For this reason, they are dubbed "artificial atom". 
	
	Multiple methods exists to form such devices: gate trapping single electrons in-between the electrodes (logical \emph{qubit}), nanometers-sized grains formed by the precipitation of semiconductors in a solution (colloidal dots), thickness variation of a quantum well, strains relaxation of a semiconductor layer... I will focus in this thesis on the later type of QDs, usually grown using Molecular Beam Epitaxy (MBE). They are formed by small island, with a characteristic size of a few nanometers, of a small gap semiconductor inserted in a wide gap semiconductor matrix. Well known example are InAs/GaAs (for III-V semiconductors), CdSe/ZnSe or CdTe/ZnTe (for II-VI semiconductors). More specifically, in this thesis, I studied optically active QDs: carriers can be injected in via excitation from a laser, and their relaxation comes with the emission of a photon.
	
	The spin of the carriers injected in a QD is a good candidate for the realization of a two level quantum system. For a single logical \emph{qubit}, coherence time of the carriers as high as 1 $\mu$s was found~\cite{PettaCohManipElSpin}. Moreover, it has been demonstrated that QDs can be used to control electrically (for the logical \emph{qubit}) or optically (for the optically active dots) the spin of the injected carriers~\cite{GreilichControlElSpin, PressOptControlSpin}. Finally, the optical preparation of the carrier spin state takes only a few nanoseconds. All of this makes the spin of carriers trapped in a QD a really promising system for the realization of a \emph{qubit}~\cite{LossQuantComput, ImamogluQDQuantInfo, AwschalomSpinCohSC, WolfSpintronics}. However, the dephasing time of an ensemble of QDs is a lot lower than the coherence time of single QD, falling to about 10 ns~\cite{GurudevOptGenElSpinCoh, BrackerOptPumpElNucSpin, BraunElSpinRelax}. This is a bit too short to do any significant data storage.
	 
	Exiting the world of QDs, two systems can be proposed to get longer spin coherence coherence: Nitrogen-Vacancy (NV) centers in diamond~\cite{FuchsNVQMem} or atomic spins directly inserted in the semiconductors~\cite{PierreAtomSpin}. In both those systems, the quantum information can be stored on the spin of the NV center or the magnetic atom. In NV centers, electronic spin coherence time in the milliseconds range was found in ultrapure isotopically purified diamonds~\cite{NVCohTime}. However, the preparation of the electronic spin of the NV center takes hundreds of nanoseconds, which would slow the calculations down~\cite{FuchsNVQMem}. The same kind of coherence and manipulation time can be expected for the atomic spins.
	 
	Another approach, in-between the spin of carriers in QDs and the spin of NV centers comes from the Diluted Magnetic Semiconductors (DMS). In these material, a low density of magnetic atoms are inserted in the semiconductor lattice. The semiconductor keeps its conventional optical and electrical properties, well known, and new one arises from the presence of the magnetic atoms.	It was shown that there is a strong exchange interaction between the carriers and the magnetic atoms spins. For the realization of spin \emph{qubits}, the idea is to insert the magnetic atoms in QDs in order to use this exchange interaction to control the magnetic atoms spins via the injected carriers. In this thesis, this reasoning is pushed to its limit, inserting a single magnetic atom in a QD, and controlling it optically. Such individual spins are promising for the implementation of emerging quantum information technologies in the solid state~\cite{VeldhorstTwoQbitLogGate, SaeediRoomTQbitStor,BarGillElSpinCoh}. They were to present many desirable features for the realization of spin \emph{qubits}, such as reproducible quantum properties, stability, and potential scalability for further applications~\cite{KoenradSingDop}. Thanks to their point-like character, a longer spin coherence time (compared to carriers’ spins) can also be expected at low temperature. All of this makes single magnetic dopants in QDs good candidate to store quantum information.
	
	The control of the spin state of individual~\cite{FirstMn, ClairStarkEffect, KudelskiMnInAsFineStruct, KrebsMnInAsMagAniso, BaudinOptPumpInAs, KobakDesignQDSolotronic} or pairs~\cite{BesombesTwoMn,KrebsTwoMn} of magnetic atoms has been demonstrated. The spin of a magnetic atom in a QD can be prepared by the injection of spin polarized carriers and its state can be read through the energy and polarization of the photons emitted by the QD~\cite{OptControlSpin, OptSignSpinSwitch, OptManipMn}. The insertion of a magnetic atom in a QD where the strain or the charge states can be controlled also offers degrees of freedom to tune the properties of the localized spin such as its magnetic anisotropy~\cite{ObergContMagAniso}.

\begin{table} \centering
	\begin{tabular}{|m{3cm}|m{1cm}|m{1cm}|m{1cm}|m{1cm}|m{1cm}|m{1cm}|m{1cm}|}
		\hline		
		Inserted atom & V$^{2+}$ & Cr$^{2+}$ & Mn$^{2+}$ & Fe$^{2+}$ & Co$^{2+}$ & Ni$^{2+}$ & Cu$^{2+}$ \\
		\hline \hline
		$d$-shell & $d^3$ & $d^4$ & $d^5$ & $d^6$ & $d^7$ & $d^8$ & $d^9$ \\
		\hline
		Electronic spin & 3/2 & 2 & 5/2 & 2 & 3/2 & 1 & 1/2 \\
		\hline
		Nuclear spin & 7/2 & 0 & 5/2 & 0 & 7/2 & 0 & 3/2 \\
		\hline
	\end{tabular}
	\caption{List of different possible transition metals and their key properties in the context of our study.}
	\label{DMSAtoms}
\end{table}

	Tab.~\ref{DMSAtoms} lists the different magnetic atoms that can be inserted in a semiconductor lattice. Each of those atoms has a unique set of electronic spin, nuclear spin and orbital momentum. For a given semiconductor structure, those properties change the magnetic atom behaviour. Each can be used for different applications. Mn was the first atom to be successfully inserted and optically probed in CdTe/ZnTe QDs, in 2004~\cite{FirstMn}. Since then, other magnetic atoms have been embedded in II-VI QDs and studied: Co (2014)~\cite{KobakCo} and Fe (2016)~\cite{SmolenskiFe}. Mn was also embedded in InAs dots (III-V semiconductor)~\cite{KudelskiMnInAsFineStruct}, but this has not be realized for other magnetic atom until now.
	
	Mn in II-VI semiconductors has been widely studied in the last decades. In bulk semiconductors, its relaxation time was found to reach the milliseconds range, for vanishing Mn concentration~\cite{ScalbertSpinRelaxCdMnTe, DietlDynaSpinMnDMS}. Inserted in II-VI QDs, it was demonstrated that a single Mn spin could be optically prepared in a few tens of nanoseconds, depending on the laser power~\cite{ClaireTh}. In the same time, a relaxation time of the Mn spin of a few microseconds was found~\cite{OptControlSpin}. The dynamic of a Mn spin was also probed in a positively charged QD, forming a hybrid spin by coupling with the resident hole~\cite{DynhMn}, and in a strain-free environment~\cite{LucienSFD}.
	
	Single Cr atom in a QD is also of particular interest: thanks to its orbital momentum, it is really sensible to the strains. This opens new ways to manipulate the spin state of this magnetic atom without having to use optical excitation. It also opens the possibility to realize spin mechanical system where the Cr is used as a \emph{qubit} to interact with an oscillator. It can be used to probe position of the oscillator, cool it down or create non-classical states. Moreover, the Cr atom in a II-VI matrix presents no nuclear spin. There is therefore no hyperfine interaction for Cr atom in a CdTe/ZnTe QD. This is expected to simplify the Cr spin structure and lead to longer coherence time.
	
	In this thesis, I will present a detailed study of the hole-Mn hybrid spin, and to start the study of a single Cr atom in a QD. Those two systems are promising for the realization of spin \emph{qubit} coupled to strains. Growth of the Mn-doped QDs was done in Grenoble, in the INAC-CNRS joined team NPSC, by Herv\'e Boukari. The Cr-doped QDs were grown in Tsukuba, in the team of Pr. Shinji Kuroda, by Hayato Ustumi, Masahiro Sunaga and myself. The optical study of the magnetic QDs was performed at the N\'eel Institut in Grenoble.
	
	This thesis is organized as follows:
	\begin{description}
		\item[Chapter~I%~\ref{DMSQDTh}
] I give in this chapter the theoretical background of this thesis. I begin to discuss the properties of a semiconductor crystal. This discussion is then used as a basis to present the physics of the QDs and their properties. I leave then the world of QDs to study the interaction between carriers and magnetic atom in a diluted magnetic semiconductor. Then I discuss in more details the interaction between the carriers and the two atoms studied in this thesis: the Mn and the Cr. I also show how the inclusion of these magnetic atoms in a crystal affects their spin energy structure. Finally, I present a short example of application of these theories on singly Mn doped QDs.% This theoretical basis will hopefully help the reader to understand the rich physics behind the seemingly simple system of a magnetic atom interacting with an electron and a hole in a QD.
		\item[Chapter~II%~\ref{Growth}
] The growth of Cr doped QDs was an important part of this thesis. I present here the techniques used to grow the samples studied optically. I begin with a general explanation of the MBE process. I then explain how the Cr-doped QDs were grown and how they are formed using Stranski-Krastanov (SK) relaxation. I also present some tests that were done for the growth of two other kinds of Cr-doped samples: charge tunable sample, and strain-free dots formed by the thickness variation of a quantum well. For each kind of sample, I present basic optical characterization.
		\item[Chapter~III%~\ref{CoDynMn}
] I discuss in this chapter the dynamic of the hole-Mn hybrid spin. I show that optical $\Lambda$-level systems exist in the spin structure. Those systems link two hole-Mn ground states to one X$^+$-Mn excited state via two transitions of opposite polarizations. They were used to study the dynamics of the hole-Mn hybrid spin. A fast hole-Mn spin relaxation, in the nanoseconds range, by an interplay of the interplay with acoustic phonons and the hole-Mn exchange interaction is evidenced. I also show that two X$^+$-Mn level can be coupled by the in-plane strain anisotropy and study this strain induced coherent dynamics.
		\item[Chapter~IV%~\ref{MagOptStud}
] In this chapter, I show that it is possible to include single Cr spin in CdTe/ZnTe QDs and probe its spin optically. The Cr spin structure is deduced from magneto-optical experiments, and shows strong influence of strain. A value of the magnetic anisotropy $D_0$ between 2 and 3 meV is extracted, two orders of magnitude higher than what is found in Mn-doped QD or in NV centers in diamond. The sign of the hole-Cr exchange interaction is also deduced from these experiments.% Finally, I discuss the possibility for a Cr close in the ZnTe barrier close to the QD to influence the QD emission. 
		\item[Chapter~V%~\ref{CrDyn}
] This chapter explore the dynamics of the Cr spin in a QD. The study begin with photon correlation experiments. In order to get more precise results, resonant optical pumping experiments were performed. I begin to present the optical setup, before discussing the results. The success of this experiment shows it the possibility to prepare the Cr spin by optical means. A strong influence of phonons on the Cr spin dynamics is evidenced. A Cr spin relaxation time in the microsecond range is extracted from the experiments. Finally, I also demonstrate the possibility to control the Cr spin by optical Stark effect.  
	\end{description}
	
\tableofcontents
\clearpage
\thispagestyle{empty}

\pagenumbering{arabic}
\clearpage
\thispagestyle{empty}
		
\printbibliography

\end{document}