\documentclass[a4paper,12pt,nofrench]{thujf}

\usepackage[utf8]{inputenc}
\usepackage[T1]{fontenc}
\usepackage{array}
\usepackage{amsmath}
\usepackage[english]{babel}
\usepackage{bm}
\usepackage{graphicx}
\usepackage[a4paper]{geometry}
\usepackage[colorlinks=true,urlcolor=blue,linkcolor=blue]{hyperref}
\usepackage{url}
\usepackage[nottoc,numbib]{tocbibind}
\usepackage{color}
\usepackage{epstopdf}
\usepackage{xcolor}
\usepackage[backend=biber,style=phys]{biblatex}
\usepackage{upgreek}
\usepackage[capbesideposition={right,center}]{floatrow}
\usepackage[ampersand]{easylist}
\usepackage{lipsum}
\usepackage{pdfpages}
\usepackage[english]{minitoc}

%%%%%%%%%%%%%%%%%%%% Bibliography %%%%%%%%%%%%%%%%%%%%

\addbibresource{../Bibliography.bib}



%%%%%%%%%%%%%%%%%%%% Chapter numbering %%%%%%%%%%%%%%%%%%%%

\makeatletter
	\renewcommand{\thechapter}{\Roman{chapter}}
\makeatother



%%%%%%%%%%%%%%%%%%%% Special command for tables %%%%%%%%%%%%%%%%%%%%

\newcolumntype{M}[1]{>{\centering\arraybackslash}m{#1}}

\floatsetup[table]{style=plaintop}



%%%%%%%%%%%%%%%%%%%% Header %%%%%%%%%%%%%%%%%%%%

\newcommand\upun[1]{\uppercase{\underline{\underline{#1}}}}
\FormatHeadingsWith\upun

\newcommand\itheadings[1]{\textit{#1}}
\FormatHeadingsWith{\itheadings}

%%for a line under the header:
\setlength{\HeadRuleWidth}{0.4pt}



%%%%%%%%%%%%%%%%%%%% Start of the document %%%%%%%%%%%%%%%%%%%%

\begin{document}

%%%%%%%%%%%%%%%%%%%% Header and footer positions and style %%%%%%%%%%%%%%%%%%%%

\OddHead={{\leftmark\rightmark}{\hfil\slshape\rightmark}}
\EvenHead={{\leftmark}{{\slshape\leftmark}\hfil}}
\OddFoot={\hfil\thepage}
\EvenFoot={\thepage\hfil}
\pagestyle{ThesisHeadingsII}


%%%%%%%%%%%%%%%%%%%% Style of the Table of Content %%%%%%%%%%%%%%%%%%%%

\FrameChaptersInToc

%%small ToC for each chapter:
\dominitoc


%%%%%%%%%%%%%%%%%%%% Start of the thesis %%%%%%%%%%%%%%%%%%%%

\includepdf{Pictures/THES_MOD_02_UGA_couverture_these_cotutelle.pdf}

%\shipout\null

\clearpage
\thispagestyle{empty}

\chapter*{Acknowledgement}
\pagenumbering{roman}

	I couldn't have finished this PhD if it was not for the help of a lot of person. I cannot have done it alone, and I would like to begin this manuscript taking a step backward to thank everyone that has been there for me, though those times that have not always been easy.
	
	Let's begin with the most obvious one. I want to express my more sincere gratitude to all my thesis supervisors. First, Lucien Besombes, who have been the first one to believe in me, and never let me get content with me. He always asked me to get the best of myself, and even pushed me to do more. Although not always easy to hear, his advices was always precious and helped me to get better, be it while doing experiment or writing my thesis. I also want to thank Pr. Shinji Kuroda, who offered me the possibility to do this double degree PhD, and did a lot so my stay in the University of Tsukuba would be as simple as possible. He helped me pass through the different demands of the university, and really supported me along my research in his laboratory. And finally, I should not forget Herv\'e Boukari, who helped me a lot on the growth side, and, along with Pr. Shinji Kuroda, supported me on all the link between the two universities. The double degree would have been a lot more difficult without him, if not impossible. And he always had some reassuring words, nice to hear after discussion with Lucien.
	
	The PhD would not have been completed without the juries to judge it. I want to deeply thank the member of my juries, both on the French side and on the Japanese side. The discussion we had during the questions on my defence was really interesting and challenging. A big thank you to Pr. Maria Chamarro and Aristide Brian for having read this thesis and given their honest opinion on it. I also want to thank Pr. Etienne Gheeraert, who has always been there to support me for the link between Grenoble and Tsukuba, and kindly accepted to preside the jury. And finally, for the French side, I want to than Bernhard Urbaszek to have accepted to be part of the jury. On the Japanese side, I want to thank Pr. Yasuhiro Tokura, Pr. Takashi Suemasu and Kazuaki Sakoda to have agreed to be part of my jury.
	
	But the supervisors and the jury are only a part of the people surrounding you during a thesis. Some many more persons make the working environment, and the thesis would be quite different without them. I would like to thank the person of the NPSC team. I think I was not the most well integrated PhD student, due in part to my long absences caused by my double degree with a Japanese university, but also to my office being a bit far from the usual offices of the team, and my usual difficulties to socialize. Nonetheless, it was always interesting to meet people of the team and discuss with them, be it about physics or other subjects. Among them, I want to extend a special thank to some PhD student Valentin Delmonte, Mathieu Jeannin, Alberto Artioli and Kimon Moratis, with whom I had quite a lot of discussion on a lot of subjects, and who where always interesting. In the permanent of the team, I especially want to thank Jo\¨el Cibert, always smiling and helpful, and Henri Mariette, always willing to discuss around a nice table in Japan.
	
	As I said, my office was a bit far from the main offices of the team NPSC. It made the contact with my team a bit more difficult, but it also gave me the possibility to meet a lot of PhD student from other team. Among them, I especially want to thank Yann P\'erin, Andr\'e Dias and Dayane De-Chouza-Chaves, with whom I shared an office for two or three years. It was a really lively, and always fun to be here. We had some great discussion, and often some great laugh. I will with no doubt miss the atmosphere of this open space. The arriving of Hawa Abdul-Latiff in the office was quite a good surprise too, and I want to thank her particularly. We bonded quite quickly, in part thank to our shared experience being double degree student between the university of Grenoble and the university of Tsukuba. This year and a half we shared seems way to short and it would be pleasure to meet again to continue our discussion. Speaking of encounter started in Grenoble thanks to my Japanese experience, I do not want to forget Ryogen Fujiwara. It was really nice to be able to discuss our two countries and to exchange about our experience.
	
	The Tsukuba side is not to be neglected. I meet several great persons during my time there. First, I want to thank the whole Kuroda team. I had some really great moment with them, and the whole have always been helpful every time I had a problem. They also was really accepting about the small error I did in Japanese etiquette, and I was able to learn a lot thank to them. I especially want to thank the whole dot team in the Kuroda lab: Hayato Utsumi, Masahiro Sunaga and Kenji Makita. It was a pleasure to work with them, and also to meet them outside of work. Even though the language was sometimes a problem, we found ways to have nice discussions. I also want to deeply thank Ryo Ishikawa and Takuma Nakamura. They did a lot to help me while I was there, and to go through the Japanese administrative side. I could not have done anything without them.
	
	The completion of this PhD and all these travels would not have been possible without the help of the administrative. I would like to thank especially Aur\'elie Laurent on Grenoble side, which always was of great help to prepare my travel, even though I was not always on time myself. In the University Grenoble-Alpes, Sandrine Ferrari has always been of great help, always being patient and giving quick answer to any question I had. On Tsukuba's side, I want to thank Shiromi Kikawada. She helped me with the paper needed for the University of Tsukuba, and was always really welcoming even when I did small errors.
	
	This travels would not have been possible save for the help of several organism on the financing side. The LANEF paid for three travels, the R\'egion Rhônes-Alpes for one and the CNRS for the last one. I want to thank these three entities to have agreed to help for the realisation of this thesis.
	
	The working environment is essential for the completion of a thesis, but people supporting you in your everyday life is at least as much important as the one supporting you at your workplace. And for that, I thank all my family to have been so supporting during this three last years. Thank you to my mother, my father, my grand parents, my father's girlfriend, Florence, and my aunt, Sylvie, to have always been here to support me, especially during the hardest time of my thesis. Friends are also particularly important in those periods, and I would like to thank all of them. First and foremost, I want to especially thank Ga\¨el Fatou, for all the discussion and good time we had together. And not to forget these great RPG session we had. Those really helped me to let of some stress. I also want to deeply thank Florent Auvray. He was the one who convinced me I could do things with Japan. Without his initial impulse, I would not even had thought of doing a PhD between France and Japan. I am really thankful of B. Matten for all the advice in English he gave me, and all the rant he allowed me. My gratitude also goes to Nicolas "Kabu" Ludi\`eres, for all the great discussion we had about animation that helped me change my mind during hard times, and for covering for me when I was not able to fulfil my role as president of Nijikai. More globally, I want to thank all the person who have been here for me during time of need: Adrien Daubois, Amaury Josse, C\'eline "Nobody" Gallien, Avatar Z. Brown, Rebecca Wright. And finally, I would like to thank all the member of the non-profit organisation Nijikai, who have been really understanding when I had to step back as a president during a pre-convention rush because I had to work on my thesis.

\clearpage
\thispagestyle{empty}

\chapter*{Introduction}

	Building a quantum computer is one of the challenges of this century. The core component of such a computer is the \emph{qubit}, the quantum bit. Instead of regular bits, which can take the states (values) $|0\rangle$ and $|1\rangle$, the \emph{qubits}, being quantum devices, can also be in a superposition of states, $\alpha |0\rangle + \beta |1\rangle$. An important step in the realization of the computer is to find a system to store and control these quantum states, which do not exist yet. The two main criteria for this system are its characteristic time, that must be long enough to do the operation and store the result, and the speed of preparation in a given state, determining the speed of each operation. Moreover, it has to be possible to build gate of one or two \emph{qubits}. The system has also to be scalable, in order to be able to build quantum component with a large number of \emph{qubits}.
%	
%	Miniaturisation is one of the challenge of this century: electronic system keep getting smaller. On the same time, our need in informatics power keeps growing. In order to keep up, memories have to became smaller and the physical system storing the information too. We propose in this thesis, to study a system that reach the ultimate limit of information storing: using a single atom as a bit of information.
%	
%	In a computer, the operation are done on the time scale of a few tens of picosecond. The system stocking the information must last longer than the characteristic time of those operations.
	
	Several approaches exist for the fabrication of a \emph{qubit}, such as cold atoms, superconductors... A promising system for their realization is a single quantum dot (QD), nanometer-sized objects designed to confine carriers in all three dimensions. This confinement leads to a quantization of the carrier energy, akin to the energy level of the electron in an isolated atom.
	
	Multiple methods exist to form such devices: gate trapping of single electron between electrodes, nanometer-sized grains formed by the precipitation of semiconductors in a solution (colloidal dots), thickness variation of a quantum well, strain relaxation of a semiconductor layer... I will focus in this thesis on the later type of QDs, usually grown by Molecular Beam Epitaxy (MBE). They are formed by small island, with a characteristic size of a few nanometers, of a small gap semiconductor inserted in a wide gap semiconductor. Well-known examples are InAs/GaAs (for III-V semiconductors), CdSe/ZnSe or CdTe/ZnTe (for II-VI semiconductors). More specifically, in this thesis, I studied optically active QDs: carriers can be injected in the QD by a laser excitation, and their relaxation occurs with the emission of a photon.
	
	The spin of the carriers injected in a QD is a good candidate for the realization of a two level quantum system. For a single gate \emph{qubit}, coherence time of the electrons as long as 1 $\mu$s was found~\cite{PettaCohManipElSpin}. Moreover, it has been demonstrated that QDs can be used to control electrically (for the gate \emph{qubit}) or optically (for the optically active dots) the spin of the injected carriers~\cite{GreilichControlElSpin, PressOptControlSpin}. Finally, the optical preparation of the carrier spin state takes only a few nanoseconds. All of this makes the spin of carriers trapped in a QD a promising system for the fabrication of \emph{qubits}~\cite{LossQuantComput, ImamogluQDQuantInfo, AwschalomSpinCohSC, WolfSpintronics}. However, the dephasing time of an ensemble of QDs is a lot shorter than the coherence time of single QD, falling to about 10 ns~\cite{GurudevOptGenElSpinCoh, BrackerOptPumpElNucSpin, BraunElSpinRelax}. This is too short to do any significant data storage or processing.
	 
	Exiting the world of QDs, several systems were proposed to get longer spin coherence time, such as Nitrogen-Vacancy (NV) centers in diamond~\cite{FuchsNVQMem} or atomic spins directly inserted in the semiconductors~\cite{PierreAtomSpin}. In NV centers, electronic spin coherence time in the milliseconds range was found in ultrapure isotopically purified diamonds~\cite{NVCohTime}. However, the preparation of the electronic spin of the NV center takes hundreds of nanoseconds, which would slow the calculations down~\cite{FuchsNVQMem}. The same kind of coherence and manipulation time can be expected for the atomic spins.
	 
	Another approach comes from the Diluted Magnetic Semiconductors (DMS). In these materials, a low density of magnetic atoms is inserted in the semiconductor lattice. The semiconductor keeps its conventional optical and electrical properties and new ones arise from the presence of the magnetic atoms.	It was shown that there is a strong exchange interaction between the carriers and the magnetic atom spins. When inserting magnetic atoms in a quantum dot, the carriers are confined with them. Their interaction is enhanced, enabling the control of the magnetic atoms spins via the injected carriers. In this thesis, this reasoning is pushed to its limit, inserting a single magnetic atom in a QD, and controlling its spin optically. Such individual spins are promising for the implementation of emerging quantum information technologies in the solid state~\cite{VeldhorstTwoQbitLogGate, SaeediRoomTQbitStor,BarGillElSpinCoh}. They were to present many desirable features for the realization of spin \emph{qubits}, such as reproducible quantum properties, stability, and potential scalability by coupling dots~\cite{KoenradSingDop}. Thanks to their point-like character, a longer spin coherence time (compared to carriers’ spins) can also be expected at low temperature. All of this makes single magnetic dopants in QDs a good candidate to store quantum information.
	
	The control of the spin state of individual~\cite{FirstMn, ClairStarkEffect, KudelskiMnInAsFineStruct, KrebsMnInAsMagAniso, BaudinOptPumpInAs, KobakDesignQDSolotronic} or pairs~\cite{BesombesTwoMn,KrebsTwoMn} of magnetic atoms has been demonstrated. The spin of a magnetic atom in a QD can be prepared by the injection of spin polarized carriers and its state can be read through the energy and polarization of the photons emitted by the QD~\cite{OptControlSpin, OptSignSpinSwitch, OptManipMn}. The insertion of a magnetic atom in a QD where the strain or the charge states can be controlled also offers degrees of freedom to tune the properties of the localized spin such as its magnetic anisotropy~\cite{ObergContMagAniso}.

\begin{table} \centering
	\begin{tabular}{|m{3cm}|m{1cm}|m{1cm}|m{1cm}|m{1cm}|m{1cm}|m{1cm}|m{1cm}|}
		\hline		
		Inserted atom & V$^{2+}$ & Cr$^{2+}$ & Mn$^{2+}$ & Fe$^{2+}$ & Co$^{2+}$ & Ni$^{2+}$ & Cu$^{2+}$ \\
		\hline \hline
		$d$-shell & $d^3$ & $d^4$ & $d^5$ & $d^6$ & $d^7$ & $d^8$ & $d^9$ \\
		\hline
		Electronic spin & 3/2 & 2 & 5/2 & 2 & 3/2 & 1 & 1/2 \\
		\hline
		Nuclear spin & 7/2 & 0 & 5/2 & 0 & 7/2 & 0 & 3/2 \\
		\hline
	\end{tabular}
	\caption{List of different possible transition metals and their key properties in the context of our study.}
	\label{DMSAtoms}
\end{table}

	Tab.~\ref{DMSAtoms} lists some magnetic atoms that can be inserted in a semiconductor lattice. Each of those atoms has a unique set of electronic spin, nuclear spin and orbital momentum. For a given semiconductor structure, those properties change the magnetic atom behaviour. Each can be used for different applications, such as the realization of a spin mechanical \emph{qubit} for the elements with an orbital momentum. The first atom to have been inserted in a quantum dot is the Mn, first in II-VI (2004)~\cite{FirstMn}, and then in III-V (2007)~\cite{KudelskiMnInAsFineStruct}. Since then, other magnetic atoms have been embedded in II-VI QDs and studied: Co (2014)~\cite{KobakCo} and Fe (2016)~\cite{SmolenskiFe}. They have not been inserted successfully in III-V semiconductors yet.
	
	Mn in II-VI semiconductors has been widely studied in the last decades. In bulk semiconductors, its relaxation time was found to reach the milliseconds range, for vanishing Mn concentration~\cite{ScalbertSpinRelaxCdMnTe, DietlDynaSpinMnDMS}. Inserted in II-VI QDs, it was demonstrated that a single Mn spin could be optically prepared in a few tens of nanoseconds, depending on the laser power~\cite{ClaireTh}. At the same time, a relaxation time of the Mn spin of a few microseconds was found~\cite{OptControlSpin}. The dynamic of a Mn spin was also probed in a positively charged QD, forming a hybrid spin by coupling with the resident hole~\cite{DynhMn}, and in a strain-free environment~\cite{LucienSFD}.
	
	Single Cr atom in a QD is also of particular interest: thanks to its orbital momentum, it should be very sensitive to strains. This opens new ways to manipulate the spin state of this magnetic atom without having to use optical excitation. It also opens the possibility to realize spin mechanical system where the Cr is used as a \emph{qubit} coupled to an oscillator. Cr could be used to probe the movement of the oscillator, cool the oscillator down or create non-classical states of the oscillator. Moreover, the Cr atom in a II-VI matrix presents no nuclear spin. There is therefore no hyperfine interaction for Cr atom in a CdTe/ZnTe QD. This simplifies the Cr spin structure and leads to an expected longer coherence time.
	
	In this thesis, I will present a detailed study of the hole-Mn hybrid spin, and to start the study of a single Cr atom in a QD. Those two systems are promising for the realization of spin \emph{qubit} coupled to strains. Growth of the Mn-doped QDs was done in Grenoble, in the INAC-CNRS joined team NPSC, by Herv\'e Boukari. The Cr-doped QDs were grown in Tsukuba, in the team of Pr. Shinji Kuroda, by Hayato Ustumi, Masahiro Sunaga and myself. The optical study of the magnetic QDs was performed at the N\'eel Institut in Grenoble, where an optical setup for the study of single quantum dots doped with a single magnetic atom had been developed.
	
	This thesis is organized as follows:
	\begin{description}
		\item[Chapter~I%~\ref{DMSQDTh}
] This chapter focuses on the theoretical background of this thesis. I begin to discuss the properties of a semiconductor crystal. This discussion is then used as a basis to present the physics of the QDs and their properties. Then the interaction between carriers and magnetic atom in a diluted magnetic semiconductor is presented. Particular attention is given to the interaction between the carriers and the two atoms studied in this thesis: the Mn and the Cr. I also show how the inclusion of these magnetic atoms in a crystal affects their spin energy structure. Finally, I present a short example of application of these theories on CdTe/ZnTe QDs doped with single Mn.% This theoretical basis will hopefully help the reader to understand the rich physics behind the seemingly simple system of a magnetic atom interacting with an electron and a hole in a QD.
		\item[Chapter~II%~\ref{Growth}
] The growth of Cr doped QDs was an important part of this thesis. I present here the techniques used to grow the samples studied optically. I begin with a general explanation of the MBE process. I then explain how the Cr-doped QDs were grown and how they are formed using Stranski-Krastanov (SK) relaxation. I also present some tests that were done for the growth of two other kinds of Cr-doped samples: charge tunable sample, and strain-free dots formed by the thickness variation of a quantum well. For each kind of sample, I present basic optical characterization.
		\item[Chapter~III%~\ref{CoDynMn}
] I discuss in this chapter the dynamic of the hole-Mn hybrid spin. I show that spin states form optical $\Lambda$-level systems. Two hole-Mn ground states are connected to one X$^+$-Mn excited state via two transitions of opposite polarizations. They were used to study the dynamics of the hole-Mn hybrid spin. A fast hole-Mn spin relaxation, in the nanoseconds range, caused by the interplay between acoustic phonons and the hole-Mn exchange interaction is evidenced. I also show that two X$^+$-Mn level can be coupled by the in-plane strain anisotropy and study this strain induced coherent dynamics.
		\item[Chapter~IV%~\ref{MagOptStud}
] In this chapter, I show that it is possible to include single Cr spin in CdTe/ZnTe QDs and probe its spin optically. The Cr spin structure is deduced from magneto-optical experiments. It is confirmed that the Cr spin is strongly coupled to strains. A value of the magnetic anisotropy $D_0$ between 2 and 3 meV is extracted. This is two orders of magnitude higher than what is found in Mn-doped QD or in NV centers in diamond. The sign of the hole-Cr exchange interaction is also extracted from these experiments and found to be anti-ferromagnetic.% Finally, I discuss the possibility for a Cr close in the ZnTe barrier close to the QD to influence the QD emission. 
		\item[Chapter~V%~\ref{CrDyn}
] This chapter explore the dynamics of the Cr spin in a QD. The study begins with photon correlation experiments. In order to get more precise results, resonant optical pumping experiments were performed. The optical setup is first presented, before discussing the results. The success of this experiment shows the possibility to prepare the Cr spin by spin pumping. A strong influence of phonons on the Cr spin dynamics is evidenced. A Cr spin relaxation time under excitation in the 10 nanoseconds range is extracted from the experiments. In the dark, the relaxation time of the Cr is found to be in the microsecond range. Finally, I also demonstrate the possibility to tune the energy of the Cr spin by optical Stark effect.  
	\end{description}
	
\tableofcontents
\clearpage
\thispagestyle{empty}

\pagenumbering{arabic}
\clearpage
\thispagestyle{empty}
		
\printbibliography

\end{document}