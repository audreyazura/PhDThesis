\documentclass[a4paper,12pt]{report}

\usepackage[utf8]{inputenc}
\usepackage[T1]{fontenc}
\usepackage{array}
\usepackage{amsmath}
\usepackage[english]{babel}
\usepackage{graphicx}
\usepackage[a4paper]{geometry}
\usepackage[colorlinks=true,urlcolor=blue,linkcolor=blue]{hyperref}
\usepackage{url}
\usepackage[nottoc,numbib]{tocbibind}
\usepackage{color}
\usepackage{epstopdf}
\usepackage{xcolor}
\usepackage[backend=biber,style=phys]{biblatex}

\addbibresource{../Bibliography.bib}

\makeatletter
	\renewcommand{\thechapter}{\Roman{chapter}}
\makeatother

\begin{document}

\chapter{Diluted magnetic semiconductor quantum dots\label{DMSQDTh}}

	\section{II-VI semiconductor quantum dots}
	
		\subsection{Band structure of CdTe/ZnTe\label{BandStruct}}
		
		ZnTe and CdTe are two II-VI semiconductor, meaning they are composed of an anion from the column VI of periodic table (Te), and a cation from the column II (Cd and Zn). They both crystallize as zinc blend when grown in Molecular Beam Epitaxy. As shown in Fig.~\ref{Zinc-Blende&Brillouin}, in this structure, each species is organized in a face centred lattice, one them being shift from the other by a quarter of the [111] diagonal. Each ion is then in a tetragonal environment, meaning the zinc-blende structure is of the $T_d$ space-group.

	\begin{figure}[h!]
	\begin{center}
		\includegraphics[width=10cm]{Pictures/ZincBlende-Wikipedia.png}
	\end{center}
	\caption{Zinc-blende crystal structure and first Brillouin zone.}
	\label{Zinc-Blende&Brillouin}
	\end{figure}

	The external orbital of the cation are $s$ for the cation (4$d^{10}$5$s^2$ for Cd,  3$d^{10}$4$s^2$ for Zn) and $p$ for the anion (4$d^{10}$5$s^2$5$p^4$ for Te). Considering a N unit crystal, it then contain 8N valence electron, coming from the $s$ and $p$ levels of the ions. The $s$ and $p$ orbital of these atoms hybridize to form 8 levels, 4 bonding and 4 anti-bonding.
	
	The lowest band of the bonding levels, coming from $s$ orbitals, will be filled by 2N valence electron. 6N will be taken to fill the three higher energy bonding band, formed by the hybridization of $p$ orbitals. Those bonding states form the valence band. At higher energy, the anti-bonding states form the conduction band. Since all the electron available are used to fill the valence band, the conduction band is empty in the ground state. The lower energy band of the conduction band are formed by the anti-symmetric combination of the $s$ orbitals. At higher energy, the anti-symmetric hybridization of $p$ orbitals form three other bands.

	Introducing the spin-orbit interaction, the conduction band, formed by the hybridazation of $s$ orbitals, is of $\Gamma_6$ (spherical) symmetry at the center of the Brillouin zone, two-fold degenerated, with an orbital momentum (spin) $\sigma = 1/2$. In a similar fashion, the valence band will be split into to band: a first one of $\Gamma_8$ symmetry, with a spin $J = 3/2$, four-fold degenerated ; and the second one, at lower energy, of $\Gamma_7$ symmetry, with a spin $1/2$, two-fold degenerated. The splitting $\Gamma_7-\Gamma_8$ is of $\Delta_{SO} \simeq 0.9$ eV in II-VI semiconductor.

	\begin{figure}[h!]
	\begin{center}
		\includegraphics[width=10cm]{../FillingPicture.png}
	\end{center}
	\caption{CdTe/ZnTe band structures}
	\label{BandStruct}
	\end{figure}
	
	The whole CdTe band structure is presented on Fig.~\ref{BandStruct}. One can note that CdTe is a direct gap semiconductor: the highest energy point of the valence band correspond to the lowest energy point of the conduction band, in $\Gamma$. As we move away from this point, the valence band into two branches: one with small curvature, meaning a high effective mass for the carriers on it, is called the heavy-hole (hh) band, while the one presenting the highest curvature and smallest effective mass is called the light-hole (lh) band.
	
	One way to understand this evolution is to apply the $\mathbf{k}.\mathbf{p}$ approximation, as proposed by Kane in 1957~\cite{KaneBandkp}. This model gives an estimation of the electronic band structure starting from the exact solution and energy of the Schrödinger equation at the center of the Brillouin. The hamiltonian to resolve is then :	
	\begin{align}
		\left(\frac{p^2}{2m_0} + U(\mathbf{r})\right)\psi_{n,\mathbf{k}} = E_{n,\mathbf{k}} \psi_{n,\mathbf{k}}
	\end{align}
with $U(\mathbf{r})$ the potential of the crystal and $\psi_{n,\mathbf{k}}$ the Bloch wave, separated between a periodic part $u_{n,\mathbf{k}} (\mathbf{r})$ and plane-wave part $e^{i\mathbf{k}.\mathbf{r}}$ as follow :
	\begin{align}
		\psi_{n,\mathbf{k}} = u_{n,\mathbf{k}} (\mathbf{r}) e^{i\mathbf{k}.\mathbf{r}}
	\end{align}

	Neglecting the $\Gamma_7$ band at lower energy, we solve this hamiltonian for carrier on the $\Gamma_6$ and $\Gamma_8$ bands~\cite{ClaireTh}. The $z$-axis is defined along the growth direction of the semiconductor and chosen as the quantization axis. We then find the energy:
	\begin{align}
		\begin{array}{l}
		E_c (k_z) = E_c + \frac{\hbar^2 k_z^2}{2m_c} \\
		E_{v, \pm\frac{1}{2}} (k_z) = E_v - \frac{\hbar^2 k_z^2}{2m_{lh}} \\
		E_{v, \pm\frac{3}{2}} (k_z) = E_v + \frac{\hbar^2 k^2_z}{2m_{0}}
		\end{array}	
	\end{align}
with $E_c$ ($E_v$) the energy of the conduction band (respectively, the valence band), $m_c$ the effective mass of the carrier on the conduction and $m_{lh}$ the effective mass of the light hole. One can see that the splitting of the valence band separate the carrier with a spin $J_z = \pm \frac{3}{2}$ (hh) from the one with a spin $J_z = \pm \frac{1}{2}$ (lh). However, the neglecting of the bands other than $\Gamma_6$ and $\Gamma_8$ lead to a positive curvature for the hh. To correct this problem, we would have to take into account higher energy conduction band, which will repel the hh band and give it its negative curvature.

	Another solution to have the matrix describing the $\Gamma_8$ band is to use symmetry consideration. Luttinger showed in 1956~\cite{LuttingerHam} that the only Hamiltonian fulfilling the cubic symmetry is:
	\begin{align}
	\label{LuttHamil}
		{\cal H}_L = - \frac{h^2}{2m_0}\left(\gamma_1 k^2 I_4 - 2 \gamma_2 \sum_i k_i^2 \left(J_i^2 - \frac{1}{3} J^2\right) - 2 \gamma_3 (k_x k_y (J_x J_y + J_y J_x) + c.p.) \right)
	\end{align}
with $\gamma_1$, $\gamma_2$ and $\gamma_3$ the Luttinger parameters, $I_4$ the 4 $\times$ 4 identity matrix, $\mathbf{k}$ a vector of the Brillouin zone, $\mathbf{J}$ the orbital momentum operator with $J_x$, $J_y$ and $J_z$ being 4 $\times$ 4 matric satisfying $[J_x, J_y] = iJ_z$ and circular permutation, and $c.p$ standing for "circular permutation". 
	This hamiltonian can be simplified using the parameters:
	\begin{align}
		\begin{array}{l}
			A = \gamma_1 + \frac{5}{2} \gamma_2 \\
			B = 2 \gamma_2 \\
			C = 2(\gamma_3 - \gamma_2)
		\end{array}
	\end{align}
Using these, the Luttinger hamiltonian can be rewritten:
	\begin{align}
		{\cal H}_L = - \frac{h^2}{2m_0}(Ak^2 I_4 - B(\mathbf{k}.\mathbf{J})^2 + C(k_x k_y (J_x J_y + J_y J_x) + c.p.))
	\end{align}
The $B$-term lift the degeneracy of the $\Gamma_8$ band into two sub-bands as shown above, and is invariant under arbitrary rotations. The $C$-term describes the warping of the valence band.

	In the spherical approximation, the Luttinger hamiltonian has two eigenvalues:
	\begin{align}
		\begin{array}{l}
			E_{hh} = - \frac{\hbar^2 k^2}{2m_0 (A-2.25B)^{-1}} = - \frac{\hbar^2 k^2}{2m_0 (\gamma_1 - 2 \gamma_2 )^{-1}} = - \frac{\hbar^2 k^2}{2m_{hh}} \\
			E_{lh} = - \frac{\hbar^2 k^2}{2m_0 (A-0.25B)^{-1}} = - \frac{\hbar^2 k^2}{2m_0 (\gamma_1 + 2 \gamma_2 )^{-1}} = - \frac{\hbar^2 k^2}{2m_{lh}}
		\end{array}
	\end{align}
We find back the value of the effective mass for the lh, along with a value for the hh. The hh band also presents here a negative curvature, as expected.

	The parameters and carriers effective masses are given in the Tab.~\ref{Param}.
	
	\begin{table} \centering
		\setlength\extrarowheight{2pt}
		\label{Param}
		\begin{tabular}{m{3cm}|m{3cm}|m{3cm}}
			\hline \hline
			 & \multicolumn{1}{c|}{CdTe} & \multicolumn{1}{c}{ZnTe} \\
			\hline \hline
			$E_g$ & 1606 meV & 2391 meV \\
			\hline
			$\epsilon_r$ & 10.6 & 9.7 \\
			\hline
			$a_0$ & 6.48 \AA & 6.10 \AA \\
			\hline
			$\Delta_{SO}$ & 0.90 eV & 0.91 eV \\
			\hline
			$\gamma_1$ & 4.8 & 4.07 \\
			\hline
			$\gamma_2$ & 1.5 & 0.78 \\
			\hline
			$\gamma_3$ & 1.9 & 1.59 \\
			\hline
			$m_{hh, z}$ & 0.556 & 0.398 \\
			\hline
			$m_{hh, \bot}$ & 0.159 & 0.206 \\
			\hline
			$m_{lh, z}$ & 0.128 & 0.178 \\
			\hline
			$m_{lh, \bot}$ & 0.303 & 0.303 \\
			\hline
			$m_e$ & 0.096 & 0.116 \\
			\hline
		\end{tabular}
		\caption{Physical parameters for CdTe and ZnTe.}
	\end{table}

		\subsection{Lattice mismatch and the Bir-Pikus Hamiltonian \label{BPSec}}
		
	ZnTe crystal has a lattice parameter of $a_{ZnTe} = $6.10 \AA, while CdTe one is of $a_{CdTe} = $6.48 \AA. This lattice mismatch results in stress in a CdTe layer grown on a ZnTe substrate:
		\begin{align}
			\epsilon_{\bot} = \frac{a_{ZnTe} - a_{CdTe}}{a_{CdTe}} = -5.8\%
		\end{align}
		
		In order to represent this strain and see their effect on the band, especially the $\Gamma_8$ band, we need to define a hamiltonian representing them. These strains deform the structure, so let's begin the representation with an volume $V = (x\mathbf{u_x} + y\mathbf{u_y} + z\mathbf{u_z}$, with $(\mathbf{u_x}, \mathbf{u_y}, \mathbf{u_z})$ an ortho-normalized basis. This volume will transform into another one $V' = (x\mathbf{u_x'} + y\mathbf{u_y'} + z\mathbf{u_z'})$, where:
		\begin{align}
			\begin{array}{l}							
				\mathbf{u_x'} = (1 + \epsilon_{xx}')\mathbf{u_x} + \epsilon_{xy}' \mathbf{u_y} + \epsilon_{xz}' \mathbf{u_z} \\
				\mathbf{u_y'} = \epsilon_{yx}' \mathbf{u_x} + (1 + \epsilon_{yy}')\mathbf{u_y} + \epsilon_{yz}' \mathbf{u_z} \\
				\mathbf{u_z'} = \epsilon_{zx}' \mathbf{u_x} + \epsilon_{zy}' \mathbf{u_y} + (1 + \epsilon_{zz}')\mathbf{u_z}
			\end{array}
		\end{align}
		
	$\epsilon_{ij}'$ represents an expansion of the vector $i$ in the direction $j$. They are small deformation of the lattice, so we choose $|\epsilon_{ij}'| \ll 1$. Such transformations can be decomposed in a symmetric part and an antisymetric one. We note $\overline{\overline{\epsilon}}$ the symmetric part, called the strain tensor, defined such as:
	\begin{align}
		\epsilon_{ ii} &= \epsilon_{ii}' \\
		\epsilon_{ij} &= \frac{1}{2} \left(\epsilon_{ij}' + \epsilon_{ji}'\right)
	\end{align}

	In the linear regime, the strain tensor $\overline{\overline{\epsilon}}$ is proportional to the stress tensor $\overline{\overline{\sigma}}$, where $\sigma_{ij}$ describe a force parallel to $i$ applied on a surface perpendicular to $j$. Therefore, $\sigma_{ii}$ will describe an elongation or compression stress, while $\sigma_{ij}$ ($i\neq j$) represents a shear stress. Since these tensor are symmetric, we can reduce the number of coefficient from nine to six: $\sigma_{xx}$, $\sigma_{yy}$, $\sigma_{zz}$, $\sigma_{xy} = \sigma_{yx}$, $\sigma_{xz} = \sigma_{zx}$ and $\sigma_{yz} = \sigma_{zy}$. Therefore, in the linear regime and for a cubic crystal, we can write the Hooke's law:
	\begin{align}
	\label{HookeLaw}
		\begin{bmatrix}
			\sigma_{xx} \\
			\sigma_{yy} \\
			\sigma_{zz} \\
			\sigma_{xy} \\
			\sigma_{xz} \\
			\sigma_{yz}
		\end{bmatrix}
		=
		\begin{bmatrix}
			C_{11} & C_{12} & C_{12} & 0 & 0 & 0 \\
			C_{12} & C_{11} & C_{12} & 0 & 0 & 0 \\
			C_{12} & C_{12} & C_{11} & 0 & 0 & 0 \\
			0 & 0 & 0 & 2C_{44} & 0 & 0 \\
			0 & 0 & 0 & 0 & 2C_{44} & 0 \\
			0 & 0 & 0 & 0 & 0 & 2C_{44}
		\end{bmatrix}
		\begin{bmatrix}
			\epsilon_{xx} \\
			\epsilon_{yy} \\
			\epsilon_{zz} \\
			\epsilon_{xy} \\
			\epsilon_{xz} \\
			\epsilon_{yz}
		\end{bmatrix}
	\end{align}
Since $x$, $y$ and $z$ are physically equivalent, as well as $xy$, $xz$ and $yz$, only two diagonal coefficient are needed, $C_{11}$ and $C_{44}$. These coefficient coupling strains in a direction to a force in the same direction are obviously positives.

	When the considered cube is compressed in one direction (e.g. $\epsilon_{zz} < 0$), it will expand in the other direction in order to minimize elastic energy ($\epsilon_{xx}$, $\epsilon_{yy} > 0$ in the example). If we don't allow strain in these other directions ($\epsilon_{xx} = \epsilon_{yy} = 0$), a stress in the x and y directions had to be applied to keep the cube from expanding in these directions ($\sigma_{xx}$, $\sigma_{yy} < 0$ in the example). We can therefore physically expect $C_{12} > 0$.
	
	The strain hamiltonian can be constructed noticing that the strain tensor $\overline{\overline{\epsilon}}$ induces a shift in the bands energy, and that any $\epsilon_{ij}$ has the same symmetry as $k_i k_j$. The hamiltonian should then be formarly identical to the Luttinger hamiltonian. In the $\Gamma_8$ subspace, we can then use the Luttinger Hamiltonian, written in Eq.~\ref{LuttHamil}, replacing the $k_i k_j$ by $\epsilon_{ij}$. We obtain the Bir-Pikus Hamiltonian by replacing the $\gamma_j$ parameters by the Bir-Pikus parameters $a_{\nu}$, $b_{\nu}$ and $d_{\nu}$~\cite{BirPikus}:
	\begin{align}
	\label{BPHamil}
		{\cal H}_{BP} = a_{\nu} \epsilon I_4 + b_{\nu} \sum_i \epsilon_{ii} \left(J_i^2 - \frac{1}{3}J^2\right) + \frac{d_{\nu}}{\sqrt{3}}(\epsilon_{xy} (J_x J_y + J_y J_x) + c.p)
	\end{align}
with $\epsilon = Tr(\overline{\overline{\epsilon}}) = \epsilon_{xx} + \epsilon_{yy}+ \epsilon_{zz}$.

	The $a_{\nu}$ term, called the hydrostatic term, is to shift the $\Gamma_8$ energy. The $b_{\nu}$ term represents the shear strain. In case of non-equal $\epsilon_{ii}$, its effect is to lift up the two $\Gamma_8$ sub-bands as did a $k\neq 0$ in the Luttinger hamiltonian. The $d_{\nu}$ term, the pure shear strain (i.e $\epsilon_{ij}$ with $i \neq j$), has the same effect on the $\Gamma_8$ band.
	
	One can notice that the Bir-Pikus hamiltonian is completely independant from $\mathbf{k}$, meaning that the band hamiltonian of a strain semiconductor is simply the sum of the sum of the Luttinger hamiltonian ${\cal H}_L$ (Eq.~\ref{LuttHamil}) and the Bir-Pikus hamiltonian ${\cal H}_{BP}$ (Eq.~\ref{BPHamil}).
	
	Let see how this apply to a CdTe layer deposited on a ZnTe layer. We define $z$ as the growth direction. As shown at the begin of this part, CdTe and ZnTe have a lattice mismatch of 5.8\%. Since both crystallize in a cubic lattice, the strain is the same in the $x$ and $y$ direction. We can then write the strain in the $xy$ plane:
	\begin{align}
	\label{epsperp}
		\epsilon_{xx} = \epsilon_{yy} = \epsilon_{\bot} = \frac{a_{ZnTe} - a_{CdTe}}{a_{CdTe}}
	\end{align}
In the $z$ direction, however, no stress apply: the crystal is free to expand in this direction in order to reduce the elastic energy. Therefore, we can write $\sigma_{zz} = 0$ and, according to Hooke's law in Eq.~\ref{HookeLaw}:
	\begin{align}
		\begin{array}{rl}
			\sigma_{zz} &= C_{12} \epsilon_{xx} + C_{12} \epsilon_{yy} + C_{11} \epsilon_{zz} \\
						&= 0
		\end{array}		
	\end{align}
Using equality \ref{epsperp}, we can then deduce:
	\begin{align}
	\label{epsz}
		\epsilon_{zz} = - \frac{2C_{12}}{C_{11}} \epsilon_{\bot} = - \frac{2C_{12}}{C_{11}} \frac{a_{ZnTe} - a_{CdTe}}{a_{CdTe}}
	\end{align}
	
	Since we grow CdTe over a ZnTe substrate, the CdTe lattice is compressed in the plane, i.e. $\epsilon_{\bot} < 0$. Since $C_{11}$, $C_{12} > 0$ and $\epsilon_{\bot} < 0$ for CdTe over ZnTe (see Eq.~\ref{epsperp}, one can easily deduce that $\epsilon_{zz} > 0$. In the hypothesis of no defect created by the lattice mismatch, all the other strain terms are equal to zero. We can then decompose this strain into two component: a hydrostatic part describing the volume variation without breaking the cubic symmetry, and a shear part introducing an anisotropy, breaking this symmetry:
	\begin{align}
		\overline{\overline{\epsilon_{hyd}}} &= \frac{1}{3}(\epsilon_{xx} + \epsilon_{yy} + \epsilon_{zz})I_3 \\
		\overline{\overline{\epsilon_{sh}}} &= \overline{\overline{\epsilon}} - \overline{\overline{\epsilon_{hyd}}}
	\end{align}
	
	One can notice that $Tr(\overline{\overline{\epsilon_{hybd}}}) = Tr(\overline{\overline{\epsilon}}) = \epsilon$. Since in the case of a hydrostatic compression, such as what is the case with CdTe over ZnTe, $\epsilon_{hybd} < 0$, we then have $\epsilon < 0$ and, according to the Bir-Pikus hamiltonian (Eq.~\ref{BPHamil}), the gap of CdTe increase. For CdTe, Bir-Pikus parameter are $a_{\nu} = 0.91$ eV, $b_{\nu} = 0.99$ eV and $d_{\nu} = 2.76$ eV~\cite{CdTeDefPot}.
	
	Seeing that $\epsilon_{ij} = 0$ for $i\neq j$, we can rewrite the Bir-Pikus hamiltonian without the shear strain term. Moreover, since $J^2 = J_x^2 + J_y^2 + J_z^2$ and that $\epsilon_{xx} = \epsilon_{yy} = \epsilon_{\bot}$, we can simplify this hamiltonian to:
	\begin{align}
		{\cal H}_{BP, biax} = a_{\nu} \epsilon I_4 + \frac{b_{\nu}}{3}(\epsilon_{\bot} - \epsilon_{zz})(J_x^2 + J_y^2 - 2 J_z^2)
	\end{align}
And, since we are in the valence band with $J = \frac{3}{2}$ and $J_x^2 + J_y^2 + J_z^2 = J(J+1)I_4$, we can simplify the Bir-Pikus hamiltonian to its final form in the case of biaxal strain:
	\begin{align}
		{\cal H}_{BP, biax} = \left(a_{\nu} \epsilon + \frac{5}{4} b_{\nu} (\epsilon_{\nu} - \epsilon_{zz})\right)I_4 - b_{\nu} (\epsilon_{\nu} - \epsilon_{zz})J_z^2
	\end{align}
	
	Using Eq.~\ref{epsperp} and \ref{epsz}, we can easily calculate $\epsilon_{\bot} - \epsilon_{z}$. Since $J_z|n\rangle = n|n\rangle$, we find:
	\begin{align}
		\begin{array}{rl}
			E_{\pm \frac{3}{2}} - E_{\pm \frac{1}{2}} &= - 2b_{\nu} \left(1 + \frac{2C_{12}}{C_{11}} \right) \frac{a_{ZnTe} - a_{CdTe}}{a_{CdTe}} \\
													&= 2b_{\nu} \left(1 + \frac{2C_{12}}{C_{11}} \right) \frac{a_{CdTe} - a_{ZnTe}}{a_{CdTe}}
		\end{array}
	\end{align}
We find that, in a fully strained CdTe layer over a ZnTe substrate, the hh band is 300 meV above the lh one. In first approximation, we can then neglect the lh contribution in these nanostructures.
		
		\subsection{3D confinement: the quantum dot\label{e-hIntQD}}
		
		Embedding a semiconductor in another one of larger creates trap for carrier, confining them in one or multiple. CdTe conduction band (resp. valence band) is a lower (resp. higher) energy than ZnTe ones, creating such a trap. Using the procedure described in Chap.~\ref{Growth}, we can create nanometre size island of CdTe in a ZnTe lattice, effectively confining electron in all three direction, acting like an 3D well for the free carriers. This confinement lead to a quantization of the carriers energy level and a discretization of the optical properties. This confinement being analogue to the Coulomb interaction of an isolated atom, such a structure is often dubbed "artificial atom". However, the interaction between the hole and the electron cannot be overlook, adding Coulomb interaction between the particule and the quasi-particule. It consists of an attractive term, shifting energy levels, and an exchange interaction (discussed in Chap.~\ref{ExchIntQD}). Moreover, the hole being the absence of an electron, its energy, charge, spin, orbital momentum, $\mathbf{k}$ and mass are, by definition, opposite to the missing electron. The electron-hole system has a hydrogen-like behaviour and is called an exciton.

	\begin{figure}[h!]
	\begin{center}
		\includegraphics[width=10cm]{../FillingPicture.png}
	\end{center}
	\caption{Dots STM images}
	\label{STM}
	\end{figure}

	Before going to the exchange part, we will see the optical properties of this system. In order to do so, we develop the carrier wave-function on the Bloch states:
	\begin{align}
		\Psi (\mathbf{r}) = \sum_{n,k} c_{n,k} u _{n,k} (\mathbf{r}) e^{i\mathbf{k}.\mathbf{r}}
	\end{align}
Since we are in a confined environment, we can only consider the states around $\mathbf{k} = 0$. Since we consider the band extrema, we neglect the inter-band wave function mixing in a first time and use the effective-mass approximation. We can then limit the expansion of Bloch state to an expansion on the $u _{n,0} (\mathbf{r}) e^{i\mathbf{k}.\mathbf{r}}$, with $n = \Gamma_6$ for the conduction band and $n = \Gamma_8$ for the valence. We can then write:
	\begin{align}
		\Psi_c (\mathbf{r}) & \simeq \sum_{k} c_{k} u _{\Gamma_6,0} (\mathbf{r}) e^{i\mathbf{k}.\mathbf{r}} = F_e (\mathbf{r}) u _{\Gamma_6,0} \label{eEnvFunc} \\
		\Psi_v (\mathbf{r}) & \simeq \sum_{J_z = \{\pm \frac{3}{2}, \pm \frac{1}{2}\}, k} c_{J_z,k} u _{\Gamma_8,J_z} (\mathbf{r}) e^{i\mathbf{k}.\mathbf{r}} = \sum_{J_z = \{\pm \frac{3}{2}, \pm \frac{1}{2}\}} F_{J_z} (\mathbf{r}) u _{\Gamma_8,J_z} \label{hEnvFunc}
	\end{align}
with $F_e (\mathbf{r}) = \sum_{n,k} c_{k} e^{i\mathbf{k}.\mathbf{r}}$ the electron envelop function and $F_{J_z} (\mathbf{r}) = c_{J_z,k} e^{i\mathbf{k}.\mathbf{r}}$, $J_z = \{\pm \frac{3}{2}, \pm \frac{1}{2}\}$ the hole envelop functions.

	The effective mass approximation allow us to replace the periodic crystal potential and the free-electron kinetic energy by the effective Hamiltonian representing the band extrema, using $m_e$ for the conduction band 
and ${\cal H}_L + {\cal H}_{BP}$ for the top of the valence band. Considering the effective mass is the same in CdTe and ZnTe, we can now work with the simple picture of an effective mass carrier with the envelop function defined in Eqs.~\ref{eEnvFunc} and \ref{hEnvFunc}, trapped in a potential $V_e (\mathbf{r})$ for the conduction band or $V_h (\mathbf{r})$ for the valence band, creating by the band offset between the two semiconductors. We write the Schrödinger equations for these particle:
	\begin{align}
		\left(\frac{\hbar^2}{2m_e}\Delta \right)F_e (\mathbf{r}) + V_e (\mathbf{r}) F_e (\mathbf{r}) &= EF_e (\mathbf{r}) \label{eSchroding} \\
		(\tilde{{\cal H}}_L + \tilde{{\cal H}}_{BP} + V_h (\mathbf{r})) 
			\begin{pmatrix}
				F_{+ \frac{3}{2}} (\mathbf{r}) \\
				F_{+ \frac{1}{2}} (\mathbf{r}) \\
				F_{- \frac{1}{2}} (\mathbf{r}) \\
				F_{- \frac{3}{2}} (\mathbf{r})
			\end{pmatrix}
			&= E
			\begin{pmatrix}
				F_{+ \frac{3}{2}} (\mathbf{r}) \\
				F_{+ \frac{1}{2}} (\mathbf{r}) \\
				F_{- \frac{1}{2}} (\mathbf{r}) \\
				F_{- \frac{3}{2}} (\mathbf{r})
			\end{pmatrix}
			\label{hSchroding}
	\end{align}
with $\tilde{{\cal H}}_L$ and $\tilde{{\cal H}}_{BP}$ the hole hamiltonians, opposite to the electron hamiltonians defined in Eq.~\ref{LuttHamil} and \ref{BPHamil}. In $\tilde{{\cal H}}_L$, the $k$-terms transform into a gradient of the envelop function with the form $i\nabla$. For simplicity, the $\tilde{}$ will be dropped in the next equations. The derivation of the effective mass approximation can be found in reference \cite{LuttElecMotion}.

	In order to analytically solve these hamiltonian, another approximation is necessary. As discussed in section \ref{BandStruct}, ${\cal H}_{L}$ couple the hh and lh through its non-diagonal terms, while ${\cal H}_{BP}$ lift their degeneracy, as shown in Sec.~\ref{BPSec}. Considering the presence of biaxial strain, we can neglect the non-diagonal terms of ${\cal H}_{L}$ in regard to the action of ${\cal H}_{BP}$. This is the heavy hole approximation, un coupling the four differential equations defined in Eq.~\ref{hSchroding}. Only the ground states $|\pm \frac{3}{2}\rangle$ are considered, with the effective mass given by the diagonal term of ${\cal H}_{L}$, noted $m_{h, \parallel}$ in the plane and $m_{h, z}$ along the growth axis.
	
	In the general case, we still cannot solve this problem. However, it is however possible for some chosen potential. Let's consider a lens like quantum dot, with a radius in the $xy$ plane, noted $\rho$, much larger than it height $L_z$. We can therefore define two different harmonic oscillator of potential: a 2D oscillator $V_{c,v} (\rho)$ in the plane, and a 1D oscillator $V_{c,v} (z)$ along the growth axis:
	\begin{align}
		V_{c,v} (\rho) = 4 \Delta E_{c,v} \frac{\rho ^2}{L_z^2} \\
		V_{c,v} (z) = 4 \Delta E_{c,v} \frac{z^2}{L_z^2}
	\end{align}
with $\Delta E_{c,v}$ the difference of conduction (resp. valence) band energy between the semiconductor. The potential of the whole quantum will the be $V_{c,v} (\mathbf{r}) = V_{c,v} (\rho) + V_{c,v} (z)$. Separating the potential in those two parts means we are searching for solution of the form $F(z, \rho, \theta) = \chi (z) \phi_{n,m} (\rho, \theta)$, with $\theta$ the angle between the position vector and the $x$ axis.

	We write the characteristic spatial width and characteristic frequency of the 2D harmonic oscillator felt by the hole:
	\begin{align}
		\Sigma_{\rho}^h &= \sqrt{\frac{\hbar}{m_{h, \parallel} \omega_{\rho}^{h}}} \\
		\omega_{\rho}^{h} &= \sqrt{\frac{8\Delta E_{v}}{m_{h, \parallel} L_{\rho}^2}}
	\end{align}
We can write the same equality along $z$ replacing $\rho$ by $z$ and $m_{h, \parallel}$ by $m_{h,z}$. The same can be done for electron, replacing the  $m_{h, \parallel}$ or $m_{h,z}$ by $m_e$ and $E_v$ by $E_c$.

	We can find in textbook such as ref.~\cite{MecQBasvant} the solution of a harmonic oscillator from which we can deduce the solution for the ground state (GS) and the first two degenerated excited states. The first excited state is found to have an angular momentum $l_z = \pm 1$, and is then noted $Exc, \pm 1$. The envelop functions and energy are then found to be:
	\begin{align}
			& F_{c,v}^{GS} (z, \rho, \theta) = \frac{1}{(\sqrt{\pi}\Sigma_z)^{\frac{1}{2}}} \exp \left(- \frac{z^2}{2 \Sigma_z^2}\right) \frac{1}{(\sqrt{\pi}\Sigma_{\rho})^{\frac{1}{2}}} \exp \left(- \frac{\rho^2}{2 \Sigma_{\rho}^2}\right) \\
			& E_{e,h}^{GS} = \hbar \frac{\omega_z^{e,h} + \omega_{\rho}^{e,h}}{2} \\
		\nonumber \\
			& F_{c,v}^{Exc, \pm 1} (z, \rho, \theta) = \frac{1}{(\sqrt{\pi}\Sigma_z)^{\frac{1}{2}}} \exp \left(- \frac{z^2}{2 \Sigma_z^2}\right) \frac{1}{(\sqrt{\pi}\Sigma_{\rho})^{\frac{1}{2}}} \exp \left(- \frac{\rho^2}{2 \Sigma_{\rho}^2}\right) \frac{\rho}{\sigma_{\rho}} e^{\pm i\theta} \\
			& E_{e,h}^{Exc, \pm 1} = \hbar \frac{\omega_z^{e,h} + 3 \omega_{\rho}^{e,h}}{2}
	\end{align}
We see that this energy level are quantified in a way looking like an isolated atom, as pointed earlier. In reference to the atomic notation, the ground state, lower energy level, is noted $S$ and the two first degenerated level are noted $P$, even though atomic p-states usually are 3 fold degenerated.

	One remarkable feature of the envelop functions is that both GS and the two first excited states present the same envelop along the $z$ axis. The cause is directly the symmetry of the QD: since $L_z \ll L_{\rho}$, $\omega_z^{e,h} \gg \omega_{\rho}^{e,h}$, and since $E_{osc.\ harmo.} = (n+\frac{1}{2})\hbar \omega$, the next possible envelop function along the $z$ axis is at higher energy than the next one in the plane. This geometry is also responsible for the 2 fold degeneracy of the $P$-states.
	
	Both the GS and the excited states are once again degenerated due to the spin of the electron and the hole. The electron is in the conduction band with the $\Gamma_6$ symmetry: it can then take the value $\sigma_z = \pm \frac{1}{2}$ (noted $|\uparrow\rangle$ for $+\frac{1}{2}$ and $|\downarrow\rangle$ for $-\frac{1}{2}$). Since we are in the hh approximation, considering the lh are high enough energy to be negligible, the hole spin can only take the values $J_z = \pm \frac{3}{2}$ (noted $|\Uparrow\rangle$ for $+\frac{3}{2}$ and $|\Downarrow\rangle$ for $-\frac{3}{2}$). As pointed ahead, the hole is defined with the opposed characteristic of the missing electron (which may not be the one trapped in the QD). For instance, a hole $|\Downarrow\rangle$ corresponds to the absence of a valence electron $\Psi_v (\mathbf{r}) = u_{\Gamma_8, \frac{3}{2}} (\mathbf{r}) F_{\frac{3}{2}} (\mathbf{r})$.
	
	In order to find the optical properties of the quantum dot, we use the dipole approximation, giving the coupling to light as $H = -\frac{q}{m}\mathbf{p}.\mathbf{A}$, with $\mathbf{p}$ the momentum and $\mathbf{A}$ the vector potential. We can then determine the optical properties looking at the coupling through the $\mathbf{p}$ operator. In a QD, the light-matter interaction occur mainly through two processes: the absorption of a photon creating an exciton, and the recombination of an exciton emitting a photon. To model this, we consider the interband matrix element between the two electronic states $\Psi_c$ and $\Psi_v$, as written in Eqs.~\ref{eEnvFunc} and \ref{hEnvFunc}:
	\begin{align}
		|\langle \Psi_v | \mathbf{p} | \Psi_c \rangle |^2 = |\langle F_v | F_c \rangle |^2|\langle u_{\Gamma_8, J_z} | \mathbf{p} | u_{\Gamma_6, \sigma_z} \rangle |^2
	\end{align}
The first term is just the overlap of the envelop function, making sure the hole and the electron: a transition between a $P$ state of the valence band and a $S$ state in the conduction band is then forbidden.

	The second term, showing the interband matrix elements, depending only on the symmetry of the Bloch functions, will then draw the rule for the recombination. We write the spin states $|+\rangle$ and $|-\rangle$. Since, as shown in Sec.~\ref{BandStruct}, the conduction band is formed by $s$ atomic states, we can now note them:
	\begin{align}
		|u_{\Gamma_6, \uparrow}\rangle = |+\rangle \\
		|u_{\Gamma_6, \downarrow}\rangle = |-\rangle
	\end{align}
The valence, being formed by $p$ atomic states, has to also take the $|X\rangle$, $|Y\rangle$ and $|Z\rangle$ electronic states. We can then write the three eigenstates of the electronic part:
	\begin{align}
		|+1\rangle &= -\frac{|X\rangle + i|Y\rangle}{\sqrt{2}} \\
		|0\rangle &= |Z\rangle \\
		|-1\rangle &= \frac{|X\rangle - i|Y\rangle}{\sqrt{2}}
	\end{align}
We can now write the states of the conduction band by composing these electronic states to the spin states:
	\begin{align}
		|u_{\Gamma_8, +\frac{3}{2}}\rangle &= |+1\rangle|+\rangle \\
		|u_{\Gamma_8, +\frac{1}{2}}\rangle &= \sqrt{\frac{2}{3}}|0\rangle|+\rangle + \sqrt{\frac{2}{3}}|+1\rangle|-\rangle \\
		|u_{\Gamma_8, -\frac{1}{2}}\rangle &= \sqrt{\frac{2}{3}}|0\rangle|-\rangle + \sqrt{\frac{2}{3}}|-1\rangle|+\rangle \\
		|u_{\Gamma_8, -\frac{3}{2}}\rangle &= |-1\rangle|-\rangle
	\end{align}
Since we are working in the hh approximation, we neglect the contribution of the states $|u_{\Gamma_8, \pm \frac{1}{2}}\rangle$, calculating the interband matrix element only for $|u_{\Gamma_8, \pm \frac{3}{2}}\rangle$. Since $|+\rangle$ and $|-\rangle$ are orthogonal states, it is then clear that there is only two optically active transitions:
	\begin{itemize}
		\item between $|u_{\Gamma_6, \uparrow}\rangle$ and $|u_{\Gamma_8, +\frac{3}{2}}\rangle$ (hole $|\Downarrow\rangle$), coupled by $p_- = p_x - i p_y$, corresponding to $\sigma_-$ photon absorption or emission.
		\item between $|u_{\Gamma_6, \downarrow}\rangle$ and $|u_{\Gamma_8, -\frac{3}{2}}\rangle$ (hole $|\Uparrow\rangle$), coupled by $p_+ = p_x + i p_y$, corresponding to $\sigma_+$ photon absorption or emission.
	\end{itemize}
	
	We see thus that, for an exciton in the $S$ state of the QD, four configurations are possible. First, we have the bright states $|\uparrow \Downarrow \rangle$, with an angular momentum $X_z = \sigma_z + J_z = \frac{1}{2} - \frac{3}{2} = -1$, and $|\downarrow \Uparrow \rangle$, with an angular momentum $X_z = -\frac{1}{2} + \frac{3}{2} = +1$. Both of these states are optically active, meaning their transitions are permitted and they can recombine radiatively. Two other states can exists, called dark states: $\uparrow \Uparrow \rangle$, with an angular momentum $X_z = +2$, and $\downarrow \Downarrow \rangle$, with an angular momentum $X_z = -2$.
	
	Approximating the QD potential as harmonic usually overestimate the confinement, and thus the single-particle energy. But the wave-functions found in this chapter can still be used as trial wave-functions for variational calculations in other potential, in order to estimate the correct energy level.
	
	We discussed in this chapter about the neutral exciton ($X$), formed by a single electron-hole pair. However, several types of exciton can be observed in a quantum dot. First to consider are the charged excitons. In this case, a supplementary charge is injected in the QD in addition to the exciton, forming a hole-hole-electron ($X^+$) or hole-electron-electron ($X^-$) complex. It also happen that two exciton with opposed spins are trapped in a dot. This complex is called biexciton, noted $X^2$, and relax to leave a single neutral exciton in the QD. Charged biexciton and other multi-exciton complex also exist but are not discussed on the thesis. Even if the physics of each of this system is different, the selection rule devised in this chapter apply to all of them.

		\subsection{Valence band mixing}	
		
	\begin{figure}[h!]
	\begin{center}
		\includegraphics[width=10cm]{../FillingPicture.png}
	\end{center}
	\caption{Linear polarization of empty dot}
	\label{XAlLinPol}
	\end{figure}
		
	This single picture, however, don't represent accurately most of the experiment. In a lot of found dots, both $X$ and $X^2$ presents two peaks with linear polarization dependency, as presented in Fig.~\ref{XAlLinPol}. This picture could be explained by the long-range exchange interaction, presented in Sec.~\ref{ExchDMS}. However, it predicts an angle between the two linearly polarized states of $90^\circ$, while the observed angle is far from this value. Moreover, in charged species, the electron (resp. hole) interact with two holes (resp. electrons) with opposite spins, leading to an absence of exchange interaction. Therefore, in the hh approximation, neither $X^+$ nor $X^-$ should present linear polarization dependency. However, such a dependency can be observed in actual dots.
	
	However, we saw in Sec.~\ref{e-hIntQD} that the hh approximation supposed neglecting lh contribution in the Luttinger hamiltonian, through the presence of biaxial strain lifting their degeneracy with hh. However, we can see in the Bir-Pikus hamiltonian (Eq.~\ref{BPHamil}) that introducing an anisotropic distribution of strain ($\epsilon_{xx} \neq \epsilon_{yy}$ or $\epsilon_{ij} \neq 0$ with $i \neq j$) can mix hh states $J_z = \pm \frac{3}{2}$ and lh states $J_z = \mp \frac{1}{2}$. Such a mixing, called Valence Band Mixing (VBM) can also occur through the Luttinger hamltonian in an anisotropic dot ($L_x \neq L_y$), but it would be weaker than the one occurring through Bir-Pikus hamiltonian.

	Curabitur eget ipsum egestas dui viverra suscipit. Cras aliquet lacus vitae erat finibus semper. Nulla pharetra eget urna vitae sodales. Nunc faucibus velit lacus, nec ornare eros aliquet quis. Donec a orci nec sem pulvinar ultricies sit amet ut arcu. Nullam id vehicula enim, at tincidunt velit. Duis vestibulum lorem a molestie fringilla. Nullam tincidunt semper placerat. Donec nibh sem, ornare eget cursus ac, luctus sit amet eros. Phasellus eget interdum nisi. Donec mollis risus id lectus fringilla, et commodo risus iaculis. Donec at lacus sed nibh posuere posuere sit amet eget sapien. In dignissim, enim sit amet convallis fermentum, lacus nulla gravida tortor, non facilisis ex nisl sit amet augue. Maecenas eu enim condimentum, consectetur ligula vel, tincidunt nisl. Nam laoreet dictum volutpat. Donec at erat venenatis, ultrices lorem ac, vestibulum neque. 
	
	
	\section{Exchange interaction between carrier and magnetic atom}
	
		\subsection{Exchange interaction in Diluted Magnetic Semiconductors\label{ExchDMS}}
		
		Lorem ipsum dolor sit amet, consectetur adipiscing elit. Curabitur tortor quam, imperdiet quis facilisis sed, fringilla a quam. Cras ante odio, hendrerit ac ante nec, cursus imperdiet urna. Mauris convallis ultricies purus, nec condimentum erat bibendum vel. Aliquam erat volutpat. Pellentesque condimentum, eros a consequat accumsan, turpis sem euismod nisi, sed fringilla quam turpis sit amet erat. Mauris dictum odio sed nisi dapibus, et molestie mauris rutrum. Praesent convallis dolor in nibh blandit bibendum. Quisque sit amet arcu consectetur lorem luctus venenatis nec quis dui. Aliquam erat volutpat. Aenean auctor elit nec tristique dignissim. Nulla massa mi, efficitur semper ex id, pretium eleifend massa. Vivamus sit amet orci scelerisque, gravida est ut, vulputate odio.

	Curabitur eget ipsum egestas dui viverra suscipit. Cras aliquet lacus vitae erat finibus semper. Nulla pharetra eget urna vitae sodales. Nunc faucibus velit lacus, nec ornare eros aliquet quis. Donec a orci nec sem pulvinar ultricies sit amet ut arcu. Nullam id vehicula enim, at tincidunt velit. Duis vestibulum lorem a molestie fringilla. Nullam tincidunt semper placerat. Donec nibh sem, ornare eget cursus ac, luctus sit amet eros. Phasellus eget interdum nisi. Donec mollis risus id lectus fringilla, et commodo risus iaculis. Donec at lacus sed nibh posuere posuere sit amet eget sapien. In dignissim, enim sit amet convallis fermentum, lacus nulla gravida tortor, non facilisis ex nisl sit amet augue. Maecenas eu enim condimentum, consectetur ligula vel, tincidunt nisl. Nam laoreet dictum volutpat. Donec at erat venenatis, ultrices lorem ac, vestibulum neque.
	
		\subsection{Mn case}
		
		Lorem ipsum dolor sit amet, consectetur adipiscing elit. Curabitur tortor quam, imperdiet quis facilisis sed, fringilla a quam. Cras ante odio, hendrerit ac ante nec, cursus imperdiet urna. Mauris convallis ultricies purus, nec condimentum erat bibendum vel. Aliquam erat volutpat. Pellentesque condimentum, eros a consequat accumsan, turpis sem euismod nisi, sed fringilla quam turpis sit amet erat. Mauris dictum odio sed nisi dapibus, et molestie mauris rutrum. Praesent convallis dolor in nibh blandit bibendum. Quisque sit amet arcu consectetur lorem luctus venenatis nec quis dui. Aliquam erat volutpat. Aenean auctor elit nec tristique dignissim. Nulla massa mi, efficitur semper ex id, pretium eleifend massa. Vivamus sit amet orci scelerisque, gravida est ut, vulputate odio.

	Curabitur eget ipsum egestas dui viverra suscipit. Cras aliquet lacus vitae erat finibus semper. Nulla pharetra eget urna vitae sodales. Nunc faucibus velit lacus, nec ornare eros aliquet quis. Donec a orci nec sem pulvinar ultricies sit amet ut arcu. Nullam id vehicula enim, at tincidunt velit. Duis vestibulum lorem a molestie fringilla. Nullam tincidunt semper placerat. Donec nibh sem, ornare eget cursus ac, luctus sit amet eros. Phasellus eget interdum nisi. Donec mollis risus id lectus fringilla, et commodo risus iaculis. Donec at lacus sed nibh posuere posuere sit amet eget sapien. In dignissim, enim sit amet convallis fermentum, lacus nulla gravida tortor, non facilisis ex nisl sit amet augue. Maecenas eu enim condimentum, consectetur ligula vel, tincidunt nisl. Nam laoreet dictum volutpat. Donec at erat venenatis, ultrices lorem ac, vestibulum neque.
	
		\subsection{Cr case\label{CrDMS}}
		
		Lorem ipsum dolor sit amet, consectetur adipiscing elit. Curabitur tortor quam, imperdiet quis facilisis sed, fringilla a quam. Cras ante odio, hendrerit ac ante nec, cursus imperdiet urna. Mauris convallis ultricies purus, nec condimentum erat bibendum vel. Aliquam erat volutpat. Pellentesque condimentum, eros a consequat accumsan, turpis sem euismod nisi, sed fringilla quam turpis sit amet erat. Mauris dictum odio sed nisi dapibus, et molestie mauris rutrum. Praesent convallis dolor in nibh blandit bibendum. Quisque sit amet arcu consectetur lorem luctus venenatis nec quis dui. Aliquam erat volutpat. Aenean auctor elit nec tristique dignissim. Nulla massa mi, efficitur semper ex id, pretium eleifend massa. Vivamus sit amet orci scelerisque, gravida est ut, vulputate odio.

	Curabitur eget ipsum egestas dui viverra suscipit. Cras aliquet lacus vitae erat finibus semper. Nulla pharetra eget urna vitae sodales. Nunc faucibus velit lacus, nec ornare eros aliquet quis. Donec a orci nec sem pulvinar ultricies sit amet ut arcu. Nullam id vehicula enim, at tincidunt velit. Duis vestibulum lorem a molestie fringilla. Nullam tincidunt semper placerat. Donec nibh sem, ornare eget cursus ac, luctus sit amet eros. Phasellus eget interdum nisi. Donec mollis risus id lectus fringilla, et commodo risus iaculis. Donec at lacus sed nibh posuere posuere sit amet eget sapien. In dignissim, enim sit amet convallis fermentum, lacus nulla gravida tortor, non facilisis ex nisl sit amet augue. Maecenas eu enim condimentum, consectetur ligula vel, tincidunt nisl. Nam laoreet dictum volutpat. Donec at erat venenatis, ultrices lorem ac, vestibulum neque.
		
		\subsection{Effect of the confinement\label{ExchIntQD}}
		
		Lorem ipsum dolor sit amet, consectetur adipiscing elit. Curabitur tortor quam, imperdiet quis facilisis sed, fringilla a quam. Cras ante odio, hendrerit ac ante nec, cursus imperdiet urna. Mauris convallis ultricies purus, nec condimentum erat bibendum vel. Aliquam erat volutpat. Pellentesque condimentum, eros a consequat accumsan, turpis sem euismod nisi, sed fringilla quam turpis sit amet erat. Mauris dictum odio sed nisi dapibus, et molestie mauris rutrum. Praesent convallis dolor in nibh blandit bibendum. Quisque sit amet arcu consectetur lorem luctus venenatis nec quis dui. Aliquam erat volutpat. Aenean auctor elit nec tristique dignissim. Nulla massa mi, efficitur semper ex id, pretium eleifend massa. Vivamus sit amet orci scelerisque, gravida est ut, vulputate odio.

	Curabitur eget ipsum egestas dui viverra suscipit. Cras aliquet lacus vitae erat finibus semper. Nulla pharetra eget urna vitae sodales. Nunc faucibus velit lacus, nec ornare eros aliquet quis. Donec a orci nec sem pulvinar ultricies sit amet ut arcu. Nullam id vehicula enim, at tincidunt velit. Duis vestibulum lorem a molestie fringilla. Nullam tincidunt semper placerat. Donec nibh sem, ornare eget cursus ac, luctus sit amet eros. Phasellus eget interdum nisi. Donec mollis risus id lectus fringilla, et commodo risus iaculis. Donec at lacus sed nibh posuere posuere sit amet eget sapien. In dignissim, enim sit amet convallis fermentum, lacus nulla gravida tortor, non facilisis ex nisl sit amet augue. Maecenas eu enim condimentum, consectetur ligula vel, tincidunt nisl. Nam laoreet dictum volutpat. Donec at erat venenatis, ultrices lorem ac, vestibulum neque.	


	\section{Fine and hyperfine structure of a magnetic atom in II-VI semiconductor}
	
		\subsection{Mn atom in II-VI semiconductor}

		Mn in a lattice -> modification of orbital -> spin-orbit interaction. Magnetic anisotropy + anisotropy of strain. (Mn has nuclear spin 5/2 -> hyperfine interaction?)

	\begin{figure}[h!]
	\begin{center}
		\includegraphics[width=10cm]{../FillingPicture.png}
	\end{center}
	\caption{Mn in Zinc-Blend lattice}
	\label{MnInclusion}
	\end{figure}
		
		Lorem ipsum dolor sit amet, consectetur adipiscing elit. Curabitur tortor quam, imperdiet quis facilisis sed, fringilla a quam. Cras ante odio, hendrerit ac ante nec, cursus imperdiet urna. Mauris convallis ultricies purus, nec condimentum erat bibendum vel. Aliquam erat volutpat. Pellentesque condimentum, eros a consequat accumsan, turpis sem euismod nisi, sed fringilla quam turpis sit amet erat. Mauris dictum odio sed nisi dapibus, et molestie mauris rutrum. Praesent convallis dolor in nibh blandit bibendum. Quisque sit amet arcu consectetur lorem luctus venenatis nec quis dui. Aliquam erat volutpat. Aenean auctor elit nec tristique dignissim. Nulla massa mi, efficitur semper ex id, pretium eleifend massa. Vivamus sit amet orci scelerisque, gravida est ut, vulputate odio.

	\begin{figure}[h!]
	\begin{center}
		\includegraphics[width=10cm]{../FillingPicture.png}
	\end{center}
	\caption{Mn fine and hyperfine structure}
	\label{MnFineHyperfines}
	\end{figure}

	Curabitur eget ipsum egestas dui viverra suscipit. Cras aliquet lacus vitae erat finibus semper. Nulla pharetra eget urna vitae sodales. Nunc faucibus velit lacus, nec ornare eros aliquet quis. Donec a orci nec sem pulvinar ultricies sit amet ut arcu. Nullam id vehicula enim, at tincidunt velit. Duis vestibulum lorem a molestie fringilla. Nullam tincidunt semper placerat. Donec nibh sem, ornare eget cursus ac, luctus sit amet eros. Phasellus eget interdum nisi. Donec mollis risus id lectus fringilla, et commodo risus iaculis. Donec at lacus sed nibh posuere posuere sit amet eget sapien. In dignissim, enim sit amet convallis fermentum, lacus nulla gravida tortor, non facilisis ex nisl sit amet augue. Maecenas eu enim condimentum, consectetur ligula vel, tincidunt nisl. Nam laoreet dictum volutpat. Donec at erat venenatis, ultrices lorem ac, vestibulum neque.
	
		\subsection{Cr atom in II-VI semiconductor\label{CrSemiCon}}

	\begin{figure}[h!]
	\begin{center}
		\includegraphics[width=10cm]{../FillingPicture.png}
	\end{center}
	\caption{Cr in Zinc-Blend lattice}
	\label{CrInclusion}
	\end{figure}
	
		Lorem ipsum dolor sit amet, consectetur adipiscing elit. Curabitur tortor quam, imperdiet quis facilisis sed, fringilla a quam. Cras ante odio, hendrerit ac ante nec, cursus imperdiet urna. Mauris convallis ultricies purus, nec condimentum erat bibendum vel. Aliquam erat volutpat. Pellentesque condimentum, eros a consequat accumsan, turpis sem euismod nisi, sed fringilla quam turpis sit amet erat. Mauris dictum odio sed nisi dapibus, et molestie mauris rutrum. Praesent convallis dolor in nibh blandit bibendum. Quisque sit amet arcu consectetur lorem luctus venenatis nec quis dui. Aliquam erat volutpat. Aenean auctor elit nec tristique dignissim. Nulla massa mi, efficitur semper ex id, pretium eleifend massa. Vivamus sit amet orci scelerisque, gravida est ut, vulputate odio.

	\begin{figure}[h!]
	\begin{center}
		\includegraphics[width=10cm]{../FillingPicture.png}
	\end{center}
	\caption{Atomic configuration in Jahn-Teller effect + three minima}
	\label{Jahn-Teller}
	\end{figure}

	\begin{figure}[h!]
	\begin{center}
		\includegraphics[width=10cm]{../FillingPicture.png}
	\end{center}
	\caption{Degeneracy breaking under Jahn-Teller effect}
	\label{DegenBreakJT}
	\end{figure}

	Curabitur eget ipsum egestas dui viverra suscipit. Cras aliquet lacus vitae erat finibus semper. Nulla pharetra eget urna vitae sodales. Nunc faucibus velit lacus, nec ornare eros aliquet quis. Donec a orci nec sem pulvinar ultricies sit amet ut arcu. Nullam id vehicula enim, at tincidunt velit. Duis vestibulum lorem a molestie fringilla. Nullam tincidunt semper placerat. Donec nibh sem, ornare eget cursus ac, luctus sit amet eros. Phasellus eget interdum nisi. Donec mollis risus id lectus fringilla, et commodo risus iaculis. Donec at lacus sed nibh posuere posuere sit amet eget sapien. In dignissim, enim sit amet convallis fermentum, lacus nulla gravida tortor, non facilisis ex nisl sit amet augue. Maecenas eu enim condimentum, consectetur ligula vel, tincidunt nisl. Nam laoreet dictum volutpat. Donec at erat venenatis, ultrices lorem ac, vestibulum neque.

	\begin{figure}[h!]
	\begin{center}
		\includegraphics[width=10cm]{../FillingPicture.png}
	\end{center}
	\caption{Strain effect on ground state + degeneracy breaking by this symetry reduction}
	\label{StrainGS}
	\end{figure}

	\begin{figure}[h!]
	\begin{center}
		\includegraphics[width=10cm]{../FillingPicture.png}
	\end{center}
	\caption{Overall energy structure (with +/- 2 which doesn't luminesce)}
	\label{EnerStruct}
	\end{figure}
	
	
	\section{A simple example: the X-Mn system}
	
	Lorem ipsum dolor sit amet, consectetur adipiscing elit. Curabitur tortor quam, imperdiet quis facilisis sed, fringilla a quam. Cras ante odio, hendrerit ac ante nec, cursus imperdiet urna. Mauris convallis ultricies purus, nec condimentum erat bibendum vel. Aliquam erat volutpat. Pellentesque condimentum, eros a consequat accumsan, turpis sem euismod nisi, sed fringilla quam turpis sit amet erat. Mauris dictum odio sed nisi dapibus, et molestie mauris rutrum. Praesent convallis dolor in nibh blandit bibendum. Quisque sit amet arcu consectetur lorem luctus venenatis nec quis dui. Aliquam erat volutpat. Aenean auctor elit nec tristique dignissim. Nulla massa mi, efficitur semper ex id, pretium eleifend massa. Vivamus sit amet orci scelerisque, gravida est ut, vulputate odio.

	\begin{figure}[h!]
	\begin{center}
		\includegraphics[width=10cm]{../FillingPicture.png}
	\end{center}
	\caption{QD spectra 0 Mn, 1 Mn, 2 Mn}
	\label{MnSpectra}
	\end{figure}

	Curabitur eget ipsum egestas dui viverra suscipit. Cras aliquet lacus vitae erat finibus semper. Nulla pharetra eget urna vitae sodales. Nunc faucibus velit lacus, nec ornare eros aliquet quis. Donec a orci nec sem pulvinar ultricies sit amet ut arcu. Nullam id vehicula enim, at tincidunt velit. Duis vestibulum lorem a molestie fringilla. Nullam tincidunt semper placerat. Donec nibh sem, ornare eget cursus ac, luctus sit amet eros. Phasellus eget interdum nisi. Donec mollis risus id lectus fringilla, et commodo risus iaculis. Donec at lacus sed nibh posuere posuere sit amet eget sapien. In dignissim, enim sit amet convallis fermentum, lacus nulla gravida tortor, non facilisis ex nisl sit amet augue. Maecenas eu enim condimentum, consectetur ligula vel, tincidunt nisl. Nam laoreet dictum volutpat. Donec at erat venenatis, ultrices lorem ac, vestibulum neque.

	\begin{figure}[h!]
	\begin{center}
		\includegraphics[width=10cm]{../FillingPicture.png}
	\end{center}
	\caption{Mn energy level in a QD}
	\label{MnLevel}
	\end{figure}
		
	
\printbibliography

\end{document}