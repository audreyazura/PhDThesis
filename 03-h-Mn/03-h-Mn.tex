\documentclass[a4paper,12pt]{report}

\usepackage[utf8]{inputenc}
\usepackage[T1]{fontenc}
\usepackage{array}
\usepackage{amsmath}
\usepackage[english]{babel}
\usepackage{graphicx}
\usepackage[a4paper]{geometry}
\usepackage[colorlinks=true,urlcolor=blue,linkcolor=blue]{hyperref}
\usepackage{url}
\usepackage[nottoc,numbib]{tocbibind}
\usepackage{color}
\usepackage{epstopdf}
\usepackage{xcolor}
\usepackage[backend=biber,style=phys]{biblatex}

\addbibresource{../Bibliography.bib}

\makeatletter
	\renewcommand{\thechapter}{\Roman{chapter}}
\makeatother

\begin{document}

\chapter{Coherent dynamics of Mn-doped positively charged quantum dots}

	\section{Mn in a II-VI positively charged quantum dot}

	Cf Optical control of the spin of a magnetic atom in a semiconductor QD, L. Besombes et. al., Sept 2014
	
		\subsection{Quantum dot charged state selection\label{ChargeSelec}}
		
		Lorem ipsum dolor sit amet, consectetur adipiscing elit. Curabitur tortor quam, imperdiet quis facilisis sed, fringilla a quam. Cras ante odio, hendrerit ac ante nec, cursus imperdiet urna. Mauris convallis ultricies purus, nec condimentum erat bibendum vel. Aliquam erat volutpat. Pellentesque condimentum, eros a consequat accumsan, turpis sem euismod nisi, sed fringilla quam turpis sit amet erat. Mauris dictum odio sed nisi dapibus, et molestie mauris rutrum. Praesent convallis dolor in nibh blandit bibendum. Quisque sit amet arcu consectetur lorem luctus venenatis nec quis dui. Aliquam erat volutpat. Aenean auctor elit nec tristique dignissim. Nulla massa mi, efficitur semper ex id, pretium eleifend massa. Vivamus sit amet orci scelerisque, gravida est ut, vulputate odio.
		
	\begin{figure}[h!]
	\begin{center}
		\includegraphics[width=10cm]{../FillingPicture.png}
	\end{center}
	\caption{Sample with Schottky gate and micro-lens}
	\label{Schottky}
	\end{figure}

	Curabitur eget ipsum egestas dui viverra suscipit. Cras aliquet lacus vitae erat finibus semper. Nulla pharetra eget urna vitae sodales. Nunc faucibus velit lacus, nec ornare eros aliquet quis. Donec a orci nec sem pulvinar ultricies sit amet ut arcu. Nullam id vehicula enim, at tincidunt velit. Duis vestibulum lorem a molestie fringilla. Nullam tincidunt semper placerat. Donec nibh sem, ornare eget cursus ac, luctus sit amet eros. Phasellus eget interdum nisi. Donec mollis risus id lectus fringilla, et commodo risus iaculis. Donec at lacus sed nibh posuere posuere sit amet eget sapien. In dignissim, enim sit amet convallis fermentum, lacus nulla gravida tortor, non facilisis ex nisl sit amet augue. Maecenas eu enim condimentum, consectetur ligula vel, tincidunt nisl. Nam laoreet dictum volutpat. Donec at erat venenatis, ultrices lorem ac, vestibulum neque.

	\begin{figure}[h!]
	\begin{center}
		\includegraphics[width=10cm]{../FillingPicture.png}
	\end{center}
	\caption{Example of charge variation and selection of the charged state}
	\label{StateSelection}
	\end{figure}
		
		\subsection{Energy structure}
		
		Cf XplusMnRes.pptx to detail the e-Mn levels
		
	\begin{figure}[h!]
	\begin{center}
		\includegraphics[width=10cm]{../FillingPicture.png}
	\end{center}
	\caption{X+-Mn spectra and linear polarization}
	\label{Spectra&LinPolar}
	\end{figure}
		
		Lorem ipsum dolor sit amet, consectetur adipiscing elit. Curabitur tortor quam, imperdiet quis facilisis sed, fringilla a quam. Cras ante odio, hendrerit ac ante nec, cursus imperdiet urna. Mauris convallis ultricies purus, nec condimentum erat bibendum vel. Aliquam erat volutpat. Pellentesque condimentum, eros a consequat accumsan, turpis sem euismod nisi, sed fringilla quam turpis sit amet erat. Mauris dictum odio sed nisi dapibus, et molestie mauris rutrum. Praesent convallis dolor in nibh blandit bibendum. Quisque sit amet arcu consectetur lorem luctus venenatis nec quis dui. Aliquam erat volutpat. Aenean auctor elit nec tristique dignissim. Nulla massa mi, efficitur semper ex id, pretium eleifend massa. Vivamus sit amet orci scelerisque, gravida est ut, vulputate odio.
		
	\begin{figure}[h!]
	\begin{center}
		\includegraphics[width=10cm]{../FillingPicture.png}
	\end{center}
	\caption{Mn in charged QD simple energy structure}
	\label{SimpleEnerStruct}
	\end{figure}

	Curabitur eget ipsum egestas dui viverra suscipit. Cras aliquet lacus vitae erat finibus semper. Nulla pharetra eget urna vitae sodales. Nunc faucibus velit lacus, nec ornare eros aliquet quis. Donec a orci nec sem pulvinar ultricies sit amet ut arcu. Nullam id vehicula enim, at tincidunt velit. Duis vestibulum lorem a molestie fringilla. Nullam tincidunt semper placerat. Donec nibh sem, ornare eget cursus ac, luctus sit amet eros. Phasellus eget interdum nisi. Donec mollis risus id lectus fringilla, et commodo risus iaculis. Donec at lacus sed nibh posuere posuere sit amet eget sapien. In dignissim, enim sit amet convallis fermentum, lacus nulla gravida tortor, non facilisis ex nisl sit amet augue. Maecenas eu enim condimentum, consectetur ligula vel, tincidunt nisl. Nam laoreet dictum volutpat. Donec at erat venenatis, ultrices lorem ac, vestibulum neque.
	
	\begin{figure}[h!]
	\begin{center}
		\includegraphics[width=10cm]{../FillingPicture.png}
	\end{center}
	\caption{Energy structure of h-Mn/X+-Mn with valence band mixing, perturbative two holes, with the linear polarization as an example (experiment + model)}
	\label{CompleteEnerStruct}
	\end{figure}
	
	Curabitur eget ipsum egestas dui viverra suscipit. Cras aliquet lacus vitae erat finibus semper. Nulla pharetra eget urna vitae sodales. Nunc faucibus velit lacus, nec ornare eros aliquet quis. Donec a orci nec sem pulvinar ultricies sit amet ut arcu. Nullam id vehicula enim, at tincidunt velit. Duis vestibulum lorem a molestie fringilla. Nullam tincidunt semper placerat. Donec nibh sem, ornare eget cursus ac, luctus sit amet eros. Phasellus eget interdum nisi. Donec mollis risus id lectus fringilla, et commodo risus iaculis. Donec at lacus sed nibh posuere posuere sit amet eget sapien. In dignissim, enim sit amet convallis fermentum, lacus nulla gravida tortor, non facilisis ex nisl sit amet augue. Maecenas eu enim condimentum, consectetur ligula vel, tincidunt nisl. Nam laoreet dictum volutpat. Donec at erat venenatis, ultrices lorem ac, vestibulum neque.
	
	\begin{figure}[h!]
	\begin{center}
		\includegraphics[width=10cm]{../FillingPicture.png}
	\end{center}
	\caption{Linear polarization modelization with variation of parameter to show influence.}
	\label{LinPolModelMn}
	\end{figure}
		
		\subsection{Optical $\lambda$-level identification}	
		
		Lorem ipsum dolor sit amet, consectetur adipiscing elit. Curabitur tortor quam, imperdiet quis facilisis sed, fringilla a quam. Cras ante odio, hendrerit ac ante nec, cursus imperdiet urna. Mauris convallis ultricies purus, nec condimentum erat bibendum vel. Aliquam erat volutpat. Pellentesque condimentum, eros a consequat accumsan, turpis sem euismod nisi, sed fringilla quam turpis sit amet erat. Mauris dictum odio sed nisi dapibus, et molestie mauris rutrum. Praesent convallis dolor in nibh blandit bibendum. Quisque sit amet arcu consectetur lorem luctus venenatis nec quis dui. Aliquam erat volutpat. Aenean auctor elit nec tristique dignissim. Nulla massa mi, efficitur semper ex id, pretium eleifend massa. Vivamus sit amet orci scelerisque, gravida est ut, vulputate odio.

	\begin{figure}[h!]
	\begin{center}
		\includegraphics[width=10cm]{../FillingPicture.png}
	\end{center}
	\caption{Luminescence under laser scan (map)}
	\label{ResPLE}
	\end{figure}

	Curabitur eget ipsum egestas dui viverra suscipit. Cras aliquet lacus vitae erat finibus semper. Nulla pharetra eget urna vitae sodales. Nunc faucibus velit lacus, nec ornare eros aliquet quis. Donec a orci nec sem pulvinar ultricies sit amet ut arcu. Nullam id vehicula enim, at tincidunt velit. Duis vestibulum lorem a molestie fringilla. Nullam tincidunt semper placerat. Donec nibh sem, ornare eget cursus ac, luctus sit amet eros. Phasellus eget interdum nisi. Donec mollis risus id lectus fringilla, et commodo risus iaculis. Donec at lacus sed nibh posuere posuere sit amet eget sapien. In dignissim, enim sit amet convallis fermentum, lacus nulla gravida tortor, non facilisis ex nisl sit amet augue. Maecenas eu enim condimentum, consectetur ligula vel, tincidunt nisl. Nam laoreet dictum volutpat. Donec at erat venenatis, ultrices lorem ac, vestibulum neque.
	
	\begin{figure}[h!]
	\begin{center}
		\includegraphics[width=10cm]{../FillingPicture.png}
	\end{center}
	\caption{Identification of $\lambda$-systems with each $\lambda$-system drawn}
	\label{LambdaLevem}
	\end{figure}
			
	
	\section{Spin dynamics under resonant excitation}
	
		Cf article 2016/01
	
		\subsection{Cycling and escaping the $\lambda$-level system}

		Mn in a lattice -> modification of orbital -> spin-orbit interaction. Magnetic anisotropy + anisotropy of strain. (Mn has nuclear spin 5/2 -> hyperfine interaction?)

	\begin{figure}[h!]
	\begin{center}
		\includegraphics[width=10cm]{../FillingPicture.png}
	\end{center}
	\caption{Isolated $\lambda$-system with the loop (excitation, recombination, relaxation to initial state)}
	\label{LambdLoop}
	\end{figure}
		
		Lorem ipsum dolor sit amet, consectetur adipiscing elit. Curabitur tortor quam, imperdiet quis facilisis sed, fringilla a quam. Cras ante odio, hendrerit ac ante nec, cursus imperdiet urna. Mauris convallis ultricies purus, nec condimentum erat bibendum vel. Aliquam erat volutpat. Pellentesque condimentum, eros a consequat accumsan, turpis sem euismod nisi, sed fringilla quam turpis sit amet erat. Mauris dictum odio sed nisi dapibus, et molestie mauris rutrum. Praesent convallis dolor in nibh blandit bibendum. Quisque sit amet arcu consectetur lorem luctus venenatis nec quis dui. Aliquam erat volutpat. Aenean auctor elit nec tristique dignissim. Nulla massa mi, efficitur semper ex id, pretium eleifend massa. Vivamus sit amet orci scelerisque, gravida est ut, vulputate odio.

	\begin{figure}[h!]
	\begin{center}
		\includegraphics[width=10cm]{../FillingPicture.png}
	\end{center}
	\caption{Pumping experiment on each $\lambda$-system}
	\label{AllPumpB0}
	\end{figure}

	Curabitur eget ipsum egestas dui viverra suscipit. Cras aliquet lacus vitae erat finibus semper. Nulla pharetra eget urna vitae sodales. Nunc faucibus velit lacus, nec ornare eros aliquet quis. Donec a orci nec sem pulvinar ultricies sit amet ut arcu. Nullam id vehicula enim, at tincidunt velit. Duis vestibulum lorem a molestie fringilla. Nullam tincidunt semper placerat. Donec nibh sem, ornare eget cursus ac, luctus sit amet eros. Phasellus eget interdum nisi. Donec mollis risus id lectus fringilla, et commodo risus iaculis. Donec at lacus sed nibh posuere posuere sit amet eget sapien. In dignissim, enim sit amet convallis fermentum, lacus nulla gravida tortor, non facilisis ex nisl sit amet augue. Maecenas eu enim condimentum, consectetur ligula vel, tincidunt nisl. Nam laoreet dictum volutpat. Donec at erat venenatis, ultrices lorem ac, vestibulum neque.
	
	\begin{figure}[h!]
	\begin{center}
		\includegraphics[width=10cm]{../FillingPicture.png}
	\end{center}
	\caption{Autocorrelation experiment on each $\lambda$-system}
	\label{AllAutocorB0}
	\end{figure}
	
	Curabitur eget ipsum egestas dui viverra suscipit. Cras aliquet lacus vitae erat finibus semper. Nulla pharetra eget urna vitae sodales. Nunc faucibus velit lacus, nec ornare eros aliquet quis. Donec a orci nec sem pulvinar ultricies sit amet ut arcu. Nullam id vehicula enim, at tincidunt velit. Duis vestibulum lorem a molestie fringilla. Nullam tincidunt semper placerat. Donec nibh sem, ornare eget cursus ac, luctus sit amet eros. Phasellus eget interdum nisi. Donec mollis risus id lectus fringilla, et commodo risus iaculis. Donec at lacus sed nibh posuere posuere sit amet eget sapien. In dignissim, enim sit amet convallis fermentum, lacus nulla gravida tortor, non facilisis ex nisl sit amet augue. Maecenas eu enim condimentum, consectetur ligula vel, tincidunt nisl. Nam laoreet dictum volutpat. Donec at erat venenatis, ultrices lorem ac, vestibulum neque.
	
		\subsection{Relaxation mechanism}
	
	Lorem ipsum dolor sit amet, consectetur adipiscing elit. Curabitur tortor quam, imperdiet quis facilisis sed, fringilla a quam. Cras ante odio, hendrerit ac ante nec, cursus imperdiet urna. Mauris convallis ultricies purus, nec condimentum erat bibendum vel. Aliquam erat volutpat. Pellentesque condimentum, eros a consequat accumsan, turpis sem euismod nisi, sed fringilla quam turpis sit amet erat. Mauris dictum odio sed nisi dapibus, et molestie mauris rutrum. Praesent convallis dolor in nibh blandit bibendum. Quisque sit amet arcu consectetur lorem luctus venenatis nec quis dui. Aliquam erat volutpat. Aenean auctor elit nec tristique dignissim. Nulla massa mi, efficitur semper ex id, pretium eleifend massa. Vivamus sit amet orci scelerisque, gravida est ut, vulputate odio.

	\begin{figure}[h!]
	\begin{center}
		\includegraphics[width=10cm]{../FillingPicture.png}
	\end{center}
	\caption{Autocorrelation evolution under magnetic field and power variation - experimental result}
	\label{AutocorExpBPw}
	\end{figure}

	Curabitur eget ipsum egestas dui viverra suscipit. Cras aliquet lacus vitae erat finibus semper. Nulla pharetra eget urna vitae sodales. Nunc faucibus velit lacus, nec ornare eros aliquet quis. Donec a orci nec sem pulvinar ultricies sit amet ut arcu. Nullam id vehicula enim, at tincidunt velit. Duis vestibulum lorem a molestie fringilla. Nullam tincidunt semper placerat. Donec nibh sem, ornare eget cursus ac, luctus sit amet eros. Phasellus eget interdum nisi. Donec mollis risus id lectus fringilla, et commodo risus iaculis. Donec at lacus sed nibh posuere posuere sit amet eget sapien. In dignissim, enim sit amet convallis fermentum, lacus nulla gravida tortor, non facilisis ex nisl sit amet augue. Maecenas eu enim condimentum, consectetur ligula vel, tincidunt nisl. Nam laoreet dictum volutpat. Donec at erat venenatis, ultrices lorem ac, vestibulum neque.

	\begin{figure}[h!]
	\begin{center}
		\includegraphics[width=10cm]{../FillingPicture.png}
	\end{center}
	\caption{Pumping evolution under magnetic field and power variation - experimental result}
	\label{PumpExpBPw}
	\end{figure}
	
	Lorem ipsum dolor sit amet, consectetur adipiscing elit. Curabitur tortor quam, imperdiet quis facilisis sed, fringilla a quam. Cras ante odio, hendrerit ac ante nec, cursus imperdiet urna. Mauris convallis ultricies purus, nec condimentum erat bibendum vel. Aliquam erat volutpat. Pellentesque condimentum, eros a consequat accumsan, turpis sem euismod nisi, sed fringilla quam turpis sit amet erat. Mauris dictum odio sed nisi dapibus, et molestie mauris rutrum. Praesent convallis dolor in nibh blandit bibendum. Quisque sit amet arcu consectetur lorem luctus venenatis nec quis dui. Aliquam erat volutpat. Aenean auctor elit nec tristique dignissim. Nulla massa mi, efficitur semper ex id, pretium eleifend massa. Vivamus sit amet orci scelerisque, gravida est ut, vulputate odio.

	\begin{figure}[h!]
	\begin{center}
		\includegraphics[width=10cm]{../FillingPicture.png}
	\end{center}
	\caption{Autocorrelation evolution under magnetic field and power variation - model}
	\label{AutocorModBPw}
	\end{figure}

	Curabitur eget ipsum egestas dui viverra suscipit. Cras aliquet lacus vitae erat finibus semper. Nulla pharetra eget urna vitae sodales. Nunc faucibus velit lacus, nec ornare eros aliquet quis. Donec a orci nec sem pulvinar ultricies sit amet ut arcu. Nullam id vehicula enim, at tincidunt velit. Duis vestibulum lorem a molestie fringilla. Nullam tincidunt semper placerat. Donec nibh sem, ornare eget cursus ac, luctus sit amet eros. Phasellus eget interdum nisi. Donec mollis risus id lectus fringilla, et commodo risus iaculis. Donec at lacus sed nibh posuere posuere sit amet eget sapien. In dignissim, enim sit amet convallis fermentum, lacus nulla gravida tortor, non facilisis ex nisl sit amet augue. Maecenas eu enim condimentum, consectetur ligula vel, tincidunt nisl. Nam laoreet dictum volutpat. Donec at erat venenatis, ultrices lorem ac, vestibulum neque.

	\begin{figure}[h!]
	\begin{center}
		\includegraphics[width=10cm]{../FillingPicture.png}
	\end{center}
	\caption{Pumping evolution under magnetic field and power variation - model}
	\label{PumpModBPw}
	\end{figure}
	

	\section{Influence of the strain anisotropy}
		
	Lorem ipsum dolor sit amet, consectetur adipiscing elit. Curabitur tortor quam, imperdiet quis facilisis sed, fringilla a quam. Cras ante odio, hendrerit ac ante nec, cursus imperdiet urna. Mauris convallis ultricies purus, nec condimentum erat bibendum vel. Aliquam erat volutpat. Pellentesque condimentum, eros a consequat accumsan, turpis sem euismod nisi, sed fringilla quam turpis sit amet erat. Mauris dictum odio sed nisi dapibus, et molestie mauris rutrum. Praesent convallis dolor in nibh blandit bibendum. Quisque sit amet arcu consectetur lorem luctus venenatis nec quis dui. Aliquam erat volutpat. Aenean auctor elit nec tristique dignissim. Nulla massa mi, efficitur semper ex id, pretium eleifend massa. Vivamus sit amet orci scelerisque, gravida est ut, vulputate odio.
		
	\begin{figure}[h!]
	\begin{center}
		\includegraphics[width=10cm]{../FillingPicture.png}
	\end{center}
	\caption{Energy structure with |3, +1> and |3, -1>, and |2, +1> and |2, -1> coupled by E}
	\label{LambdaMixed}
	\end{figure}

	Curabitur eget ipsum egestas dui viverra suscipit. Cras aliquet lacus vitae erat finibus semper. Nulla pharetra eget urna vitae sodales. Nunc faucibus velit lacus, nec ornare eros aliquet quis. Donec a orci nec sem pulvinar ultricies sit amet ut arcu. Nullam id vehicula enim, at tincidunt velit. Duis vestibulum lorem a molestie fringilla. Nullam tincidunt semper placerat. Donec nibh sem, ornare eget cursus ac, luctus sit amet eros. Phasellus eget interdum nisi. Donec mollis risus id lectus fringilla, et commodo risus iaculis. Donec at lacus sed nibh posuere posuere sit amet eget sapien. In dignissim, enim sit amet convallis fermentum, lacus nulla gravida tortor, non facilisis ex nisl sit amet augue. Maecenas eu enim condimentum, consectetur ligula vel, tincidunt nisl. Nam laoreet dictum volutpat. Donec at erat venenatis, ultrices lorem ac, vestibulum neque.

	\begin{figure}[h!]
	\begin{center}
		\includegraphics[width=10cm]{../FillingPicture.png}
	\end{center}
	\caption{Experiment configuration |3, +1>  + Polarization decline and polar rate}
	\label{hMnPolarRate}
	\end{figure}
		
		Lorem ipsum dolor sit amet, consectetur adipiscing elit. Curabitur tortor quam, imperdiet quis facilisis sed, fringilla a quam. Cras ante odio, hendrerit ac ante nec, cursus imperdiet urna. Mauris convallis ultricies purus, nec condimentum erat bibendum vel. Aliquam erat volutpat. Pellentesque condimentum, eros a consequat accumsan, turpis sem euismod nisi, sed fringilla quam turpis sit amet erat. Mauris dictum odio sed nisi dapibus, et molestie mauris rutrum. Praesent convallis dolor in nibh blandit bibendum. Quisque sit amet arcu consectetur lorem luctus venenatis nec quis dui. Aliquam erat volutpat. Aenean auctor elit nec tristique dignissim. Nulla massa mi, efficitur semper ex id, pretium eleifend massa. Vivamus sit amet orci scelerisque, gravida est ut, vulputate odio.

	\begin{figure}[h!]
	\begin{center}
		\includegraphics[width=10cm]{../FillingPicture.png}
	\end{center}
	\caption{Schema of the QD with spin and magnetic field orientation, and action of the magnetic field on the spin.}
	\label{QDMagField}
	\end{figure}

	Curabitur eget ipsum egestas dui viverra suscipit. Cras aliquet lacus vitae erat finibus semper. Nulla pharetra eget urna vitae sodales. Nunc faucibus velit lacus, nec ornare eros aliquet quis. Donec a orci nec sem pulvinar ultricies sit amet ut arcu. Nullam id vehicula enim, at tincidunt velit. Duis vestibulum lorem a molestie fringilla. Nullam tincidunt semper placerat. Donec nibh sem, ornare eget cursus ac, luctus sit amet eros. Phasellus eget interdum nisi. Donec mollis risus id lectus fringilla, et commodo risus iaculis. Donec at lacus sed nibh posuere posuere sit amet eget sapien. In dignissim, enim sit amet convallis fermentum, lacus nulla gravida tortor, non facilisis ex nisl sit amet augue. Maecenas eu enim condimentum, consectetur ligula vel, tincidunt nisl. Nam laoreet dictum volutpat. Donec at erat venenatis, ultrices lorem ac, vestibulum neque.

	\begin{figure}[h!]
	\begin{center}
		\includegraphics[width=10cm]{../FillingPicture.png}
	\end{center}
	\caption{Polarization rate evolution in B(x and z) and simulation}
	\label{hMnPolarRateB}
	\end{figure}
	
	
\printbibliography

\end{document}