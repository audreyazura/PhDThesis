\documentclass[a4paper,12pt]{report}

\usepackage[utf8]{inputenc}
\usepackage[T1]{fontenc}
\usepackage{array}
\usepackage{amsmath}
\usepackage[english]{babel}
\usepackage{graphicx}
\usepackage[a4paper]{geometry}
\usepackage[colorlinks=true,urlcolor=blue,linkcolor=blue]{hyperref}
\usepackage{url}
\usepackage[nottoc,numbib]{tocbibind}
\usepackage{color}
\usepackage{epstopdf}
\usepackage{xcolor}
\usepackage[backend=biber,style=phys]{biblatex}
\usepackage{lipsum}
\usepackage[capbesideposition={right,center}]{floatrow}
\usepackage{bm}

\addbibresource{../Bibliography.bib}

\makeatletter
	\renewcommand{\thechapter}{\Roman{chapter}}
\makeatother

\floatsetup[table]{style=plaintop}

\begin{document}

\chapter{Coherent dynamics of Mn-doped positively charged quantum dots}

	\section{Mn in a II-VI positively charged quantum dot}

	Cf Optical control of the spin of a magnetic atom in a semiconductor QD, L. Besombes et. al., Sept 2014
		
		\subsection{Spin structure of a positively charged Mn doped quantum dot}
		
		We saw in Sec.~\ref{MnSemiCon} that the exchange interaction between the carriers and the Mn spin lift the degeneracy of the Mn spin states. The recombination lines of the excitonic structures are then  each split into six, each line in a given polarization corresponding to a give Mn spin state. Applying a positive bias voltage on the sample via a Schottky gate (see Sec.~\ref{ChargedSample}), a single hole is trapped in the magnetic QDs and only the emission of the positively charged exciton is observed (Fig.~\ref{hMnspectra} (a)). The PL of such a QD is presented on Fig.~\ref{hMnspectra} (b).
		
	\begin{figure}[h!]
	{\caption{(a) Color scale plot of the PL intensity of the studied Mn doped QD inserted in Schottky structure showing the emission of the neutral (X-Mn) and positively charged (X$^+$-Mn) exciton as a function of energy and bias voltage. (b) PL of the Mn-doped QD under a positive bias voltage of V=5.5V.}\label{hMnspectra}}
	{\begin{center}
		\includegraphics[width=12cm]{Pictures/DotPres.eps}
	\end{center}}
	\end{figure}
	
	A Mn atom in a strained self-assembled CdTe QD exhibits a fine structure dominated by a weak magnetic anisotropy with an easy axis along the QD axis. Neglecting the tetrahedral crystal field of the CdTe matrix, this fine structure is described by the effective spin Hamiltonian:
\begin{align}
\label{MnCF}
{\cal H}_{Mn,CF}=D_0S^2_z+E(S_x^2-S_y^2)
\end{align}
with $D_0$ depicting the effect of the biaxial strain and $E$ describing the anisotropy of the strain in the plane of the QD. It was shown that the anisotropy of strain was essential to understand the absence of pumping for Mn in strain-free quantum dots~\cite{OptControlSpin} and was thus include here to keep generality. We will study more in details its effect on Sec.~\ref{SpinDyn}.

	When a hole is trapped in a QD containing a single Mn, the spin structure is controlled by the Mn-hole exchange interaction that reads:
	\begin{align}
		{\cal H}_{hMn}^{ex}=I_{hMn}\mathbf{S}\cdot\mathbf{J}
	\end{align}
with $I_{hMn}$ the exchange energy between the hole and the Mn ($S=5/2$) and $\mathbf{J}$ the hole spin operator. In the presence of heavy-hole/light-hole mixing, $\mathbf{J}$, represented in the basis of the two low energy heavy-hole states, is related to the Pauli matrices by $J_z=3/2\tau_z$ and $J_{\pm}= \xi \tau_{\pm}$ with $\xi=-2\sqrt{3}e^{-2i\theta}\rho_c/\Delta_{lh}$. $\rho_c$ is the coupling energy between heavy holes and light holes separated by an effective energy splitting $\Delta_{lh}$. $\theta$ is the angle relative to the [110] axis of the principal axis of the anisotropy (shape and/or strain) responsible for the heavy-hole/light-hole mixing \cite{SingleExcSpectro,VBMArticle}. For a weak valence band mixing, the Mn-hole energy levels are mainly controlled by $I_{hMn}S_zJ_z$ and form a spin ladder with a quantization axis along the QDs growth direction. These states are labelled $|S_z,J_z\rangle$.

	The valence-band mixing also affect the system. However, as well as the interaction with the carrier introduced big enough energy gap between the energy to keep most of them to be affected in the X-Mn system presented in Sec.~\ref{XMn}, the interaction with a the hole injected in the QD keep most of the h-Mn states to be affected by the VBM. Once again, only the hole states coupled the Mn spin states $S_z = \pm 1/2$ are mixed, resulting in linear polarization of the central peaks.

	\begin{figure}[h!]
		\begin{center}
			\includegraphics[width=14.8cm]{Pictures/Recomb.png}
		\end{center}
		\caption{Electron-Mn spin states for each $|M,M_z\rangle$. For each $M$, the $\sigma-$ (red) and $\sigma+$ (blue) probability is highlighted. This probability is directly linked to the intensity of each peak. In the center, the different possible recombination path for $M=3$ and $M=2$ are presented. A schema of the resulting spectra is drawn below.}
		\label{Recomb}
	\end{figure}
	
	In the QD excited state, an exciton is injected in it. Since a single hole is already inside, the X$^+$-Mn is formed. The two holes paired up with anti-parallel spins, and thus the system is dominated by the electron-Mn exchange interaction:
	\begin{align}
		{\cal H}_{eMn}^{ex} = I_{eMn} \mathbf{S}.\bm{\upsigma}
	\end{align}
with, as usual, $\bm{\upsigma}$ as the electron and $I_{eMn}$ as the e-Mn exchange energy. This interaction is isotrope, and results in a ground state septuplet of total spin $M = 3$ and a fivefold degenerated manifold of total spin $M = 2$, for a total of twelve electron-Mn states. In the absence of perturbations, the energy levels of each of those manifold are degenerated. Each of those states are labelled $|M,M_z\rangle$ These states are presented in Fig.~\ref{Recomb}, as well as the result of recombination from each of the manifold. One can see that each state emit at a given energy, resulting in a twelve peaks spectra. Each electron-Mn state correspond to a given peak energy and polarization.

	Since the two holes have opposed spins, their exchange interaction with the Mn atom only introduce a perturbation to the electron-Mn wavefunction~\cite{CarInSpinSplit,BiexFinStruct,LucienSFD}. This can be represented for one hole by an effective spin Hamiltonian ${\cal H}_{scat}=-\eta S_z^2$ with $\eta>0$. This perturbation has to be taken into account twice for X$^+$-Mn where two holes interact with the Mn. This perturbation affects the energy of the optical recombination of X$^+$-Mn to the Mn-hole ground state and can be observed in the emission spectra \cite{LucienSFD}.

	\begin{figure}[h!]
	\begin{center}
		\includegraphics[width=12cm]{Pictures/Spinstructv2.png}
	\end{center}
	\caption{(a) Energy levels of the ground (h-Mn) and excited ($X^+$-Mn) states as a function of their angular momentum (M$_z$). The levels in dotted lines corresponds to the h-Mn states $|-1/2\rangle|\Uparrow\rangle$ and $|+1/2\rangle|\Downarrow\rangle$ coupled by the valence band mixing. Optical recombination towards these levels leads to the linearly polarized lines observed in (b). (b) Experimental (left) and calculated (right) color-scale plot of the linear polarization dependence of the PL of X$^+$-Mn at B = 0 T (top) and B$_\perp$ = 0.42 T (bottom). The parameters used in the calculation are listed in Table~\ref{paraQD}.}
	\label{CompleteEnerStruct}
	\end{figure}
	
	We then have the ground state hamiltonian:
	\begin{align}
		\begin{array}{rl}
			{\cal H}_{hMn} =& {\cal H}_{hMn}^{ex} + {\cal H}_{CF, Mn} \\
							=& I_{hMn} \mathbf{S}.\mathbf{J} + D_0 S_z^2 + E (S_x^2 - S_y^2)
		\end{array}
	\end{align}
and, in the excited state, X$^+$-Mn:
	\begin{align}
		\begin{array}{rl}
			{\cal H}_{X^+-Mn} =& {\cal H}_{eMn}^{ex} + 2 {\cal H}_{scat} + {\cal H}_{CF, Mn} \\
							=& I_{eMn} \mathbf{S}.\bm{\upsigma} - 2\eta S_z^2 + D_0 S_z^2 + E (S_x^2 - S_y^2)
		\end{array}
	\end{align}
The energy structure given by those hamiltonian is presented in Fig.~\ref{CompleteEnerStruct}, along with the experimental and calculated linear polarization map. Values of $I_{hMn}$, $I_{eMn}$, $\rho_c/\Delta_{lh}$ and $\eta$ for a given QD can be obtained by comparing the linear polarization dependence of the experimental PL data to the optical transition probabilities calculated with the discussed effective spin model (Fig.~\ref{CompleteEnerStruct} (c)) \cite{DynhMn}. A Boltzmann distribution function $P^i_{eMn}=e^{-E^i_{eMn}/k_BT_{eff}}/\sum_{i}e^{-E^i_{eMn}/k_BT_{eff}}$ with an effective spin temperature $T_{eff}$ is used to describe the population of the emitting states (electron-Mn energy levels $E^i_{eMn}$). The obtained parameters are listed in Tab.~\ref{paraQD} for the QD presented in Fig.~\ref{hMnspectra} and \ref{CompleteEnerStruct}.

	\begin{table}[t] \centering
		\caption{Values of the parameters used in the model of the positively charged Mn-doped QD presented in Fig.~\ref{hMnspectra}. I$_{eMn}$, I$_{hMn}$, $\frac{\rho_s}{\Delta_{lh}}$, $\theta$, $\eta$ and $T_{eff}$ are used to model the linear polarization intensity map of Fig.~\ref{CompleteEnerStruct}. The other parameters cannot be extracted from the PL measurements and values for typical Mn-doped QDs are chosen for the calculation of the spin dynamics presented in Sec.~\ref{SpinDyn} and \ref{StrainInfl}.}
		\renewcommand{\arraystretch}{1.0}
		\begin{tabular}{cccccc|ccccc}
			\hline\hline
			I$_{eMn}$ & I$_{hMn}$ & $\frac{\rho_s}{\Delta_{lh}}$ & $\theta$    & $\eta$   & $T_{eff}$  & $g_{e}$ & $g_{h}$   	& $g_{Mn}$ & $D_0$    &  $E$      \\
			$\mu eV$  & $\mu eV$  &                              & $^{\circ}$  & $\mu eV$ &    K       &         &           &          & $\mu eV$ &  $\mu eV$ \\
			\hline
			-175    &     345   &        0.09                  &    0        &     30   &   20       &  -0,4   &  0.6      &     2    &    7     &   1.5     \\
			\hline\hline
		\end{tabular}
		\label{paraQD}
	\end{table}
	
%	\lipsum[11]

		\subsection{Optical $\Lambda$-level identification\label{LambdaId}}	
		
		\lipsum[11]
	
	\begin{figure}[h!]
	\begin{center}
		\includegraphics[width=12cm]{Pictures/Lambdasyst.png}
	\end{center}
	\caption{(a) Non resonant (Non Res.) and resonant (Res.) PL of X$^+$-Mn. Co and cross circularly polarized PL spectra are collected for three different energies of the CW resonant laser (green). Inset: intensity map of the cross-circularly polarized PL detected on the low energy side of X$^+$-Mn as the CW laser is scanned through the high energy side. (b) Energy levels of X$^+$-Mn and identification of the three resonances observed in (a) corresponding to the optical $\Lambda$ systems associated with the e-Mn states $|3,+1\rangle$, $|3,+2\rangle$ and $|2,+2\rangle$.}
	\label{LambdaLevel}
	\end{figure}
	
	Using Fig.~\ref{Recomb}, we can assign each peak to a transition and therefore, for a given polarization, to an electron-Mn state.
			
	
	\section{Spin dynamics under resonant excitation\label{SpinDyn}}
	
%		Cf article 2016/01
	
		\subsection{Cycling and escaping the $\lambda$-level system}

%		Mn in a lattice -> modification of orbital -> spin-orbit interaction. Magnetic anisotropy + anisotropy of strain. (Mn has nuclear spin 5/2 -> hyperfine interaction?)\newline

	Under resonant excitation of one high energy level of X$^+$-Mn, only one cross-circularly polarized emission line is observed. It corresponds to the optically allowed recombination on the second branch of the $\Lambda$ system. This recombination occurs with a flip-flop of the electron and Mn spins \cite{DynhMn}. The energy splitting between the resonant absorption and the emission corresponds to the splitting between the two ground states of the $\Lambda$ system. It is given by 4$\times$3/2$I_{hMn}$($\approx$2.1 meV for the studied QD) for an excitation of $|3,+2\rangle$ or $|2,+2\rangle$ and 2$\times$3/2$I_{hMn}$($\approx$1.05 meV for the studied QD) for an excitation of $|3,+1\rangle$. For an excitation of $|3,+2\rangle$ or $|2,+2\rangle$, the weak co-polarized PL signal, which depends on the excitation intensity, comes from a possible direct excitation of the low energy branch of the $\Lambda$ system through the acoustic phonon side-band \cite{BesombesAccPhon}.
	
	For an isolated $\Lambda$ system, under resonant excitation of one of the branch, a fast optical pumping controlled by the generation rate and the radiative lifetime of the excited state is expected: The population is expected to be stored in the level which is not excited and the resonant PL should vanish. In the case of X$^+$-Mn, the PL intensity observed under resonant excitation of the high energy branch of the $\Lambda$ systems is similar to the PL intensity obtained under non-resonant excitation. This suggests a very inefficient optical pumping of the Mn-hole spin and an efficient spin-flip mechanism which links the two ground states of the $\Lambda$ systems.
	
	\begin{figure}[h!]
	\begin{center}
		\includegraphics[width=14.8cm]{Pictures/ResAutocor.eps}
	\end{center}
	\caption{Auto-correlation of the resonant PL for cross-circularly polarized excitation and detection of the electron-Mn states (a) $|3, +1\rangle$, (b) $|3, +2\rangle$ and (c) $|2, +2\rangle$.}
	\label{AllAutocorB0}
	\end{figure}
	
	The dynamics of the Mn spin coupled to carriers was first analyzed, under resonant optical excitation, through the statistics of the time arrival of the photons given by the second order correlation function of the resonant PL intensity, $g^{(2)}(\tau)$. For the three resonant excitation conditions reported in Fig.\ref{AllAutocorB0}, $g^{(2)}(\tau)$ is mainly characterized by a large photon bunching with a full width at half maximum (FWHM) in the 20 ns range. The amplitude of the bunching reaches 9 for line (2) and is slightly weaker for the two other lines. This large bunching, reflecting an intermittency in the emission of the QD, is not sensitive to a longitudinal magnetic field B$_z$ except for an excitation on (1).

	The presence of a photon bunching is at first sight surprising: under resonant excitation of an isolated $\Lambda$ system, an anti-bunching of the resonant PL controlled by the transfer time between the two ground states is indeed expected. For X$^+$-Mn, the observed short anti-bunching (dip near zero delay, better evidenced in Fig.~\ref{AllAutocorB0} (b)) suggests a fast transfer time in the nanosecond range between the two ground states of the $\Lambda$ systems.
	
	\begin{figure}[h!]
	\fcapside{\caption{Schema of the energy levels of the optical $\Lambda$ system associated with the electron-Mn state $|3, +2\rangle$ extracted from the full level structure of a positively charged Mn-doped QD (Fig.~\ref{LambdaLevel}). The different processes discussed in this section are presented on it.}\label{LambdLoop}}	
	{\begin{center}
		\includegraphics[width=6cm]{Pictures/RelaxMecanism.png}
	\end{center}}
	\end{figure}

	In the presence of a transfer process connecting the two Mn-hole ground states in a nanosecond time-scale, the photon bunching can be explained by leaks outside the resonantly excited $\Lambda$ system. Under $cw$ excitation, the population is cycled inside the $\Lambda$ system until a spin flip occurs and drives the carrier-Mn spin out of the $\Lambda$ levels under investigation. The resonant PL is then switched off until multiple spin-flips drives back the carriers and Mn spin inside the $\Lambda$ system under excitation. The selected QD line can be either in a ON or OFF state depending on the fluctuations of the carrier and Mn spins. The amplitude of the bunching is then given by $\Gamma_{Out}/\Gamma_{In}$ the ratio of the transition rates from OFF to ON ($\Gamma_{In}$) and from ON to OFF ($\Gamma_{Out}$). An amplitude of bunching larger than 1 is expected for the multilevel system considered here where, after a spin relaxation, multiple spin flips are in average required to come back to the initial state ($\Gamma_{In}<\Gamma_{Out}$). Within this picture, the width of the bunching is a measurement of the escape time out of the considered $\Lambda$ level system. We present these transitions in Fig.~\ref{LambdLoop}, on the $\Lambda$ system associated with $|3, +2\rangle$ state.
	
		\begin{figure}[h!]
			\begin{center}
				\includegraphics[width=12cm]{Pictures/AutocorBPw.eps}
			\end{center}
			\caption{Excitation power dependence (a) and transverse magnetic field dependence (b) of the auto-correlation of the resonant PL obtained for an excitation on the high energy branch of the $\Lambda$ level system associated to the e-Mn state $|2,+2\rangle$.}
			\label{AutocorExpBPw}
		\end{figure}
	
	A weak transverse magnetic field, $B_x$, significantly reduces the width of the bunching signal (Fig.\ref{AutocorExpBPw} (b)). As the spin of the Mn-hole complex is highly anisotropic, with a large energy splitting induced by the exchange interaction $I_{hMn}S_z.J_z$, the weak transverse magnetic field mainly affects the electron-Mn dynamics in the excited state of the charged QD. Indeed, the transverse magnetic field couples the different electron-Mn states and induces a leak outside the resonantly excited $\Lambda$ system. Both spin-flips within the Mn-hole (ground state) and the electron-Mn (excited state) systems can contribute to the bunching signal. The significant effect of the weak transverse field shows that the probability of presence in the excited state of the $\Lambda$ system is large. This is consistent with the large excitation intensity used for these auto-correlation measurements which require a high photon count rate.

	A slight reduction of the width of the bunching signal is also observed with the increase of the excitation power (Fig.\ref{AutocorExpBPw} (a)). This shows that the leaks outside a given $\Lambda$ system slightly increases with the probability of presence of the positively charged exciton in the QD.\newline

	\begin{figure}[h!]
	\begin{center}
		\includegraphics[width=14.7cm]{Pictures/ResPumpv2.png}
	\end{center}
	\caption{Resonant optical pumping transients obtained under circular polarization switching of the resonant excitation for the $\Lambda$ systems associated with (a) $|3, +1\rangle$, (b) $|3, +2\rangle$ and (c) $|2, +2\rangle$ at zero field and under a weak longitudinal magnetic field B$_z$=0.23T. The insets present the corresponding states which are resonantly excited and detected in $\sigma-$ polarization.}
	\label{AllPumpB0}
	\end{figure}
	
	Resonant optical pumping experiments were done to estimate how long it takes, after a spin-flip, to the hybrid Mn-hole spin to relax back inside the resonantly excited $\Lambda$ system. A demonstration of resonant optical pumping of the Mn-hole system was first done by exciting the high energy branch of the $\Lambda$ systems with trains of resonant light, alternating the circular polarization and recording the circularly polarized PL of the low energy branch. As observed in Fig.~\ref{AllPumpB0}, for an excitation on resonance with the electron-Mn states $|3,+2\rangle$ or $|2,+2\rangle$, switching the polarization of the excitation from co to cross circular produces a change of the PL intensity with two transients: first, an abrupt PL increase (or decrease), reflecting the population change of the observed spin-polarized charged excitons; then a slower transient with a characteristic time of a few tens of nanoseconds, depending on the laser excitation power.
	
	The progressive decrease of the resonant PL intensity is the signature of an optical pumping of the Mn-hole spin: the Mn-hole state which is optically addressed is partially emptied when the population is ejected out of the excited $\Lambda$ system. As presented in Fig. ~\ref{AllPumpB0}, this pumping signal is not sensitive to a longitudinal magnetic field B$_z$ except for an excitation of $|3,\pm1\rangle$ where a significant intensity difference between co and cross circular polarization is only observed under a weak B$_z$.

	\begin{figure}[h!]
	\begin{center}
		\includegraphics[width=12cm]{Pictures/PumpBPw.eps}
	\end{center}
	\caption{Excitation power dependence (a) and transverse magnetic field dependence (b) of the optical pumping signal obtained for a resonant excitation on $|3,+2\rangle$. Insets: excitation power dependence of the pumping time and transverse magnetic field dependence of the difference of resonant PL intensity between a $\sigma_{cross}$ and a $\sigma_{co}$ excitation.}
	\label{PumpExpBPw}
	\end{figure}
	
	The speed of the optical pumping increases with the excitation intensity. This is presented in Fig.~\ref{PumpExpBPw} (a) in the case of a resonant excitation of $|3,\pm2\rangle$ with alternate circular polarization. At high excitation intensity, the pumping time saturates to a value similar to the width of the bunching signal observed in the auto-correlation measurements.
	
	As observed for the auto-correlation, the resonant pumping signal is also strongly sensitive to a transverse magnetic field. Under a weak transverse field (see Fig.~\ref{PumpExpBPw} (b)), we first observe an increase of the speed of the pumping together with a decrease of the amplitude of the signal when the transient time reaches the time resolution of the set-up (around 10 ns). For a large transverse field (B$_\perp$=0.42T), the co and cross circularly polarized resonant PL intensities are identical (see the inset of Fig.~\ref{PumpModBPw} (b)) and similar pumping transients are observed when switching from $\sigma_{co}$ to $\sigma_{cross}$ or from $\sigma_{cross}$ to $\sigma_{co}$ circular polarization.
	
	\begin{figure}[h!]
	\begin{center}
		\includegraphics[width=13cm]{Pictures/DarkRelax.png}
	\end{center}
	\caption{Optical pumping experiment for an excitation of $|3,+2\rangle$ with modulated circular polarization. A dark time ($\tau_{dark} = 50ns$) is introduced in the pumping sequence. The polarization switching occurs either before (black) or during (red) the dark time. The black and red diagrams present the corresponding resonant excitation sequences. The inset presents the variation of the ratio $\Delta I/I$ as a function of $\tau_{dark}$. The solid line is an exponential fit with $\tau_{relax} = 80 ns$.}
	\label{PumpExpDark}
	\end{figure}
	
	To observe the relaxation of the prepared non–equilibrium distribution of the Mn-hole spins, the circularly polarized pump laser is switched off during a dark time $\tau_{dark}$. The amplitude of the pumping transient which appears after $\tau_{dark}$ depends on the Mn-hole spin relaxation. A dark time of 50 ns is enough to observe the reappearance of a significant pumping transient (Fig.~\ref{Fig6}). For comparison and for a better sensitivity of the measurement, the pumping transient observed in the absence of initial preparation of the Mn-hole spin (i.e. when switching of the circular polarization during the dark time) is also presented (red trace in Fig.~\ref{PumpExpDark}). The normalized difference of the amplitude of these two transients, $\Delta I/I$, as a function of $\tau_{dark}$ is presented in the inset of Fig.\ref{PumpExpDark}. This measurement shows that, when the optical excitation is off, it takes around 80 ns to the Mn-hole spin to come back to the ground state of the excited $\Lambda$ system.

	If the optical pumping was storing the Mn-hole spin in the branch of the $\Lambda$ system which is not optically excited, its characteristic time would be controlled by the exciton radiative lifetime and the generation rate. With a Mn-hole relaxation time in the 100 ns range, as observed experimentally, the pumping should take place within a few nanoseconds.

	Another source of spin pumping can be the leak outside the resonantly excited $\Lambda$ system. In this case, the speed of the pumping is controlled by the leakage time and, as observed experimentally, the pumping time is similar to the width of the photon bunching signal. This mechanism of pumping for the Mn-hole spin is confirmed by the transverse magnetic field dependence. The acceleration of the optical pumping in transverse magnetic field (Fig.~\ref{PumpExpBPw} (b)) has the same origin as the decrease of the width of the bunching signal. By mixing the different electron-Mn states, the transverse field enhances the leakage probability out of the resonantly driven $\Lambda$ system and decreases the corresponding optical pumping time.
	
		\subsection{Relaxation mechanism\label{RelaxMech}}
	
			\subsubsection*{Hole-Mn flip-flops mediated by a lattice deformation}		
		
		The observed large resonant PL amplitude of X$^+$-Mn and its dynamics can be qualitatively explained if a fast (nanosecond) and efficient spin transfer mechanism connects the two Mn-hole ground states of each $\Lambda$ system. This fast transfer is noted $\tau_{ff}$ on Fig.~\ref{LambdLoop}.

	We propose a mechanism for the Mn-hole flip-flop at low temperature resulting from a deformation induced exchange interaction~\cite{ExcSpinRelaxQD,ExcSpinDecay}. We show here that Mn-hole states are efficiently coupled via the interplay of their exchange interaction and the lattice deformation induced heavy-hole/light-hole mixing. We will focus in the following on the two Mn-hole states $|+\frac{3}{2};\Uparrow_h\rangle$ and $|+\frac{5}{2};\Downarrow_h\rangle$ in the ground states of the $\Lambda$ system associated with the electron-Mn levels $|3,+2\rangle$ and $|2,+2\rangle$. Similar results could be obtained with the Mn-hole ground states of the other $\Lambda$ systems.

	First, let us notice that the non diagonal term of the Mn-hole exchange interaction $I_{hMn}/2(S^+J^-+S^-J^+)$ couples the heavy-holes ($\Uparrow_h,\Downarrow_h)$ and light-holes $(\uparrow_h,\downarrow_h)$ levels split by $\Delta_{lh}$ through a Mn-hole flip-flop. We consider this interaction as a perturbation on the Mn heavy-hole level structure given by $I_{hMn}S_zJ_z$. To the first order in $I_{hMn}/\Delta_{lh}$, the two perturbed ground states of the $\Lambda$ system considered here $\widetilde{|+\frac{3}{2};\Uparrow_h\rangle}$ and $\widetilde{|+\frac{5}{2};\Downarrow_h\rangle}$ can be written~\cite{CohenTannoudji}:
	\begin{eqnarray}
		\widetilde{|+\frac{5}{2};\Downarrow_h\rangle}=|+\frac{5}{2};\Downarrow_h\rangle-\frac{\sqrt{15}}{2}\frac{I_{hMn}}{\Delta_{lh}}|+\frac{3}{2};\downarrow_h\rangle\nonumber\\
\widetilde{|+\frac{3}{2};\Uparrow_h\rangle}=|+\frac{3}{2};\Uparrow_h\rangle-\frac{\sqrt{15}}{2}\frac{I_{hMn}}{\Delta_{lh}}|+\frac{5}{2};\uparrow_h\rangle
	\end{align}
where we neglect the exchange energy shifts of the Mn-hole levels much smaller than $\Delta_{lh}$.

		\begin{table}[hbt] \centering
			\caption{Material (CdTe or ZnTe) \cite{CdTeBPCoef} and QD parameters used in the calculation of the coupled hole and Mn spin relaxation time.}
			\begin{tabular}{lcr}
				\hline\hline
				CdTe& &\\
				\hline
				Deformation potential constants & b &  -1.0 eV  \\
												& d &  -4.4 eV  \\
				Longitudinal sound speed & c$_l$ &  3300 m/s  \\
				Transverse sound speed & c$_t$ &  1800 m/s  \\
				Density & $\rho$ &  5860 kg/m$^3$  \\
				\hline
				ZnTe& &\\
				\hline
				Deformation potential constants & b &  -1.4 eV  \\
												& d &  -4.4 eV  \\
				Longitudinal sound speed & c$_l$ &  3800 m/s  \\
				Transverse sound speed & c$_t$ &  2300 m/s  \\
				Density & $\rho$ &  5908 kg/m$^3$  \\
				\hline
				Quantum dot& &\\
				\hline
				Hole Mn exchange energy & I$_{hMn}$ &  0.35 meV  \\
				hh-lh exciton splitting&  $\Delta_{lh}$&  15 meV  \\
				Hole wave function widths: & &   \\
				- in plane & l$_{\bot}$ &3.0 nm   \\
				- z direction  & l$_z$ &1.25 nm   \\
				\hline\hline
			\end{tabular}
			\label{paraph}
		\end{table}

	Phonon-induced deformations comes into play via the off-diagonal terms of the Bir-Pikus Hamiltonian ${\cal H}_{BP}$, describing the influence of strain on the valence band, as written in Eq.~\ref{BPHamil}. The parameters $a_{\nu}$, $b$ and $d$ are given in Tab.~\ref{paraph}. The strain produced by phonon vibrations couples the perturbed Mn-hole states $\widetilde{|+5/2\Downarrow_h\rangle}$ and $\widetilde{|+3/2\Uparrow_h\rangle}$ through the Hamiltonian term
	\begin{align}
		\label{int}
		\widetilde{\langle+\frac{5}{2};\Downarrow_h|}H_{BP}\widetilde{|+\frac{3}{2};\Uparrow_h\rangle}=2\times(-\frac{\sqrt{15}}{2}\frac{I_{hMn}}{\Delta_{lh}})\times R^*
	\end{align}
with
	\begin{align}
		\label{P}
		R=\frac{\sqrt{3}}{2}b(\epsilon_{xx}-\epsilon_{yy})-id\epsilon_{xy}
	\end{align}
a deformation dependent non-diagonal term of ${\cal H}_{BP}$~\cite{ExcSpinRelaxQD,ExcSpinDecay}. The coupling of the Mn-hole states is a result of an interplay between the Mn-hole exchange interaction and the deformation: neither the exchange interaction nor the deformation perturbation alone can couple these states.

	According to (\ref{int}), an effective Hamiltonian describing the discussed interaction mechanism with phonons in the subspace $\{|+\frac{5}{2};\Uparrow_h\rangle,|+\frac{5}{2};\Downarrow_h\rangle, |+\frac{3}{2};\Uparrow_h\rangle,|+\frac{3}{2};\Downarrow_h\rangle\}$  is
	\begin{align}
		\label{Hint}
		H_{int}=-\sqrt{15}\frac{I_{hMn}}{\Delta_{lh}}R^*|+\frac{5}{2};\Downarrow_h\rangle\langle+\frac{3}{2};\Uparrow_h|+H.c
	\end{align}

	The spin decay rates from $|+\frac{3}{2};\Uparrow_h\rangle$ to $|+\frac{5}{2};\Downarrow_h\rangle$ accompanied by the emission of an acoustic phonon is then given by Fermi's golden rule
	\begin{align}
		\label{fermi}
		\tau^{-1}&=\frac{2\pi}{\hbar}\sum_{k}\left|\langle+\frac{5}{2};\Downarrow_h;\psi;n_k+1|H_{int}|+\frac{3}{2};\Uparrow_h;\psi;n_k\rangle\right|^2 \times \delta(\hbar\omega_0-\hbar\omega_{k})
	\end{align}
where $\hbar\omega_0$ is the energy splitting between $|+\frac{5}{2};\Downarrow_h\rangle$ and $|+\frac{3}{2};\Uparrow_h\rangle$, $n_k$ the number of phonons in mode $k$ and $\psi$ the orbital part of the hole wave function.

	To evaluate the matrix element in (\ref{fermi}) we use the strain tensor components $\epsilon_{ij}$ given by
	\begin{align}
		\label{eps}
		\epsilon_{ij}=\frac{1}{2}\left(\frac{\partial u_i}{\partial r_j}+\frac{\partial u_j}{\partial r_i}\right)
	\end{align}
where $\overrightarrow{u}(\overrightarrow{r})$ is the local displacement field. For an acoustic phonon, the quantized displacement field can be written in the real space~\cite{ExcSpinDecay,ManyPartPhys}:
	\begin{align}
		\label{u}
		\overrightarrow{u}(\overrightarrow{r})=i\sum_{k,\lambda}\sqrt{\frac{\hbar}{2\rho\omega_{k,\lambda}N\nu_0}}\overrightarrow{e}_{k,\lambda}(b_{k,\lambda}+b^\dag_{-k,\lambda})e^{i\overrightarrow{k}\overrightarrow{r}}
	\end{align}
where N is the number of unit cells in the crystal, $\nu_0$ is the volume of a cell and $\rho$ the mass density. $b^\dag_{k,\lambda}$ ($b_{k,\lambda}$) is the creation (annihilation) operator of phonon in the mode $(k,\lambda)$ of energy $\hbar\omega_{k\lambda}$ and unit polarization vector $\overrightarrow{e}_{k,\lambda}$. In zinc-blend crystals there are two transverse acoustic phonon branches $\lambda=t_1, t_2$ and one longitudinal acoustic phonon branch $\lambda=l$. The polarization vectors of these phonons branches are given by \cite{hSpinRelax}
	\begin{align}
		\label{k}
		\begin{array}{rl}
	\overrightarrow{e}_{k,l}=&\frac{\overrightarrow{k}}{k}=\frac{1}{k}(k_x,k_y,k_z)\\
	\overrightarrow{e}_{k,t_1}=&\frac{1}{kk_{\bot}}(k_xk_z,k_yk_z,-k_{\bot}^2)\\
		\overrightarrow{e}_{k,t_2}=&\frac{1}{k_{\bot}}(k_y,-k_x,0)
		\end{array}
	\end{align}
with $k_{\bot}=\sqrt{k_x^2+k_y^2}$.

	Upon substitutions given by (\ref{eps}), (\ref{u}) and (\ref{k}), we obtain for the matrix element in (\ref{fermi}):
	\begin{align}
		\begin{array}{rl}
			|M_{k,\lambda}|^2=& 15\left(\dfrac{I_{hMn}}{\Delta_{lh}}\right)^2\dfrac{\hbar}{2\rho\omega_{k,\lambda}N\nu_0}\left(n_B(\omega_{k,\lambda})+1\right) \\
			& \times \left(\dfrac{3b^2}{4}(k_xe_{x,\lambda}-k_ye_{y,\lambda})^2+\dfrac{d^2}{4}(k_xe_{y,\lambda}+k_ye_{x,\lambda})^2\right \times |\mathcal{F}_{\lambda}(\overrightarrow{k})|^2
		\end{array}
	\end{align}
with
	\begin{align}
	\mathcal{F}_{\lambda}(\overrightarrow{k})=\int_{-\infty}^{\infty}d^3r\psi^*(\overrightarrow{r})e^{i\overrightarrow{k}\overrightarrow{r}}\psi(\overrightarrow{r})
	\end{align}
and $n_B(\omega_{k,\lambda})=
1/(e^{\hbar\omega_{k,\lambda}/K_BT}-1)$, the thermal phonon distribution function.

	For a Gaussian hole wave function with in-plane and z-direction parameters $l_{\bot}$ and $l_z$ respectively (full width at half maximum $2\sqrt{2\ln2}l_i$)
	\begin{align}
		\psi(\overrightarrow{r})=\frac{1}{\pi^{3/4}l_{\bot}\sqrt{l_z}}e^{-\frac{1}{2}\left(\left(\frac{r_{\bot}}{l_{\bot}}\right)^2+\left(\frac{z}{l_z}\right)^2\right)}
	\end{align}
the form factor $\mathcal{F}_{\lambda}(\overrightarrow{k})$, which is the Fourier transform of $|\psi(\overrightarrow{r})|^2$, becomes
	\begin{align}
	\mathcal{F}_{\lambda}(\overrightarrow{k})=e^{-\frac{1}{4}\left(\left(l_{\bot}k_{\bot}\right)^2+\left(l_zk_z\right)^2\right)}
	\end{align}
	
	Considering a linear dispersion of acoustic phonons $\omega_{k,\lambda}=c_{\lambda}k$ and in spherical coordinates with $\overrightarrow{k}=k(\sin\theta\cos\varphi,\sin\theta\sin\varphi,\cos\theta)$, the explicit formula of the decay rate (\ref{fermi}) is
	\begin{align}
		\begin{array}{rl}
			 \tau^{-1}=& \sum_{\lambda}\dfrac{15}{(2\pi)^2}\left(\dfrac{I_{hMn}}{\Delta_{lh}}\right)^2\left(\dfrac{\omega_0}{c_{\lambda}}\right)^3\dfrac{1}{2\hbar\rho c_{\lambda}^2}\dfrac{\pi}{4}\left(3b^2+d^2\right) \\
 					& \times \left(n_B(\omega_0)+1)\right)\int_0^{\pi}d\theta\sin\theta|\mathcal{F}_{\lambda}(\omega_0,\theta)|^2G_{\lambda}(\theta)
		\end{array}
	\end{align}
where we used the continuum limit ($\sum_k\rightarrow V/(2\pi)^3\int d^3k$ with $V=N\nu_0$ the crystal volume) and integrated over $k$ and $\varphi$. The summation is taken over the acoustic phonon branches $\lambda$ of corresponding sound velocity $c_{\lambda}$. The geometrical form factors for each phonon branch, $G_{\lambda}(\theta)$, are given by
	\begin{align}
		\begin{array}{rl}
			G_{l}(\theta)=& \sin^4\theta \\
			G_{t_1}(\theta)=& \sin^2\theta\cos^2\theta \\
			G_{t_2}(\theta)=&\sin^2\theta
		\end{array}
	\end{align}
	
	\begin{figure}[h!]
	\fcapside{\caption{Relaxation time $\tau_{ff}$, between the two Mn-hole ground states of the $\Lambda$ system  calculated with the material and QD parameters listed in Table \ref{paraph} and a temperature T=7K. The vertical line shows the energy splitting in the studied QD of the Mn-hole states involved in the $\Lambda$ systems considered here (Resonances (2) and (3) identified in Fig.~\ref{LambdaLevel}).}\label{TauRelax}}
	{\begin{center}
		\includegraphics[width=7cm]{Pictures/TauffRelax.eps}
	\end{center}}
	\end{figure}
	
	In the numerical calculation of the spin flip time $\tau_{ff}$ presented in Fig.\ref{TauRelax} we use the material parameters of CdTe or ZnTe and the typical parameters for self-assembled CdTe/ZnTe QDs listed in Table \ref{paraph}. The calculated relaxation time strongly depend on the energy separation between the Mn-hole levels $\hbar\omega_0$. This energy dependence is controlled by the size of the hole wave-function given by l$_{\bot}$ and l$_z$. The estimated flip-flop time is also strongly sensitive on the exchange induced mixing of the ground heavy-hole states with the higher energy light-hole levels. In our model, this mixing is controlled by $\Delta_{lh}$, an effective energy splitting between heavy-holes and light-holes. This simple parameter can indeed describe more complex effects such as a coupling of the confined heavy-hole with ground state light-holes in the barriers \cite{Couplinghhlhbarrier} or effective reduction of heavy-hole/light-hole splitting due to a presence of a dense manifold of heavy-hole like QD states lying between the confined heavy-hole and light-hole levels \cite{Supercouplinghhlh}. From this modelling we deduce that for a hole confined in small Cd$_x$Zn$_{1-x}$Te alloy QDs, the Mn-hole flip-flop time $\tau_{ff}$ can be easily bellow 2 ns for an effective heavy-hole/light-hole splitting $\Delta_{lh}=15$ meV and an energy separation in the meV range, typical for the Mn-hole spin splitting in magnetic QDs. We will use in the following calculations $\tau_{ff}=1.5$ ns for the ground states of each $\Lambda$ system.
	
		\subsubsection*{Model of the carrier-Mn spin dynamics under resonant excitation}
	
	Using the level scheme presented in Fig.~\ref{CompleteEnerStruct} (a) for a positively charged Mn-doped QD and the estimated Mn-hole flip-flop rates, we can calculate the time evolution of the 24x24 density matrix $\varrho$ describing the population and the coherence of the 12 electron-Mn states on the excited state and the 12 Mn-hole states on the ground state of a positively charged QD. In the Markovian approximation, the master equation which governs the evolution of $\varrho$ can be written in a general form (Lindblad form) as:	
	\begin{align}
		\label{Lindblad}
		\frac{\partial\varrho}{\partial t}=\frac{-i}{\hbar}[{\cal H},\varrho]+L\varrho
	\end{align}
where ${\cal H}$ is the Hamiltonian of the complete system (X$^+$-Mn, ${\cal H}_{X^+Mn}$, and Mn-hole, ${\cal H}_{hMn}$):
	\begin{align}
		\begin{array}{rll}
			{\cal H}_{X^+Mn}=&I_{eMn}\vec{S}\cdot\vec{\sigma}-2\eta S_z^2 &+ D_0S^2_z+E(S_y^2-S_x^2) \\
																	  & &+ g_{Mn} \mu_B \mathbf{S} \cdot \mathbf{B} + g_{e} \mu_B \bm{\upsigma}\cdot\mathbf{B}
		\end{array}
	\end{align}
and
	\begin{align}
		\begin{array}{rll}
			{\cal H}_{hMn}=&I_{hMn}\vec{S}\cdot\vec{J}-\eta S_z^2 &+ D_0S^2_z+E(S_y^2-S_x^2) \\
																& &+ g_{Mn} \mu_B \mathbf{S}\cdot\mathbf{B} + g_{h} \mu_B \mathbf{J}\cdot\mathbf{B}
		\end{array}
	\end{align}

	In Eq.~\ref{Lindblad}, $L\varrho$ describes the coupling or decay channels resulting from an
interaction with the environment~\cite{SpinQJumps, ephBroad, MnResSpinDyn}. The population transfers from level $j$ to level $i$ in an irreversible process associated with a coupling to a reservoir is described by a Lindblad term of the form
	\begin{align}
		\label{inc}
		L_{inc,j\rightarrow i}\varrho=\frac{\Gamma_{j\rightarrow i}}{2}(2|i\rangle\langle j|\varrho|j\rangle\langle i| -\varrho|j\rangle\langle j|-|j\rangle\langle j|\varrho)
	\end{align}
where $\Gamma_{j\rightarrow i}$ is the incoherent relaxation rate from level $j$ to level $i$. Such term can describe the radiative decay of the exciton (irreversible coupling to the photon modes) or the relaxation of the carriers or Mn spin (irreversible coupling to the phonon modes). It can also be used to describe the optical generation of an exciton in the low excitation regime where the energy shift induced by the coupling with the laser field is neglected.

	A pure dephasing (i.e. not related to an exchange of energy with a reservoir) can also be introduced for the different spins and described by $L_{deph,jj}\varrho$:
	\begin{align}
		\label{deph}
		L_{deph,jj}\varrho=\frac{\gamma_{jj}}{2}(2|j\rangle\langle j|\varrho|j\rangle\langle j| -\varrho|j\rangle\langle j|-|j\rangle\langle j|\varrho)
	\end{align}
where $\gamma_{jj}$ is the pure dephasing rate of level $j$.

	To identify the main spin relaxation channels responsible for the observed spin fluctuations, we first modelled the auto-correlation of the resonant PL using the full spin level structure of a p-doped magnetic QD. For a qualitative description of the observed spin dynamics, we use as an example the Mn-doped QD parameters extracted from the linear polarization intensity map listed in Tab.~\ref{paraQD} and reasonable order of magnitude for the spin relaxation times.

	As already observed in charged Mn-doped QDs under pulsed resonant excitation~\cite{VybornyMagAnisQDs}, we consider that the spin dynamics in the excited state is controlled by the time evolution of ${\cal H}_{X^+Mn}$, the generation rate of excitons $\gamma_g=1/\tau_g$ and their radiative lifetime $\tau_r=0.3$ ns. The coherence of the coupled electron-Mn spins is limited by a pure dephasing term $T_2^{eMn}=0.5$ ns, extracted from the time resolved PL (see Sec.~\ref{StrainInfl}).

	For the Mn-hole system in the ground state, we take into account a spin relaxation time of the Mn in the exchange field of the hole, $\tau_{Mn}$, describing relaxation channels involving a change of the Mn spin by one unit. This spin relaxation channel is introduced for a general description, however its characteristic time (in the $\mu$s range) is long compared to the time-scale of the dynamics considered in the resonant PL experiments and does not qualitatively affect the calculated time evolution.

	Because of the presence of valence band mixing in the QDs, spin flip of the hole independently of the Mn are expected to be more efficient. A spin flip time in the 10 ns range has indeed been calculated for a hole in the exchange field of a Mn~\cite{CaoSpinPhonCoupl,OptOrientMn}. Relaxation time of the hole spin around 5 ns has also been measured at zero magnetic field in negatively charged CdTe/ZnTe QDs~\cite{LeGallElecNuclDyn}. We then include in the model possible spin flips of the hole by one unit with a characteristic time $\tau_{h}$=10ns. The phonon induced Mn-hole flip-flops, occurring at $\tau_{ff}$, are also introduced between the two Mn-hole ground states of each $\Lambda$ system.

	For a general qualitative description, an additional pure dephasing time $T_2^{hMn}$ is also included in the dynamics of the Mn-hole system with a Lindblad term of the form (Eq.~\ref{deph}). We cannot extract this parameter from the experiments. We take $T_2^{hMn}= 5 ns$, slightly longer than what was measured for electron-Mn, as the Mn-hole system is highly split and less sensitive to effective fluctuating magnetic field such as the one produced by nuclear spins for instance~\cite{LeGallElecNuclDyn,SingleHolePopTrap}.
	
	The transition rates $\Gamma_{\gamma\rightarrow\gamma'}$ between the different Mn-hole states depend on their energy separation $E_{\gamma\gamma'}=E_{\gamma'}-E_{\gamma}$. Here we use $\Gamma_{\gamma\rightarrow\gamma'}=1/\tau_{i}$ if $E_{\gamma\gamma'}<0$ and $\Gamma_{\gamma\rightarrow\gamma'}=1/\tau_{i}e^{-E_{\gamma\gamma'}/k_BT}$ if $E_{\gamma\gamma'}>0$~\cite{PhotoSpinOrient,CaoSpinPhonCoupl}. This accounts for a thermalization among the 12 Mn-hole levels with an effective spin temperature $T$. The optical excitation ($\tau_g$), the exciton recombination ($\tau_r$), the Mn spin relaxation ($\tau_{Mn}$), the hole spin relaxation ($\tau_{h}$) and the phonon induced transfer time ($\tau_{ff}$) produce a irreversible population transfer between level $\gamma$ and $\gamma'$ and are described by Lindblad terms (\ref{inc}).

	\begin{figure}[h!]
	\begin{center}
		\includegraphics[width=10cm]{Pictures/AutocorSimu.eps}
	\end{center}
	\caption{(a) Calculated time evolution of $\rho_{|+\frac{3}{2},\uparrow_e\rangle}(t)$ with the QD parameters listed in Table~\ref{paraQD} and (unless specified) $\tau_r$=0.3ns, $\tau_{Mn}$=5 $\mu$s, $\tau_h$=10ns, $\tau_g$=0.25 ns, $\tau_{ff}$=1.5 ns, $T_2^{hMn}$= 5 ns, $T_2^{eMn}$= 0.5 ns, T=10K and B$_{\perp}$=0. (b) (c) and (d) illustrate the influence  of, respectively, $\tau_{ff}$, $\tau_g$ and $B_{\perp}$ on $\rho_{|+\frac{3}{2},\uparrow_e\rangle}(t)$. Note the different vertical scale in (b).}
	\label{AutocorModBPw}
	\end{figure}
	
	To model the auto-correlation of the $\sigma-$ PL intensity of the electron-Mn state $|3;+2\rangle$ under $cw$ $\sigma+$ resonant excitation we calculate the time evolution of $\rho_{|+\frac{3}{2};\uparrow_e\rangle}(t)$ with the initial condition $\rho_{|+\frac{3}{2};\Uparrow_h\rangle}(0)=1$ corresponding to the Mn-hole spin in the state $|+\frac{3}{2};\Uparrow_h\rangle$ just after the emission of a $\sigma-$ photon on the low energy branch of the $\Lambda$ system. This initial state is a slight approximation: in the presence of valence band mixing, the two ground states of a given $\Lambda$ system are not completely pure Mn-hole spin states but are slightly coupled by a Mn-hole flip-flop induced by the exchange interaction ${\cal H}_{hMn}^{ex}$. However, as the splitting between the states $|+\frac{3}{2};\Uparrow_h\rangle$ and $|+\frac{5}{2};\Downarrow_h\rangle$ ($\Delta=4\times3/2I_{hMn}$) is large compared with the coupling term ($W=\sqrt{15}\frac{\rho_c}{\Delta_{lh}}I_{hMn}$), their coherent coupling is weak. With a large valence band mixing $\frac{\rho_c}{\Delta_{lh}}=0.1$ as observed in the dot discussed in this paper, this leads for the Mn-hole system initialized in the state $|+\frac{3}{2};\Uparrow_h\rangle$ to a fast oscillation of the population between the two corresponding Mn-hole ground states of the $\Lambda$ system with a maximum amplitude of about 1.6\% and an average population transfer efficiency of 0.8\%~\cite{CohenTannoudji}. Under resonant excitation on the high energy branch of the the $\Lambda$ system, the QD remains OFF more than 99$\%$ of the time. As we will see in the following, the contribution of this weak coherent population transfer to the calculated auto-correlation signal is not significant.

	$\rho_{|+\frac{3}{2};\uparrow_e\rangle}(t)$ obtained with the QD parameters listed in Tab.~\ref{paraQD} is presented in Fig.~\ref{AutocorModBPw} (a). This quantity has to be normalized by $\rho_{|+\frac{3}{2};\uparrow_e\rangle}(\infty)$) to directly account for the autocorrelation signal. After a fast increase, the calculated population presents a maximum at short delay. This model is based on a large number of parameters, whose values cannot all be extracted precisely from the measurements however, with reasonable spin relaxation parameters (see details in the caption of Fig.~\ref{AutocorModBPw}), the width and the amplitude of the maximum are in good agreement with the photon bunching signals observed experimentally.

	The width of the calculated bunching is controlled by all the spin-flip terms that can induce an escape out of the resonantly excited $\Lambda$ system. At zero transverse magnetic field, it is dominated by spin flips in the Mn-hole system. As illustrated in Fig.~\ref{AutocorModBPw} (a), suppressing $\tau_h$ gives a width of bunching only controlled by the Hamiltonian evolution and the decoherence which is slightly larger than what is observed experimentally (Fig.~\ref{AutocorExpBPw}).

	The dependence on the excitation intensity, $\tau_g$, and transverse magnetic field, $B_{\perp}$, are also qualitatively well reproduced by the model (Fig.~\ref{AutocorModBPw} (c) and (d) respectively). At zero magnetic field, the leaks outside the excited $\Lambda$ systems are dominated by $\tau_h$. $\mathcal{H}_{X^+Mn}$ induces fluctuations in a slightly longer time scale. The situation is different under a weak transverse magnetic field where the electron-Mn states are mixed introducing new channel of escape and significantly reducing the width of the photon bunching (See Fig.~\ref{AutocorExpBPw} for the corresponding experiments).

	Let us note that suppressing the fast flip-flop process connecting the two Mn-hole ground states ($\tau_{ff}=\infty$ in Fig.~\ref{AutocorModBPw} (b)) still produces a bunching as with the approximated initial condition used in the calculation ($\rho_{|+\frac{3}{2};\Uparrow_h\rangle}(0)=1$) a weak coherent transfer between the two ground states of the $\Lambda$ system still exist. However, with this process only, the calculated PL intensity is always more than 50 times smaller than with $\tau_{ff}$ and its contribution to the calculated auto-correlation signal (Fig.~\ref{AutocorModBPw} (a)) can be safely neglected.

	\begin{figure}[h!]
	\begin{center}
		\includegraphics[width=12cm]{Pictures/PumpSimu.eps}
	\end{center}
	\caption{Calculated resonant optical pumping transients for a $\sigma-$ detection and an excitation of $|3,+2\rangle$ and $|3,-2\rangle$ with modulated circular polarization. The QD parameters for the calculations are those listed in table \ref{paraQD} and $\tau_r$=0.3 ns, $\tau_{Mn}$=5 $\mu$s, $\tau_h$=10 ns, $T_2^{hMn}$= 5 ns, $T_2^{eMn}$= 0.5 ns, $\tau_{ff}$=1.5 ns, T=10 K and $\tau_g$=0.25 ns. (a) Influence of a variation of $\tau_g$ and $\tau_{ff}$. (b) Influence of a transverse magnetic field $B_{\perp}$. The inset presents the transverse magnetic field dependence of the difference of population for a $\sigma+$ or a $\sigma-$ excitation.}
	\label{PumpModBPw}
	\end{figure}
	
	With this model, we can also calculate the population of the electron-Mn states under resonant excitation with alternated circular polarization and estimate the efficiency and dynamics of the optical pumping. Fig.~\ref{PumpModBPw} presents the calculated time evolution of the population of the electron-Mn state $|+\frac{3}{2},\uparrow_e\rangle$ under alternated resonant excitation of $|3,+2\rangle$ in $\sigma+$ polarization or $|3,-2\rangle$ in $\sigma-$ polarization. This corresponds to the experimental configuration where the QD is resonantly excited with modulated circular polarization at the energy of $|3,+2\rangle$ and $|3,-2\rangle$ (absorption (2) in Fig.~\ref{LambdaLevel} (b)) and the low energy resonant PL is detected in $\sigma-$ polarisation. The main features of the time-resolved optical pumping experiments (see Fig.~\ref{AllPumpB0} and Fig.~\ref{PumpExpBPw}) are well reproduced by the model. The timescale of the pumping transient, in the few tens of nanosecond range, and its excitation intensity dependence are also in good agrement with the experiments (see figure~\ref{PumpModBPw} (a)).

	The influence of a transverse magnetic field, B$_{\perp}$, on the optical pumping transient can also be described by this model. First, a significant reduction of the pumping time is observed for a weak magnetic field (B$_{\perp}=0.2$ T in Fig.~\ref{PumpModBPw}(b)). As for the autocorrelation, this acceleration comes from the increase of the leakage out of the $\Lambda$ system induced by the mixing of the electron-Mn states. Secondly, the transients obtained when switching the polarization from $\sigma_{co}$ to $\sigma_{cross}$ and from $\sigma_{cross}$ to $\sigma_{co}$ become identical for B$_{\perp}\approx0.4$T, as observed in the experiments (Fig.~\ref{PumpExpBPw} (b)).

	To understand this behaviour under B$_{\perp}$, let us remember that we resonantly excite $|3,+2\rangle$ from $|+\frac{5}{2},\Downarrow_h\rangle$ with $\sigma+$ light and excite $|3,-2\rangle$ from $|-\frac{5}{2},\Uparrow_h\rangle$ with $\sigma-$ photons. In both cases we detect the population of $|+\frac{3}{2},\uparrow_e\rangle$ in $\sigma-$ polarization (see the excitation/detection configuration illustrated in the inset of Fig.~\ref{AllPumpB0} (b)). If the states $|3,+2\rangle$ and $|3,-2\rangle$ are uncoupled, as it is the case at zero field, we do not detect any light during the $\sigma-$ excitation. With a sufficiently large mixing of $|3,+2\rangle$ and $|3,-2\rangle$ induced by the transverse magnetic field, for a $\sigma-$ excitation of $|3,-2\rangle$, the population can be coherently transferred to $|3,+2\rangle$ during the charged exciton lifetime and $\sigma-$ light is detected after a recombination towards $|+\frac{3}{2},\Uparrow_h\rangle$ (see Sec.~\ref{StrainInfl}). In the optical pumping sequence, we can then observe, in $\sigma-$ polarization, a transient when the $\sigma+$ excitation empties the state $|+\frac{5}{2},\Downarrow_h\rangle$ but also a similar transient when the $\sigma-$ excitation empties the state $|-\frac{5}{2},\Uparrow_h\rangle$. The transverse magnetic field dependence of the difference of steady state intensity observed in $\sigma_{co}$ and $\sigma_{cross}$ polarization (inset of Fig.~\ref{PumpExpBPw} (b)) is also well reproduced by the model (inset of Fig.~\ref{PumpModBPw} (b)). This depolarization curve is controlled by the anisotropy of the electron-Mn spin which is induced by $\eta$ and $D_0$~\cite{DynhMn}. Let us note finally that, as expected, suppressing $\tau_{ff}$ from the model, a very weak average resonant PL and a fast optical pumping are obtained (Fig.~\ref{PumpModBPw} (a), top curve).
	
	\begin{figure}[h!]
	\begin{center}
		\includegraphics[width=13cm]{Pictures/RelaxDarkSimu.eps}
	\end{center}
	\caption{(a) Calculated time evolution in the dark of the population of the hole-Mn state $|+\frac{5}{2},\Downarrow_h\rangle$ initialized by a sequence of $\sigma-$/$\sigma+$ resonant excitation of $|3,-2\rangle$ and $|3,+2\rangle$. The dashed black line (shifted for clarity) is an exponential fit with a characteristic time $\tau_{relax}$=85 ns. (b) Corresponding calculated time evolution of the population $|+\frac{3}{2},\uparrow_e\rangle$. The parameters are those of Fig.~\ref{PumpModBPw}.}
	\label{PumpModDark}
	\end{figure}
	
	Including a dark time in the pumping sequence, we can also numerically evaluate the time required for the Mn-hole spin to return to the ground state of the excited $\Lambda$ system. The time evolution of the population of the Mn-hole state $|+5/2,\Downarrow_h\rangle$ initially prepared by a sequence of $\sigma-$/$\sigma+$ excitation resonant with $|3;+2\rangle$ (and $|3;-2\rangle$) is presented in Fig.~\ref{PumpModDark}. When the optical excitation is switched off, after an abrupt jump due to the optical recombination of the charge exciton, the ground Mn-hole state $|+5/2,\Downarrow_h\rangle$ is repopulated in a timescale of about 100 ns, much shorter than the Mn spin relaxation time used in the model ($\tau_{Mn}=5ns$). This relaxation is induced by the presence of valence band mixing. In the presence of valence-band mixing, $\mathcal{H}_{hMn}^{ex}$ couples two by two the different Mn-hole levels. This coupling induces a transfer of population between the different Mn-hole levels. The transfer of population becomes irreversible in the presence of dephasing and controls the observed Mn-hole spin relaxation~\cite{DynhMn}.

	\begin{figure}[h!]
	\begin{center}
		\includegraphics[width=12cm]{Pictures/StrainAutocorPump.eps}
	\end{center}
	\caption{(a) Calculated time evolution of $\rho_{|+\frac{1}{2},\uparrow_e\rangle}$ with $\rho_{|+\frac{1}{2},\Uparrow_h\rangle}$=1 (Mn-hole spin in the state $|+\frac{1}{2},\Uparrow_h\rangle$ after a $\sigma-$ recombination) for a resonant $\sigma+$ excitation of the coupled electron-Mn states $|3,+1\rangle$ and $|3,-1\rangle$ without and with a longitudinal magnetic field. (b) Time evolution of $\rho_{|+\frac{1}{2},\uparrow_e\rangle}$ under excitation with modulated circular polarization. The parameters used in the calculations are those of Fig.~\ref{PumpModBPw}.}
	\label{StrainAutocorPump}
	\end{figure}
	
	The particular behaviour observed for a resonant excitation of the electron-Mn states $|3,+1\rangle$ or $|3,-1\rangle$ (weak photon bunching and no optical pumping at zero field, Fig.~\ref{AllAutocorB0} (a) and Fig.~\ref{AllPumpB0} (a) respectively) is also qualitatively explained by the model (see Fig.~\ref{StrainAutocorPump}). The $|3,+1\rangle$ and $|3,-1\rangle$ states are degenerated and differ by a change of angular momentum of two. Consequently, they are efficiently mixed by the anisotropic strain term $E(S_y^2-S_x^2)$ which induces a spin-flip of two of the Mn with a conservation of the electron spin. This coupling has no significant influence on the other e-Mn states which are initially split by $D_0 S_z^2$ and $-2\eta S_z^2$.
	
	The splitting between the two new eigenstates, formed by the mixing of $|3,+1\rangle$ and $|3,-1\rangle$, in the $\mu eV$ range, is much weaker than width of the resonant laser used in our experiments (around 10 $\mu eV$) and the width of the optical transitions (around 50 $\mu eV$). Under circularly polarized resonant excitation we either excite $|3,+1\rangle$ with $\sigma+$ photons or $|3,-1\rangle$ with $\sigma-$ photons. At zero magnetic field, the population is transferred between the two states in a time scale of a few hundreds picoseconds (see Sec.~\ref{StrainInfl}). Under circularly polarized resonant excitation, the two $\Lambda$ systems associated with $|3,\pm1\rangle$ are simultaneously excited. For alternated circular polarization, a steady state is reached and no pumping transient induced by a leak outside the $\Lambda$ systems is expected. Under a weak longitudinal magnetic field the Mn Zeeman energy dominates the strain anisotropy term and the coherent transfer is blocked. The states $|3,+1\rangle$ and $|3,-1\rangle$ are decoupled and a large amplitude of bunching and an efficient optical pumping are restored. This behaviour observed in the experiments is qualitatively reproduced by the model.

	Let us finally note that in the modelling of optical pumping at zero magnetic field presented in Fig.~\ref{StrainAutocorPump} (b), fast oscillations are obtained in the first nanoseconds after the polarization switching. These are due to the population transfer between $|3,+1\rangle$ and $|3,-1\rangle$ in the excited state (directly coupled by E) during the coherence time. These oscillations are too fast to be observed in the experiments. The calculated resonant PL intensity in $\sigma-$ polarization (proportional to $\rho_{|+\frac{1}{2},\uparrow_e\rangle}$) is also slightly larger for a $\sigma+$ excitation than for a $\sigma-$ excitation. The $\sigma-$ resonant PL probes the population of $|3,+1\rangle$ which is directly excited by a resonant $\sigma+$ laser (see the excitation/detection configuration in the inset of Fig.~\ref{AllPumpB0} (a)). On the other hand, under a $\sigma-$ laser, one excites $|3,-1\rangle$ and the charged exciton has a probability to recombine before being transferred to $|3,+1\rangle$ and detected in $\sigma-$ PL. This transfer time results in a slight difference in the steady state resonant PL intensity obtained in a $\sigma_{Co}$ or $\sigma_{Cross}$ configuration (see Fig.~\ref{AllPumpB0} (a)).
	\newline

	\begin{figure}[h!]
	\caption{Energy levels of the ground (h-Mn) and excited ($X^+$-Mn) states as a function of their angular momentum (M$_z$). The e-Mn states $|3,+1\rangle$ and $|3,-1\rangle$, as well as $|2,+1\rangle$ and $|2,-1\rangle$, are coupled by the strain anisotropy $E(S_x^2-S_y^2)$. Optical $\Lambda$ systems associated with $|3,+1\rangle$ and $|3,-1\rangle$ are presented.}\label{LambdaMixed}
	{\begin{center}
		\includegraphics[width=9cm]{Pictures/SpinStructE.png}
	\end{center}}
	\end{figure}
	
	{\Large We saw the influence of strain anisotropy. We will now see a way to extract it more precisely.}\newline

	\section{Influence of the strain anisotropy\label{StrainInfl}}
	
	We exploited the $\Lambda$ level structure evidenced in Sec.~\ref{LambdaId} to analyze the coherent dynamics of the e-Mn spin through the time evolution of the circular polarization rate, $\kappa=(\sigma_{Cross}-\sigma_{Co})/(\sigma_{Cross}+\sigma_{Co})$, of the resonant PL. The configuration of the experiment is summarized in Fig.~\ref{PolarRateTotal} (a). Circularly polarized and spectrally filtered 10 ps laser pulses are successively tuned on resonance with the three absorption lines identified in the continuous wave experiment. This corresponds to independent optical excitation of the e-Mn states $|3,+1\rangle$, $|3,+2\rangle$ and $|2,+2\rangle$. The QD is excited with sequences of $\sigma+$/$\sigma-$ pulses (Fig.~\ref{PolarRateTotal} (b)), to avoid any possible optical spin pumping of h-Mn~\cite{DynhMn} that could influence the observed dynamics.
	
	\begin{figure}[h!]
	\begin{center}
		\includegraphics[width=12.8cm]{Pictures/PolarRateTotal.eps}
	\end{center}
	\caption{(a) Configuration of the time resolved PL experiment for an excitation of $|3,+1\rangle$ (pulsed laser in green). (b) Top panel: Time resolved resonant PL of $|3,+1\rangle$ with a $\sigma+$/$\sigma-$ sequence of laser pulses and a detection in $\sigma+$ and $\sigma-$ polarization. Bottom panel: corresponding time dependence of the circular polarization rate $\kappa=(\sigma_{-}-\sigma_{+})/(\sigma_{-}+\sigma_{+})$. (c) Time dependence of the circular polarization rate of the resonant PL of the states $|3,+1\rangle$ (red), $|3,+2\rangle$ (black) and $|2,+2\rangle$ (blue). (d) Corresponding polarisation rates calculated with $D_0=7 \mu eV$~\cite{DynhMn}, $T_2^{eMn}=0.6ns$, $E=1.8\mu eV$, a radiative lifetime $T_r=0.3ns$ and the parameters listed on Tab.~\ref{paraQD}.}
	\label{PolarRateTotal}
	\end{figure}
	
	The main result is the observation of an oscillatory behavior of the polarization rate of the PL when probing the dynamics of the $|3,+1\rangle$ state. The period of the beats is 270 ps with a characteristic damping time of 0.6 ns. When probing the dynamics of the $|3,+2\rangle$ and $|2,+2\rangle$ states, we measured cross circularly polarized PL with a slow decrease of the polarization rate during the lifetime of X$^+$-Mn.
	
	The origin of this dynamics lies in the mixing of the $|3,+1\rangle$ and $|3,-1\rangle$ states by the anisotropy of strain presented in Sec.~\ref{RelaxMech}. When a pulsed laser is tuned to the high energy transition of the $\Lambda$ system associated to $|3,+1\rangle$ ($\sigma+$ absorption from the h-Mn state $|+3/2\rangle|\Downarrow\rangle$), the PL of the low energy transition of the $\Lambda$ system is first cross-circularly polarized ($\sigma-$ recombination to the h-Mn state $|+1/2\rangle|\Uparrow\rangle$). Then, after a coherent transfer of population to the e-Mn state $|3,-1\rangle=\frac{1}{\sqrt{6}}(\sqrt{2}|-3/2\rangle|\uparrow\rangle+\sqrt{4}|-1/2\rangle|\downarrow\rangle)$, induced by $E$ (Fig.\ref{PolarRateTotal} (c)), co-circularly polarized PL is emitted at the same energy from the $|3,-1\rangle$ state ($\sigma+$ recombination to $|-1/2\rangle|\Downarrow\rangle$). This coherent transfer of population is fully controlled by the in-plane anisotropy of the strain at the Mn location and is responsible for the observed oscillations of the circular polarization rate.
	
	To understand the details of this dynamics, we calculated the time evolution of the populations and coherence of the twelve X$^+$-Mn states and the twelve hole-Mn states. We neglected here the hyperfine coupling between the electronic and nuclear spins of the Mn and solved the master equation for the 24 x 24 density matrix numerically, including relaxation and pure dephasing processes in the Lindblad form, as presented in Sec.~\ref{RelaxMech}.
	
	For the initial condition in the calculation of the time evolution, we consider that a $\sigma+$ pulse on resonance with the absorption line (1) (see Fig.~\ref{PolarRateTotal}) projects the system on the $M=3$ electron-Mn subspace on all the levels that contain a component $|+3/2\rangle |\downarrow\rangle$. In the absence of transverse magnetic field and strain anisotropy term E, this excitation simply corresponds to an optical transition from the hole-Mn state $|+3/2\rangle |\Downarrow\rangle$ towards the electron-Mn state $|3, +1\rangle$. With a weak transverse magnetic field (typically lower than 0.5 T), a linear combination of the $M=3$ states is created. At large transverse magnetic field, one should consider possible mixing with the $M=2$ states. Similarly, a $\sigma+$ pulse on (2) projects the system on the $M=3$ electron-Mn subspace on the levels that contain a component $|+5/2\rangle |\downarrow\rangle$ and a $\sigma+$ pulse resonant on (3) projects the system on the $M=2$ electron-Mn subspace on the levels that contain a component $|+5/2\rangle |\downarrow\rangle$.
	
	After this excitation, the circular polarization of the resonant photoluminescence is governed by the evolution of the spin of the electron. For instance, to compute the circular polarization rate of the emission after a resonant σ+ excitation on (1) (optical excitation from the hole-Mn state $|+3/2\rangle |\Downarrow\rangle$ to $|3, +1\rangle$ : high energy branch of the $\Lambda$-system) we calculate the difference between the density matrix elements $\rho_{|+1/2\rangle |\uparrow\rangle}$ ($\sigma-$ recombination towards the hole-Mn state $|+1/2\rangle |\Uparrow\rangle$ : low energy branch of the $\Lambda$-system) and $\rho_{|-1/2\rangle |\downarrow\rangle}$ ($\sigma+$ recombination towards the hole-Mn state $|-1/2\rangle |\Downarrow\rangle$ : low energy branch of the Λ-system associated with $|3, -1\rangle$  ).
	
	In the absence of magnetic field, the period of the quantum beats observed for an excitation of $|3,+1\rangle$ depends only on the anisotropy term $E$. The experimental data can be well reproduced by the model with $E=1.8\mu eV$ (Fig.\ref{PolarRateTotal} (d)). A coherence time, $T_2^{eMn}\approx0.6ns$, of the spin of e-Mn is extracted from the damping of the oscillations. For an excitation of $|3,+2\rangle$ and $|2,+2\rangle$ one can observe a slow decrease of the polarization rate which is also qualitatively reproduced by the model.

%	\begin{figure}[h!]
%	\begin{center}
%		\includegraphics[width=14.8cm]{Pictures/PolarRate31.eps}
%	\end{center}
%	\caption{(a) Configuration of the time resolved PL experiment for an excitation of $|3,+1\rangle$ (pulsed laser in green). (b) Top panel: Time resolved resonant PL of $|3,+1\rangle$ with a $\sigma+$/$\sigma-$ sequence of laser pulses and a detection in $\sigma+$ and $\sigma-$ polarization. Bottom panel: corresponding time dependence of the circular polarization rate $\kappa=(\sigma_{-}-\sigma_{+})/(\sigma_{-}+\sigma_{+})$.}
%	\label{31PolarRate}
%	\end{figure}
%
%	\lipsum[42-43]
%
%	\begin{figure}[h!]
%	\begin{center}
%		\includegraphics[width=14.8cm]{Pictures/PolarRateFull.eps}
%	\end{center}
%	\caption{(a) Time dependence of the circular polarization rate of the resonant PL of the states $|3,+1\rangle$ (red), $|3,+2\rangle$ (black) and $|2,+2\rangle$ (blue). (b) Corresponding polarisation rates calculated with $D_0=7 \mu eV$~\cite{DynhMn}, $T_2^{eMn}=0.6ns$, $E=1.8\mu eV$, a radiative lifetime $T_r=0.3ns$ and the parameters listed on Table~\ref{paraQD}.}
%	\label{PolarRateFull}
%	\end{figure}
		
%	\lipsum[44]

	\begin{figure}[h!]
	\begin{center}
		\includegraphics[width=14cm]{Pictures/PolarRateB.png}
	\end{center}
	\caption{(a) Influence of a longitudinal (B$_z$, red) and a transverse (B$_x$, blue) magnetic field on the time dependence of the circular polarization rate $\kappa=(\sigma_{-}-\sigma_{+})/(\sigma_{-}+\sigma_{+})$ of the resonant PL of $|3,+1\rangle$, $|3,+2\rangle$ and $|2,+2\rangle$. On the top left panel, curves are shifted by 0.5 for clarity. (b) Corresponding time dependence of the circular polarization rate calculated with $g_{Mn}=2$, $g_{e}=-0.4$, $g_{h}=0.6$~\cite{DynhMn}, and the parameters listed on Table~\ref{paraQD}. The curves are shifted by 1 for clarity.}
	\label{hMnPolarRateB}
	\end{figure}
	
	The coherent transfer of population depends both on the initial splitting of the e-Mn spin states (controlled at zero field by $D_0$ and $\eta$) and on the strength of the coupling $E$. The splitting between the e-Mn states can be tuned by a magnetic field, $B_z$, applied along the growth axis. In addition, a coupling between the e-Mn spin states $M_z$ can be induced by a magnetic field, $B_x$, applied in the QD plane. The experimental and calculated evolution of the polarization rate of the e-Mn states, $|3,+1\rangle$, $|3,+2\rangle$ and $|2,+2\rangle$, versus magnetic field are presented in Fig.~\ref{hMnPolarRateB}.

	Under a longitudinal magnetic field B$_z$, the e-Mn states M$_z=\pm1$ are split and the influence of $E$ is progressively reduced. For an excitation on the $|3,+1\rangle$ state, the amplitude and period of the oscillations in the polarization rate reduce as $B_z$ increases: The resonant PL becomes cross-circularly polarized with a polarization rate constant during the lifetime of $X^+$. A weak longitudinal magnetic field stabilizes the spin of the e-Mn states $|3,+2\rangle$ and $|2,+2\rangle$ and their polarization rate remains constant during the lifetime of X$^+$-Mn.

	In a transverse magnetic field B$_x$, the quantum beats observed for an excitation of $|3,+1\rangle$ are accelerated and the measured circular polarization rate drops to zero as the period of the oscillations becomes smaller than the time resolution of the experimental setup ($\approx 60$ ps). A given transverse magnetic field induces a slower oscillation of the polarization rate for the states $|3,+2\rangle$ and $|2,+2\rangle$.

	The observed magnetic field dependence of the coherent dynamics of $|3,+1\rangle$, $|3,+2\rangle$ and $|2,+2\rangle$ can be qualitatively reproduced by the model with the exchange parameters deduced from the PL and the strain anisotropy term and coherence time deduced from the oscillations observed on $|3,+1\rangle$ at zero magnetic field (Fig.\ref{PolarRateTotal}). D$_0$ cannot be extracted from these measurements and we use a typical value D$_0$=7 $\mu eV$ corresponding to a partial relaxation of the biaxial strain~\cite{DynhMn}. The different precession periods observed for the three states in a given transverse magnetic field are particularly well described.
	
	\begin{figure}[h!]
	\begin{center}
		\includegraphics[width=11.2cm]{Pictures/LevelSplitB.eps}
	\end{center}
	\caption{(Color line) (a) Calculated energy of the electron-M, states in a longitudinal magnetic field (B$_z$) and in a transverse magnetic field (B$_{\bot}$). (b) Energy of the electron-Mn states for two orientations of the transverse magnetic field: $\phi = 0$ (B$_{\bot} = $ B$_x$) $\phi = \dfrac{\pi}{2}$ (B$_{\bot} = $ B$_y$). The parameters used in the calculations are listed in Table~\ref{paraph}, with the exception of $E$, for which the more precise value of 1.8 $\mu$eV was chosen.}
	\label{hMnPolarRateBMod}
	\end{figure}
	
	The calculated time dependence of the circular polarization rate in transverse magnetic field are presented in Fig.~\ref{hMnPolarRateBMod} for the three $\Lambda$-systems identified in Fig.~\ref{LambdaLevel}. In agreement with the calculated electron-Mn energy levels (Fig.~\ref{PolarRateTotal}), this modelling reveals a significant influence of the orientation of the transverse magnetic field on the electron-Mn coherent dynamics. This behaviour could not be observed with our experimental set-up which did not permit to change the direction of the transverse field while staying on the same Mn-doped QD. A systematic experimental study of the transverse magnetic field effect has still to be realized.
		
	
\printbibliography

\end{document}