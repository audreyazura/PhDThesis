\documentclass[a4paper,12pt]{report}

\usepackage[utf8]{inputenc}
\usepackage[T1]{fontenc}
\usepackage{array}
\usepackage{amsmath}
\usepackage[english]{babel}
\usepackage{graphicx}
\usepackage[a4paper]{geometry}
\usepackage[colorlinks=true,urlcolor=blue,linkcolor=blue]{hyperref}
\usepackage{url}
\usepackage[nottoc,numbib]{tocbibind}
\usepackage{color}
\usepackage{epstopdf}
\usepackage{xcolor}
\usepackage[backend=biber,style=phys]{biblatex}
\usepackage{lipsum}
\usepackage[capbesideposition={right,center}]{floatrow}

\addbibresource{../Bibliography.bib}

\makeatletter
	\renewcommand{\thechapter}{\Roman{chapter}}
\makeatother

\floatsetup[table]{style=plaintop}

\begin{document}

\chapter{Coherent dynamics of Mn-doped positively charged quantum dots}

	\section{Mn in a II-VI positively charged quantum dot}

	Cf Optical control of the spin of a magnetic atom in a semiconductor QD, L. Besombes et. al., Sept 2014
		
		\subsection{Spin structure of a positively charged Mn doped quantum dot}
		
		Cf XplusMnRes.pptx to detail the e-Mn levels
		
		E included in model for generality (cite Claire paper "Resonant pumping..." \cite{OptControlSpin}). The effects will be discussed in depth later.\newline
	
	As presented in Sec.~\ref{MnSemiCon}, a Mn atom in a strained self-assembled CdTe QD exhibits a fine structure dominated by a weak magnetic anisotropy with an easy axis along the QD axis. Neglecting the tetrahedral crystal field of the CdTe matrix, this fine structure is described by the effective spin Hamiltonian

\begin{align}
\label{MnCF}
{\cal H}_{Mn,CF}=D_0S^2_z+E(S_y^2-S_x^2)
\end{align}
with $D_0$ depicting the effect of the biaxial strain and $E$ describing the anisotropy of the strain in the plane of the QD. It was shown that the anisotropy of strain was essential to understand the absence of pumping for Mn in strain-free quantum dots~\cite{OptControlSpin} and was thus include here to keep generality. We will study more in details its effect on Sec.~\ref{SpinDyn}.
		
	\begin{figure}[h!]
	{\caption{(a) Color scale plot of the PL intensity of the studied Mn doped QD inserted in Schottky structure showing the emission of the neutral (X-Mn) and positively charged (X$^+$-Mn) exciton as a function of energy and bias voltage. (b) PL of the Mn-doped QD under a positive bias voltage of V=5.5V.}\label{hMnspectra}}
	{\begin{center}
		\includegraphics[width=12cm]{Pictures/DotPres.eps}
	\end{center}}
	\end{figure}
	
	\lipsum[3-4]
		
	\begin{figure}[h!]
	\begin{center}
		\includegraphics[width=14.8cm]{Pictures/Recomb.png}
	\end{center}
	\caption{Electron-Mn spin states for each $|M,M_z\rangle$. For each $M$, the $\sigma-$ (red) and $\sigma+$ (blue) probability is highlighted. This probability is directly linked to the intensity of each peak. In the center, the different possible recombination path for $M=3$ and $M=2$ are presented. A schema of the resulting spectra is drawn below.}
	\label{Recomb}
	\end{figure}

	\lipsum[5]
	
	\begin{figure}[h!]
	\begin{center}
		\includegraphics[width=12cm]{Pictures/Spinstructv2.png}
	\end{center}
	\caption{(a) Energy levels of the ground (h-Mn) and excited ($X^+$-Mn) states as a function of their angular momentum (M$_z$). The levels in dotted lines corresponds to the h-Mn states $|-1/2\rangle|\Uparrow\rangle$ and $|+1/2\rangle|\Downarrow\rangle$ coupled by the valence band mixing. Optical recombination towards these levels leads to the linearly polarized lines observed in (b). (b) Experimental (left) and calculated (right) color-scale plot of the linear polarization dependence of the PL of X$^+$-Mn at B = 0 T (top) and B$_\perp$ = 0.42 T (bottom). The parameters used in the calculation are listed in Table~\ref{paraQD}.}
	\label{CompleteEnerStruct}
	\end{figure}
	
	\lipsum[7]
	
	\begin{table}[t] \centering
		\caption{Values of the parameters used in the model of the positively charged Mn-doped QD presented in Fig.~\ref{hMnspectra}. I$_{eMn}$, I$_{hMn}$, $\frac{\rho_s}{\Delta_{lh}}$, $\theta$, $\eta$ and $T_{eff}$ are used to model the linear polarization intensity map of Fig.~\ref{CompleteEnerStruct}. The other parameters cannot be extracted from the PL measurements and values for typical Mn-doped QDs are chosen for the calculation of the spin dynamics presented in Sec.~\ref{SpinDyn} and \ref{StrainInfl}.}
		\renewcommand{\arraystretch}{1.0}
		\begin{tabular}{cccccc|ccccc}
			\hline\hline
			I$_{eMn}$ & I$_{hMn}$ & $\frac{\rho_s}{\Delta_{lh}}$ & $\theta$    & $\eta$   & $T_{eff}$  & $g_{e}$ & $g_{h}$   	& $g_{Mn}$ & $D_0$    &  $E$      \\
			$\mu eV$  & $\mu eV$  &                              & $^{\circ}$  & $\mu eV$ &    K       &         &           &          & $\mu eV$ &  $\mu eV$ \\
			\hline
			-175    &     345   &        0.09                  &    0        &     30   &   20       &  -0,4   &  0.6      &     2    &    7     &   1.5     \\
			\hline\hline
		\end{tabular}
		\label{paraQD}
	\end{table}
	
%	\lipsum[11]

		\subsection{Optical $\Lambda$-level identification}	
		
		\lipsum[11]
	
	\begin{figure}[h!]
	\begin{center}
		\includegraphics[width=12cm]{Pictures/Lambdasyst.png}
	\end{center}
	\caption{(a) Non resonant (Non Res.) and resonant (Res.) PL of X$^+$-Mn. Co and cross circularly polarized PL spectra are collected for three different energies of the CW resonant laser (green). Inset: intensity map of the cross-circularly polarized PL detected on the low energy side of X$^+$-Mn as the CW laser is scanned through the high energy side. (b) Energy levels of X$^+$-Mn and identification of the three resonances observed in (a) corresponding to the optical $\Lambda$ systems associated with the e-Mn states $|3,+1\rangle$, $|3,+2\rangle$ and $|2,+2\rangle$.}
	\label{LambdaLevel}
	\end{figure}
	
	\lipsum[12]
			
	
	\section{Spin dynamics under resonant excitation\label{SpinDyn}}
	
		Cf article 2016/01
	
		\subsection{Cycling and escaping the $\lambda$-level system}

%		Mn in a lattice -> modification of orbital -> spin-orbit interaction. Magnetic anisotropy + anisotropy of strain. (Mn has nuclear spin 5/2 -> hyperfine interaction?)\newline

	Under resonant excitation of one high energy level of X$^+$-Mn, only one cross-circularly polarized emission line is observed. It corresponds to the optically allowed recombination on the second branch of the $\Lambda$ system. This recombination occurs with a flip-flop of the electron and Mn spins \cite{DynhMn}. The energy splitting between the resonant absorption and the emission corresponds to the splitting between the two ground states of the $\Lambda$ system. It is given by 4$\times$3/2$I_{hMn}$($\approx$2.1 meV for the studied QD) for an excitation of $|3,+2\rangle$ or $|2,+2\rangle$ and 2$\times$3/2$I_{hMn}$($\approx$1.05 meV for the studied QD) for an excitation of $|3,+1\rangle$. For an excitation of $|3,+2\rangle$ or $|2,+2\rangle$, the weak co-polarized PL signal, which depends on the excitation intensity, comes from a possible direct excitation of the low energy branch of the $\Lambda$ system through the acoustic phonon side-band \cite{BesombesAccPhon}.
	
	For an isolated $\Lambda$ system, under resonant excitation of one of the branch, a fast optical pumping controlled by the generation rate and the radiative lifetime of the excited state is expected: The population is expected to be stored in the level which is not excited and the resonant PL should vanish. In the case of X$^+$-Mn, the PL intensity observed under resonant excitation of the high energy branch of the $\Lambda$ systems is similar to the PL intensity obtained under non-resonant excitation. This suggests a very inefficient optical pumping of the Mn-hole spin and an efficient spin-flip mechanism which links the two ground states of the $\Lambda$ systems.
	
	\begin{figure}[h!]
	\begin{center}
		\includegraphics[width=14.8cm]{Pictures/ResAutocor.eps}
	\end{center}
	\caption{Auto-correlation of the resonant PL for cross-circularly polarized excitation and detection of the electron-Mn states (a) $|3, +1\rangle$, (b) $|3, +2\rangle$ and (c) $|2, +2\rangle$.}
	\label{AllAutocorB0}
	\end{figure}
	
	The dynamics of the Mn spin coupled to carriers was first analyzed, under resonant optical excitation, through the statistics of the time arrival of the photons given by the second order correlation function of the resonant PL intensity, $g^{(2)}(\tau)$. For the three resonant excitation conditions reported in Fig.\ref{AllAutocorB0}, $g^{(2)}(\tau)$ is mainly characterized by a large photon bunching with a full width at half maximum (FWHM) in the 20 ns range. The amplitude of the bunching reaches 9 for line (2) and is slightly weaker for the two other lines. This large bunching, reflecting an intermittency in the emission of the QD, is not sensitive to a longitudinal magnetic field B$_z$ except for an excitation on (1).

The presence of a photon bunching is at first sight surprising: under resonant excitation of an isolated $\Lambda$ system, an anti-bunching of the resonant PL controlled by the transfer time between the two ground states is indeed expected. For X$^+$-Mn, the observed short anti-bunching (dip near zero delay, better evidenced in Fig.~\ref{AllAutocorB0} (b)) suggests a fast transfer time in the nanosecond range between the two ground states of the $\Lambda$ systems.

In the presence of a transfer process connecting the two Mn-hole ground states in a nanosecond time-scale, the photon bunching can be explained by leaks outside the resonantly excited $\Lambda$ system. Under $cw$ excitation, the population is cycled inside the $\Lambda$ system until a spin flip occurs and drives the carrier-Mn spin out of the $\Lambda$ levels under investigation. The resonant PL is then switched off until multiple spin-flips drives back the carriers and Mn spin inside the $\Lambda$ system under excitation. The selected QD line can be either in a ON or OFF state depending on the fluctuations of the carrier and Mn spins. The amplitude of the bunching is then given by $\Gamma_{Out}/\Gamma_{In}$ the ratio of the transition rates from OFF to ON ($\Gamma_{In}$) and from ON to OFF ($\Gamma_{Out}$). An amplitude of bunching larger than 1 is expected for the multilevel system considered here where, after a spin relaxation, multiple spin flips are in average required to come back to the initial state ($\Gamma_{In}<\Gamma_{Out}$). Within this picture, the width of the bunching is a measurement of the escape time out of the considered $\Lambda$ level system.

	\begin{figure}[h!]
	\begin{center}
		\includegraphics[width=14.7cm]{Pictures/ResPumpv2.png}
	\end{center}
	\caption{Resonant optical pumping transients obtained under circular polarization switching of the resonant excitation for the $\Lambda$ systems associated with (a) $|3, +1\rangle$, (b) $|3, +2\rangle$ and (c) $|2, +2\rangle$ at zero field and under a weak longitudinal magnetic field B$_z$=0.23T. The insets present the corresponding states which are resonantly excited and detected in $\sigma-$ polarization.}
	\label{AllPumpB0}
	\end{figure}
	
	\lipsum[16]
	
	\begin{figure}[h!]
	\fcapside{\caption{Schema of the energy levels of the optical $\Lambda$ system associated with the electron-Mn state $|3, +2\rangle$ extracted from the full level structure of a positively charged Mn-doped QD (Fig.~\ref{LambdaLevel}). The different processes discussed in this section are presented on it.}\label{LambdLoop}}	
	{\begin{center}
		\includegraphics[width=6cm]{Pictures/RelaxMecanism.png}
	\end{center}}
	\end{figure}
	
		\lipsum[18-19]
	
		\begin{figure}[h!]
			\begin{center}
				\includegraphics[width=12cm]{Pictures/AutocorBPw.eps}
			\end{center}
			\caption{Excitation power dependence (a) and transverse magnetic field dependence (b) of the auto-correlation of the resonant PL obtained for an excitation on the high energy branch of the $\Lambda$ level system associated to the e-Mn state $|2,+2\rangle$.}
			\label{AutocorExpBPw}
		\end{figure}
	
	\lipsum[22-23]

	\begin{figure}[h!]
	\begin{center}
		\includegraphics[width=12cm]{Pictures/PumpBPw.eps}
	\end{center}
	\caption{Excitation power dependence (a) and transverse magnetic field dependence (b) of the optical pumping signal obtained for a resonant excitation on $|3,+2\rangle$. Insets: excitation power dependence of the pumping time and transverse magnetic field dependence of the difference of resonant PL intensity between a $\sigma_{cross}$ and a $\sigma_{co}$ excitation.}
	\label{PumpExpBPw}
	\end{figure}
	
	\lipsum[24]
	
	\begin{figure}[h!]
	\begin{center}
		\includegraphics[width=13cm]{Pictures/DarkRelax.png}
	\end{center}
	\caption{Optical pumping experiment for an excitation of $|3,+2\rangle$ with modulated circular polarization. A dark time ($\tau_{dark} = 50ns$) is introduced in the pumping sequence. The polarization switching occurs either before (black) or during (red) the dark time. The black and red diagrams present the corresponding resonant excitation sequences. The inset presents the variation of the ratio $\Delta I/I$ as a function of $\tau_{dark}$. The solid line is an exponential fit with $\tau_{relax} = 80 ns$.}
	\label{PumpExpDark}
	\end{figure}
	
		\subsection{Relaxation mechanism}
	
			\subsubsection*{Relaxation through a hole:}		
		
		\lipsum[26]
	
		\begin{table}[hbt] \centering
			\caption{Material (CdTe or ZnTe) \cite{CdTeBPCoef} and QD parameters used in the calculation of the coupled hole and Mn spin relaxation time.}
			\begin{tabular}{lcr}
				\hline\hline
				CdTe& &\\
				\hline
				Deformation potential constants & b &  -1.0 eV  \\
												& d &  -4.4 eV  \\
				Longitudinal sound speed & c$_l$ &  3300 m/s  \\
				Transverse sound speed & c$_t$ &  1800 m/s  \\
				Density & $\rho$ &  5860 kg/m$^3$  \\
				\hline
				ZnTe& &\\
				\hline
				Deformation potential constants & b &  -1.4 eV  \\
												& d &  -4.4 eV  \\
				Longitudinal sound speed & c$_l$ &  3800 m/s  \\
				Transverse sound speed & c$_t$ &  2300 m/s  \\
				Density & $\rho$ &  5908 kg/m$^3$  \\
				\hline
				Quantum dot& &\\
				\hline
				Hole Mn exchange energy & I$_{hMn}$ &  0.35 meV  \\
				hh-lh exciton splitting&  $\Delta_{lh}$&  15 meV  \\
				Hole wave function widths: & &   \\
				- in plane & l$_{\bot}$ &3.0 nm   \\
				- z direction  & l$_z$ &1.25 nm   \\
				\hline\hline
			\end{tabular}
			\label{paraph}
		\end{table}

	\lipsum[28]
	
	\begin{figure}[h!]
	\fcapside{\caption{Relaxation time $\tau_{ff}$, between the two Mn-hole ground states of the $\Lambda$ system  calculated with the material and QD parameters listed in Table \ref{paraph} and a temperature T=7K. The vertical line shows the energy splitting in the studied QD of the Mn-hole states involved in the $\Lambda$ systems considered here (Resonances (2) and (3) identified in Fig.~\ref{LambdaLevel}).}\label{TauRelax}}
	{\begin{center}
		\includegraphics[width=7cm]{Pictures/TauffRelax.eps}
	\end{center}}
	\end{figure}
	
		\subsubsection*{Modelization of the relaxation of Mn spin in a positively charged quantum dot:"edx}
	
	\lipsum[30]	

	\begin{figure}[h!]
	\begin{center}
		\includegraphics[width=10cm]{Pictures/AutocorSimu.eps}
	\end{center}
	\caption{(a) Calculated time evolution of $\rho_{|+\frac{3}{2},\uparrow_e\rangle}(t)$ with the QD parameters listed in Table~\ref{paraQD} and (unless specified) $\tau_r$=0.3ns, $\tau_{Mn}$=5 $\mu$s, $\tau_h$=10ns, $\tau_g$=0.25 ns, $\tau_{ff}$=1.5 ns, $T_2^{hMn}$= 5 ns, $T_2^{eMn}$= 0.5 ns, T=10K and B$_{\perp}$=0. (b) (c) and (d) illustrate the influence  of, respectively, $\tau_{ff}$, $\tau_g$ and $B_{\perp}$ on $\rho_{|+\frac{3}{2},\uparrow_e\rangle}(t)$. Note the different vertical scale in (b).}
	\label{AutocorModBPw}
	\end{figure}

	\lipsum[32]

	\begin{figure}[h!]
	\begin{center}
		\includegraphics[width=12cm]{Pictures/PumpSimu.eps}
	\end{center}
	\caption{Calculated resonant optical pumping transients for a $\sigma-$ detection and an excitation of $|3,+2\rangle$ and $|3,-2\rangle$ with modulated circular polarization. The QD parameters for the calculations are those listed in table \ref{paraQD} and $\tau_r$=0.3 ns, $\tau_{Mn}$=5 $\mu$s, $\tau_h$=10 ns, $T_2^{hMn}$= 5 ns, $T_2^{eMn}$= 0.5 ns, $\tau_{ff}$=1.5 ns, T=10 K and $\tau_g$=0.25 ns. (a) Influence of a variation of $\tau_g$ and $\tau_{ff}$. (b) Influence of a transverse magnetic field $B_{\perp}$. The inset presents the transverse magnetic field dependence of the difference of population for a $\sigma+$ or a $\sigma-$ excitation.}
	\label{PumpModBPw}
	\end{figure}
	
	\lipsum[34-35]
	
	\begin{figure}[h!]
	\begin{center}
		\includegraphics[width=13cm]{Pictures/RelaxDarkSimu.eps}
	\end{center}
	\caption{(a) Calculated time evolution in the dark of the population of the hole-Mn state $|+\frac{5}{2},\Downarrow_h\rangle$ initialized by a sequence of $\sigma-$/$\sigma+$ resonant excitation of $|3,-2\rangle$ and $|3,+2\rangle$. The dashed black line (shifted for clarity) is an exponential fit with a characteristic time $\tau_{relax}$=85 ns. (b) Corresponding calculated time evolution of the population $|+\frac{3}{2},\uparrow_e\rangle$. The parameters are those of Fig.~\ref{PumpModBPw}.}
	\label{PumpModDark}
	\end{figure}
	
	\lipsum[36-38]

	\begin{figure}[h!]
	\begin{center}
		\includegraphics[width=12cm]{Pictures/StrainAutocorPump.eps}
	\end{center}
	\caption{(a) Calculated time evolution of $\rho_{|+\frac{1}{2},\uparrow_e\rangle}$ with $\rho_{|+\frac{1}{2},\Uparrow_h\rangle}$=1 (Mn-hole spin in the state $|+\frac{1}{2},\Uparrow_h\rangle$ after a $\sigma-$ recombination) for a resonant $\sigma+$ excitation of the coupled electron-Mn states $|3,+1\rangle$ and $|3,-1\rangle$ without and with a longitudinal magnetic field. (b) Time evolution of $\rho_{|+\frac{1}{2},\uparrow_e\rangle}$ under excitation with modulated circular polarization. The parameters used in the calculations are those of Fig.~\ref{PumpModBPw}.}
	\label{StrainAutocorPump}
	\end{figure}
	
	\lipsum[48-50]
	a\newline

	\begin{figure}[h!]
	\caption{Energy levels of the ground (h-Mn) and excited ($X^+$-Mn) states as a function of their angular momentum (M$_z$). The e-Mn states $|3,+1\rangle$ and $|3,-1\rangle$, as well as $|2,+1\rangle$ and $|2,-1\rangle$, are coupled by the strain anisotropy $E(S_x^2-S_y^2)$. Optical $\Lambda$ systems associated with $|3,+1\rangle$ and $|3,-1\rangle$ are presented.}\label{LambdaMixed}
	{\begin{center}
		\includegraphics[width=10cm]{Pictures/SpinStructE.png}
	\end{center}}
	\end{figure}
	
	{\Large We saw the influence of strain anisotropy. We will now see a way to extract it more precisely.}\newline

	\section{Influence of the strain anisotropy\label{StrainInfl}}
	
%	\lipsum[38]
	
	\begin{figure}[h!]
	\begin{center}
		\includegraphics[width=12.8cm]{Pictures/PolarRateTotal.eps}
	\end{center}
	\caption{(a) Configuration of the time resolved PL experiment for an excitation of $|3,+1\rangle$ (pulsed laser in green). (b) Top panel: Time resolved resonant PL of $|3,+1\rangle$ with a $\sigma+$/$\sigma-$ sequence of laser pulses and a detection in $\sigma+$ and $\sigma-$ polarization. Bottom panel: corresponding time dependence of the circular polarization rate $\kappa=(\sigma_{-}-\sigma_{+})/(\sigma_{-}+\sigma_{+})$. (c) Time dependence of the circular polarization rate of the resonant PL of the states $|3,+1\rangle$ (red), $|3,+2\rangle$ (black) and $|2,+2\rangle$ (blue). (d) Corresponding polarisation rates calculated with $D_0=7 \mu eV$~\cite{DynhMn}, $T_2^{eMn}=0.6ns$, $E=1.8\mu eV$, a radiative lifetime $T_r=0.3ns$ and the parameters listed on Table~\ref{paraQD}.}
	\label{PolarRateTotal}
	\end{figure}

%	\begin{figure}[h!]
%	\begin{center}
%		\includegraphics[width=14.8cm]{Pictures/PolarRate31.eps}
%	\end{center}
%	\caption{(a) Configuration of the time resolved PL experiment for an excitation of $|3,+1\rangle$ (pulsed laser in green). (b) Top panel: Time resolved resonant PL of $|3,+1\rangle$ with a $\sigma+$/$\sigma-$ sequence of laser pulses and a detection in $\sigma+$ and $\sigma-$ polarization. Bottom panel: corresponding time dependence of the circular polarization rate $\kappa=(\sigma_{-}-\sigma_{+})/(\sigma_{-}+\sigma_{+})$.}
%	\label{31PolarRate}
%	\end{figure}
%
%	\lipsum[42-43]
%
%	\begin{figure}[h!]
%	\begin{center}
%		\includegraphics[width=14.8cm]{Pictures/PolarRateFull.eps}
%	\end{center}
%	\caption{(a) Time dependence of the circular polarization rate of the resonant PL of the states $|3,+1\rangle$ (red), $|3,+2\rangle$ (black) and $|2,+2\rangle$ (blue). (b) Corresponding polarisation rates calculated with $D_0=7 \mu eV$~\cite{DynhMn}, $T_2^{eMn}=0.6ns$, $E=1.8\mu eV$, a radiative lifetime $T_r=0.3ns$ and the parameters listed on Table~\ref{paraQD}.}
%	\label{PolarRateFull}
%	\end{figure}
		
%	\lipsum[44]

	\begin{figure}[h!]
	\begin{center}
		\includegraphics[width=14cm]{Pictures/PolarRateB.png}
	\end{center}
	\caption{(a) Influence of a longitudinal (B$_z$, red) and a transverse (B$_x$, blue) magnetic field on the time dependence of the circular polarization rate $\kappa=(\sigma_{-}-\sigma_{+})/(\sigma_{-}+\sigma_{+})$ of the resonant PL of $|3,+1\rangle$, $|3,+2\rangle$ and $|2,+2\rangle$. On the top left panel, curves are shifted by 0.5 for clarity. (b) Corresponding time dependence of the circular polarization rate calculated with $g_{Mn}=2$, $g_{e}=-0.4$, $g_{h}=0.6$~\cite{DynhMn}, and the parameters listed on Table~\ref{paraQD}. The curves are shifted by 1 for clarity.}
	\label{hMnPolarRateB}
	\end{figure}
	
	\lipsum[46]
	
	\begin{figure}[h!]
	\begin{center}
		\includegraphics[width=11.2cm]{Pictures/LevelSplitB.eps}
	\end{center}
	\caption{(Color line) (a) Calculated energy of the electron-M, states in a longitudinal magnetic field (B$_z$) and in a transverse magnetic field (B$_{\bot}$). (b) Energy of the electron-Mn states for two orientations of the transverse magnetic field: $\phi = 0$ (B$_{\bot} = $ B$_x$) $\phi = \dfrac{\pi}{2}$ (B$_{\bot} = $ B$_y$). The parameters used in the calculations are listed in Table~\ref{paraph}, with the exception of $E$, for which the more precise value of 1.8 $\mu$eV was chosen.}
	\label{hMnPolarRateB}
	\end{figure}
	
	\lipsum[47]
		
	
\printbibliography

\end{document}