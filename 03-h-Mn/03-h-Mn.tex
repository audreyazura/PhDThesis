\documentclass[a4paper,12pt]{report}

\usepackage[utf8]{inputenc}
\usepackage[T1]{fontenc}
\usepackage{array}
\usepackage{amsmath}
\usepackage[english]{babel}
\usepackage{graphicx}
\usepackage[a4paper]{geometry}
\usepackage[colorlinks=true,urlcolor=blue,linkcolor=blue]{hyperref}
\usepackage{url}
\usepackage[nottoc,numbib]{tocbibind}
\usepackage{color}
\usepackage{epstopdf}
\usepackage{xcolor}
\usepackage[backend=biber,style=phys]{biblatex}
\usepackage{lipsum}
\usepackage[capbesideposition={right,center}]{floatrow}

\addbibresource{../Bibliography.bib}

\makeatletter
	\renewcommand{\thechapter}{\Roman{chapter}}
\makeatother

\floatsetup[table]{style=plaintop}

\begin{document}

\chapter{Coherent dynamics of Mn-doped positively charged quantum dots}

	\section{Mn in a II-VI positively charged quantum dot}

	Cf Optical control of the spin of a magnetic atom in a semiconductor QD, L. Besombes et. al., Sept 2014
		
		\subsection{Spin structure of a positively charged Mn doped quantum dot}
		
		Cf XplusMnRes.pptx to detail the e-Mn levels
	
	\lipsum[1]
		
	\begin{figure}[h!]
	\fcapside{\caption{(a) Color scale plot of the PL intensity of the studied Mn doped QD inserted in Schottky structure showing the emission of the neutral (X-Mn) and positively charged (X$^+$-Mn) exciton as a function of energy and bias voltage. (b) PL of the Mn-doped QD under a positive bias voltage of V=5.5V. Inset: Scheme of the energy levels of the ground (h-Mn) and excited states (X$^+$-Mn) in a positively charged Mn-doped QD as a function of their angular momentum ($M_z$). (c) Experimental (left) and calculated (right) color-scale plot of the linear polarization dependence of the PL of X$^+$-Mn at B = 0 T (top) and B$_\perp$ = 0.42 T (bottom). The parameters used in the calculation are listed in table \ref{paraQD}.}\label{hMnspectra}}
	{\begin{center}
		\includegraphics[width=7.4cm]{Pictures/DotPres.eps}
	\end{center}}
	\end{figure}
	
	\begin{table}[t] \centering
		\caption{Values of the parameters used in the model of the positively charged Mn-doped QD presented in figure 1. I$_{eMn}$, I$_{hMn}$, $\frac{\rho_s}{\Delta_{lh}}$, $\theta$, $\eta$ and $T_{eff}$ are used to model the linear polarization intensity map of Fig.~\ref{Fig1}. The other parameters cannot be extracted from the PL measurements and values for typical Mn-doped QDs are chosen for the calculation of the spin dynamics presented in section VI.}
		\renewcommand{\arraystretch}{1.0}
		\begin{tabular}{cccccc|ccccc}
			\hline\hline
			I$_{eMn}$ & I$_{hMn}$ & $\frac{\rho_s}{\Delta_{lh}}$ & $\theta$    & $\eta$   & $T_{eff}$  & $g_{e}$ & $g_{h}$   	& $g_{Mn}$ & $D_0$    &  $E$      \\
			$\mu eV$  & $\mu eV$  &                              & $^{\circ}$  & $\mu eV$ &    K       &         &           &          & $\mu eV$ &  $\mu eV$ \\
			\hline
			-175    &     345   &        0.09                  &    0        &     30   &   20       &  -0,4   &  0.6      &     2    &    7     &   1.5     \\
			\hline\hline
		\end{tabular}
		\label{paraQD}
	\end{table}	
	
	\lipsum[2]
		
	\begin{figure}[h!]
	\begin{center}
		\includegraphics[width=10cm]{../FillingPicture.png}
	\end{center}
	\caption{Mn in charged QD simple energy structure}
	\label{SimpleEnerStruct}
	\end{figure}

	\lipsum[3]
	
	\begin{figure}[h!]
	\begin{center}
		\includegraphics[width=10cm]{../FillingPicture.png}
	\end{center}
	\caption{Energy structure of h-Mn/X+-Mn with valence band mixing, perturbative two holes, with the linear polarization as an example (experiment + model)}
	\label{CompleteEnerStruct}
	\end{figure}
	
	\lipsum[4]
	
	\begin{figure}[h!]
	\begin{center}
		\includegraphics[width=10cm]{../FillingPicture.png}
	\end{center}
	\caption{Linear polarization modelization with variation of parameter to show influence.}
	\label{LinPolModelMn}
	\end{figure}
		
		\subsection{Optical $\lambda$-level identification}	
		
		\lipsum[5]

	\begin{figure}[h!]
	\begin{center}
		\includegraphics[width=10cm]{../FillingPicture.png}
	\end{center}
	\caption{Luminescence under laser scan (map)}
	\label{ResPLE}
	\end{figure}

	\lipsum[6]
	
	\begin{figure}[h!]
	\begin{center}
		\includegraphics[width=10cm]{Pictures/Lambdasyst.png}
	\end{center}
	\caption{((a) Non resonant (Non Res.) and resonant (Res.) PL of X$^+$-Mn. Co and cross circularly polarized PL spectra are collected for three different energies of the CW resonant laser (green). (b) Energy levels of X$^+$-Mn and identification of the three resonances observed in (a) corresponding to the optical $\Lambda$ systems associated with the e-Mn states $|3,+1\rangle$, $|3,+2\rangle$ and $|2,+2\rangle$.}
	\label{LambdaLevem}
	\end{figure}
	
	\lipsum[7]
			
	
	\section{Spin dynamics under resonant excitation}
	
		Cf article 2016/01
	
		\subsection{Cycling and escaping the $\lambda$-level system}

		Mn in a lattice -> modification of orbital -> spin-orbit interaction. Magnetic anisotropy + anisotropy of strain. (Mn has nuclear spin 5/2 -> hyperfine interaction?)\newline
		
	\lipsum[8]

	\begin{figure}[h!]
	\begin{center}
		\includegraphics[width=10cm]{../FillingPicture.png}
	\end{center}
	\caption{Isolated $\lambda$-system with the loop (excitation, recombination, relaxation to initial state)}
	\label{LambdLoop}
	\end{figure}
		
	\lipsum[9]

	\begin{figure}[h!]
	\begin{center}
		\includegraphics[width=10cm]{../FillingPicture.png}
	\end{center}
	\caption{Pumping experiment on each $\lambda$-system}
	\label{AllPumpB0}
	\end{figure}

	\lipsum[10]
	
	\begin{figure}[h!]
	\begin{center}
		\includegraphics[width=10cm]{../FillingPicture.png}
	\end{center}
	\caption{Autocorrelation experiment on each $\lambda$-system}
	\label{AllAutocorB0}
	\end{figure}
	
	\lipsum[11]
	
		\subsection{Relaxation mechanism}
	
	\lipsum[12]

	\begin{figure}[h!]
	\begin{center}
		\includegraphics[width=10cm]{../FillingPicture.png}
	\end{center}
	\caption{Autocorrelation evolution under magnetic field and power variation - experimental result}
	\label{AutocorExpBPw}
	\end{figure}

	\lipsum[13]

	\begin{figure}[h!]
	\begin{center}
		\includegraphics[width=10cm]{../FillingPicture.png}
	\end{center}
	\caption{Pumping evolution under magnetic field and power variation - experimental result}
	\label{PumpExpBPw}
	\end{figure}
	
	\lipsum[14]

	\begin{figure}[h!]
	\begin{center}
		\includegraphics[width=10cm]{../FillingPicture.png}
	\end{center}
	\caption{Autocorrelation evolution under magnetic field and power variation - model}
	\label{AutocorModBPw}
	\end{figure}

	\lipsum[15]

	\begin{figure}[h!]
	\begin{center}
		\includegraphics[width=10cm]{../FillingPicture.png}
	\end{center}
	\caption{Pumping evolution under magnetic field and power variation - model}
	\label{PumpModBPw}
	\end{figure}
	
	\lipsum[16]
	

	\section{Influence of the strain anisotropy}
	
	\lipsum[17]
		
	\begin{figure}[h!]
	\begin{center}
		\includegraphics[width=10cm]{../FillingPicture.png}
	\end{center}
	\caption{Energy structure with |3, +1> and |3, -1>, and |2, +1> and |2, -1> coupled by E}
	\label{LambdaMixed}
	\end{figure}

	\lipsum[18]

	\begin{figure}[h!]
	\begin{center}
		\includegraphics[width=10cm]{../FillingPicture.png}
	\end{center}
	\caption{Experiment configuration |3, +1>  + Polarization decline and polar rate}
	\label{hMnPolarRate}
	\end{figure}
		
		\lipsum[19]

	\begin{figure}[h!]
	\begin{center}
		\includegraphics[width=10cm]{../FillingPicture.png}
	\end{center}
	\caption{Schema of the QD with spin and magnetic field orientation, and action of the magnetic field on the spin.}
	\label{QDMagField}
	\end{figure}

	\lipsum[20]

	\begin{figure}[h!]
	\begin{center}
		\includegraphics[width=10cm]{../FillingPicture.png}
	\end{center}
	\caption{Polarization rate evolution in B(x and z) and simulation}
	\label{hMnPolarRateB}
	\end{figure}
	
	\lipsum[21]
	
	
\printbibliography

\end{document}