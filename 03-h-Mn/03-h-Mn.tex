\documentclass[a4paper,12pt]{report}

\usepackage[utf8]{inputenc}
\usepackage[T1]{fontenc}
\usepackage{array}
\usepackage{amsmath}
\usepackage[english]{babel}
\usepackage{graphicx}
\usepackage[a4paper]{geometry}
\usepackage[colorlinks=true,urlcolor=blue,linkcolor=blue]{hyperref}
\usepackage{url}
\usepackage[nottoc,numbib]{tocbibind}
\usepackage{color}
\usepackage{epstopdf}
\usepackage{xcolor}
\usepackage[backend=biber,style=phys]{biblatex}
\usepackage{lipsum}
\usepackage[capbesideposition={right,center}]{floatrow}

\addbibresource{../Bibliography.bib}

\makeatletter
	\renewcommand{\thechapter}{\Roman{chapter}}
\makeatother

\floatsetup[table]{style=plaintop}

\begin{document}

\chapter{Coherent dynamics of Mn-doped positively charged quantum dots}

	\section{Mn in a II-VI positively charged quantum dot}

	Cf Optical control of the spin of a magnetic atom in a semiconductor QD, L. Besombes et. al., Sept 2014
		
		\subsection{Spin structure of a positively charged Mn doped quantum dot}
		
		Cf XplusMnRes.pptx to detail the e-Mn levels
	
	\lipsum[1]
		
	\begin{figure}[h!]
	{\caption{(a) Color scale plot of the PL intensity of the studied Mn doped QD inserted in Schottky structure showing the emission of the neutral (X-Mn) and positively charged (X$^+$-Mn) exciton as a function of energy and bias voltage. (b) PL of the Mn-doped QD under a positive bias voltage of V=5.5V. Inset: Scheme of the energy levels of the ground (h-Mn) and excited states (X$^+$-Mn) in a positively charged Mn-doped QD as a function of their angular momentum ($M_z$).}\label{hMnspectra}}
	{\begin{center}
		\includegraphics[width=10cm]{Pictures/DotPres.eps}
	\end{center}}
	\end{figure}
	
	\lipsum[2]
		
	\begin{figure}[h!]
	\begin{center}
		\includegraphics[width=14.8cm]{Pictures/Recomb.png}
	\end{center}
	\caption{Mn in charged QD simple energy structure}
	\label{Recomb}
	\end{figure}

	\lipsum[3]
	
	\begin{figure}[h!]
	\begin{center}
		\includegraphics[width=10cm]{Pictures/Spinstructv2.png}
	\end{center}
	\caption{Energy structure of h-Mn/X+-Mn with valence band mixing, perturbative two holes, with the linear polarization as an example (experiment + model) (b) Experimental (left) and calculated (right) color-scale plot of the linear polarization dependence of the PL of X$^+$-Mn at B = 0 T (top) and B$_\perp$ = 0.42 T (bottom). The parameters used in the calculation are listed in table \ref{paraQD}.}
	\label{CompleteEnerStruct}
	\end{figure}
	
	\lipsum[9]
	
	\begin{table}[t] \centering
		\caption{Values of the parameters used in the model of the positively charged Mn-doped QD presented in figure 1. I$_{eMn}$, I$_{hMn}$, $\frac{\rho_s}{\Delta_{lh}}$, $\theta$, $\eta$ and $T_{eff}$ are used to model the linear polarization intensity map of Fig.~\ref{Fig1}. The other parameters cannot be extracted from the PL measurements and values for typical Mn-doped QDs are chosen for the calculation of the spin dynamics presented in section VI.}
		\renewcommand{\arraystretch}{1.0}
		\begin{tabular}{cccccc|ccccc}
			\hline\hline
			I$_{eMn}$ & I$_{hMn}$ & $\frac{\rho_s}{\Delta_{lh}}$ & $\theta$    & $\eta$   & $T_{eff}$  & $g_{e}$ & $g_{h}$   	& $g_{Mn}$ & $D_0$    &  $E$      \\
			$\mu eV$  & $\mu eV$  &                              & $^{\circ}$  & $\mu eV$ &    K       &         &           &          & $\mu eV$ &  $\mu eV$ \\
			\hline
			-175    &     345   &        0.09                  &    0        &     30   &   20       &  -0,4   &  0.6      &     2    &    7     &   1.5     \\
			\hline\hline
		\end{tabular}
		\label{paraQD}
	\end{table}
	
	\lipsum[4]

		\subsection{Optical $\lambda$-level identification}	
		
		\lipsum[5]
	
	\begin{figure}[h!]
	\begin{center}
		\includegraphics[width=10cm]{Pictures/Lambdasyst.png}
	\end{center}
	\caption{((a) Non resonant (Non Res.) and resonant (Res.) PL of X$^+$-Mn. Co and cross circularly polarized PL spectra are collected for three different energies of the CW resonant laser (green). Inset: intensity map of the cross-circularly polarized PL detected on the low energy side of X$^+$-Mn as the CW laser is scanned through the high energy side. (b) Energy levels of X$^+$-Mn and identification of the three resonances observed in (a) corresponding to the optical $\Lambda$ systems associated with the e-Mn states $|3,+1\rangle$, $|3,+2\rangle$ and $|2,+2\rangle$.}
	\label{LambdaLevem}
	\end{figure}
	
	\lipsum[6]
			
	
	\section{Spin dynamics under resonant excitation}
	
		Cf article 2016/01
	
		\subsection{Cycling and escaping the $\lambda$-level system}

		Mn in a lattice -> modification of orbital -> spin-orbit interaction. Magnetic anisotropy + anisotropy of strain. (Mn has nuclear spin 5/2 -> hyperfine interaction?)\newline

	\lipsum[10]
	
	\begin{figure}[h!]
	\begin{center}
		\includegraphics[width=14.8cm]{Pictures/ResAutocor.eps}
	\end{center}
	\caption{Auto-correlation of the resonant PL ofor a cross-circularly polarized excitation and detection of the electron-Mn states (a) $|3, +1\rangle$, (b) $|3, +2\rangle$ and (c) $|2, +2\rangle$.}
	\label{AllAutocorB0}
	\end{figure}
		
	\lipsum[7]

	\begin{figure}[h!]
	\begin{center}
		\includegraphics[width=14.8cm]{Pictures/ResPump.eps}
	\end{center}
	\caption{Resonant optical pumping transients obtained under circular polarization switching of the resonant excitation for (a) $|3, +1\rangle$, (b) $|3, +2\rangle$ and (c) $|2, +2\rangle$ at zero field. The insets present the corresponding states which are resonantly excited and detected in $\sigma-$ polarization.}
	\label{AllPumpB0}
	\end{figure}
	
	\lipsum[11]
	
	\begin{figure}[h!]
	\fcapside{\caption{Illustration of the different transfer process on the $\Lambda$ level system associated with $|3, +2\rangle$. Are represented: the ratio of the transition rates from OFF to ON, $\Gamma_{In}$; the ratio of the transition rates from ON to OFF, $\Gamma_{Out}$; and the hole-Mn flip-flop time $\tau_{ff}$.}\label{LambdLoop}}	
	{\begin{center}
		\includegraphics[width=6cm]{Pictures/RelaxMecanism.png}
	\end{center}}
	\end{figure}
		
	\lipsum[8]
	
		\subsection{Relaxation mechanism}
	
	\lipsum[12]

	\begin{figure}[h!]
	\begin{center}
		\includegraphics[width=10cm]{Pictures/AutocorBPw.eps}
	\end{center}
	\caption{Excitation power dependence (a) and transverse magnetic field dependence (b) of the auto-correlation of the resonant PL obtained for an excitation on the high energy branch of the $\lambda$ level system associated to the e-Mn state $|2,+2\rangle$.}
	\label{AutocorExpBPw}
	\end{figure}

	\lipsum[13]

	\begin{figure}[h!]
	\begin{center}
		\includegraphics[width=10cm]{Pictures/PumpBPw.eps}
	\end{center}
	\caption{Optical pumping experiment for an excitation of $|3,+2\rangle$ with modulated circular polarization. A dark time ($\tau_{dark} = 50ns$) is introduced either before (black) or during (red) the change of circular polarization. The black and red diagrams present the corresponding resonant excitation sequences. The inset presents the variation of the ratio $\Delta I/I$ as a function of $\tau_{dark}$. The solid line is an exponential fit with $\tau_{relax} = 80 ns$.}
	\label{PumpExpBPw}
	\end{figure}
	
	\lipsum[14]
	
	\begin{figure}[h!]
	\begin{center}
		\includegraphics[width=10cm]{../FillingPicture.png}
	\end{center}
	\caption{Dark time relaxation}
	\label{PumpExpBPw}
	\end{figure}
	
	\lipsum[24]
	
	\begin{figure}[h!]
	\begin{center}
		\includegraphics[width=10cm]{../FillingPicture.png}
	\end{center}
	\caption{Evolution of $\tau_{ff}$ as a function of the splitting}
	\label{PumpExpBPw}
	\end{figure}
	
	\lipsum[28]

	\begin{figure}[h!]
	\begin{center}
		\includegraphics[width=10cm]{../FillingPicture.png}
	\end{center}
	\caption{Autocorrelation evolution under magnetic field and power variation - model}
	\label{AutocorModBPw}
	\end{figure}

	\lipsum[15]

	\begin{figure}[h!]
	\begin{center}
		\includegraphics[width=10cm]{../FillingPicture.png}
	\end{center}
	\caption{Pumping evolution under magnetic field and power variation - model}
	\label{PumpModBPw}
	\end{figure}
	
	\lipsum[16]
	

	\section{Influence of the strain anisotropy}
	
	\lipsum[17]
		
	\begin{figure}[h!]
	\begin{center}
		\includegraphics[width=10cm]{../FillingPicture.png}
	\end{center}
	\caption{Energy structure with |3, +1> and |3, -1>, and |2, +1> and |2, -1> coupled by E}
	\label{LambdaMixed}
	\end{figure}

	\lipsum[18]

	\begin{figure}[h!]
	\begin{center}
		\includegraphics[width=10cm]{../FillingPicture.png}
	\end{center}
	\caption{Experiment configuration |3, +1>  + Polarization decline and polar rate}
	\label{hMnPolarRate}
	\end{figure}
		
		\lipsum[19]

	\begin{figure}[h!]
	\begin{center}
		\includegraphics[width=10cm]{../FillingPicture.png}
	\end{center}
	\caption{Schema of the QD with spin and magnetic field orientation, and action of the magnetic field on the spin.}
	\label{QDMagField}
	\end{figure}

	\lipsum[20]

	\begin{figure}[h!]
	\begin{center}
		\includegraphics[width=10cm]{../FillingPicture.png}
	\end{center}
	\caption{Polarization rate evolution in B(x and z) and simulation}
	\label{hMnPolarRateB}
	\end{figure}
	
	\lipsum[21]
	
	
\printbibliography

\end{document}