\documentclass[a4paper,12pt]{report}

\usepackage[utf8]{inputenc}
\usepackage[T1]{fontenc}
\usepackage{array}
\usepackage{amsmath}
\usepackage[english]{babel}
\usepackage{graphicx}
\usepackage[a4paper]{geometry}
\usepackage[colorlinks=true,urlcolor=blue,linkcolor=blue]{hyperref}
\usepackage{url}
\usepackage[nottoc,numbib]{tocbibind}
\usepackage{color}
\usepackage{epstopdf}
\usepackage{xcolor}

\makeatletter
	\renewcommand{\thechapter}{\Roman{chapter}}
\makeatother

\begin{document}

\chapter{Magneto-optical study of Cr-doped CdTe quantum dots}

	In this chapter, we will study the photoluminescence of a single Chromium atom in a II-VI quantum dot. We saw in the Chapter I that the magnetic anisotropy of the spin lead to a zero magnetic field splitting of the $0$, $\pm 1$ and $\pm 2$ states. In a neutral Cr-doped quantum dot, such a magnetic anisotropy is induced by the bi-axial strain in the plane of the dots. Studying the magnetic-field dependence of the quantum dots photoluminescence, we will also show the influence of the quantum dot symmetry on carrier-Cr spin coupling. The proximity of dark lead to a characteristic repartition of intensity on the three luminescence peak of the QD. However, some dots are not explained by this model, and need to take the position of Cr atom in regard of the dot into account.

	\section{A system strongly coupled to strain state at the Cr position}
	
		\subsection{Energy structure of a Cr in a quantum dot}
		
	\begin{figure}[h!]
	\begin{center}
		\includegraphics[width=10cm]{../FillingPicture.png}
	\end{center}
	\caption{Dot334 QD4 PLE, and spectra X and X2}
	\label{SpectraX/X2}
	\end{figure}
		
		Lorem ipsum dolor sit amet, consectetur adipiscing elit. Curabitur tortor quam, imperdiet quis facilisis sed, fringilla a quam. Cras ante odio, hendrerit ac ante nec, cursus imperdiet urna. Mauris convallis ultricies purus, nec condimentum erat bibendum vel. Aliquam erat volutpat. Pellentesque condimentum, eros a consequat accumsan, turpis sem euismod nisi, sed fringilla quam turpis sit amet erat. Mauris dictum odio sed nisi dapibus, et molestie mauris rutrum. Praesent convallis dolor in nibh blandit bibendum. Quisque sit amet arcu consectetur lorem luctus venenatis nec quis dui. Aliquam erat volutpat. Aenean auctor elit nec tristique dignissim. Nulla massa mi, efficitur semper ex id, pretium eleifend massa. Vivamus sit amet orci scelerisque, gravida est ut, vulputate odio.
		
	\begin{figure}[h!]
	\begin{center}
		\includegraphics[width=10cm]{../FillingPicture.png}
	\end{center}
	\caption{PLE of dot338 QD3. Highlight on different interesting part (mail 17/02/07 – 21:12).}
	\label{CrPLE}
	\end{figure}

	Curabitur eget ipsum egestas dui viverra suscipit. Cras aliquet lacus vitae erat finibus semper. Nulla pharetra eget urna vitae sodales. Nunc faucibus velit lacus, nec ornare eros aliquet quis. Donec a orci nec sem pulvinar ultricies sit amet ut arcu. Nullam id vehicula enim, at tincidunt velit. Duis vestibulum lorem a molestie fringilla. Nullam tincidunt semper placerat. Donec nibh sem, ornare eget cursus ac, luctus sit amet eros. Phasellus eget interdum nisi. Donec mollis risus id lectus fringilla, et commodo risus iaculis. Donec at lacus sed nibh posuere posuere sit amet eget sapien. In dignissim, enim sit amet convallis fermentum, lacus nulla gravida tortor, non facilisis ex nisl sit amet augue. Maecenas eu enim condimentum, consectetur ligula vel, tincidunt nisl. Nam laoreet dictum volutpat. Donec at erat venenatis, ultrices lorem ac, vestibulum neque.
	
	\begin{figure}[h!]
	\begin{center}
		\includegraphics[width=10cm]{../FillingPicture.png}
	\end{center}
	\caption{Decay time on dot338 QD3}
	\label{CrDecay}
	\end{figure}
	
	Curabitur eget ipsum egestas dui viverra suscipit. Cras aliquet lacus vitae erat finibus semper. Nulla pharetra eget urna vitae sodales. Nunc faucibus velit lacus, nec ornare eros aliquet quis. Donec a orci nec sem pulvinar ultricies sit amet ut arcu. Nullam id vehicula enim, at tincidunt velit. Duis vestibulum lorem a molestie fringilla. Nullam tincidunt semper placerat. Donec nibh sem, ornare eget cursus ac, luctus sit amet eros. Phasellus eget interdum nisi. Donec mollis risus id lectus fringilla, et commodo risus iaculis. Donec at lacus sed nibh posuere posuere sit amet eget sapien. In dignissim, enim sit amet convallis fermentum, lacus nulla gravida tortor, non facilisis ex nisl sit amet augue. Maecenas eu enim condimentum, consectetur ligula vel, tincidunt nisl. Nam laoreet dictum volutpat. Donec at erat venenatis, ultrices lorem ac, vestibulum neque.
	
	\begin{figure}[h!]
	\begin{center}
		\includegraphics[width=10cm]{../FillingPicture.png}
	\end{center}
	\caption{Spectra evolution under symmetry diminution}
	\label{CrSpectraStrain}
	\end{figure}
	
	Curabitur eget ipsum egestas dui viverra suscipit. Cras aliquet lacus vitae erat finibus semper. Nulla pharetra eget urna vitae sodales. Nunc faucibus velit lacus, nec ornare eros aliquet quis. Donec a orci nec sem pulvinar ultricies sit amet ut arcu. Nullam id vehicula enim, at tincidunt velit. Duis vestibulum lorem a molestie fringilla. Nullam tincidunt semper placerat. Donec nibh sem, ornare eget cursus ac, luctus sit amet eros. Phasellus eget interdum nisi. Donec mollis risus id lectus fringilla, et commodo risus iaculis. Donec at lacus sed nibh posuere posuere sit amet eget sapien. In dignissim, enim sit amet convallis fermentum, lacus nulla gravida tortor, non facilisis ex nisl sit amet augue. Maecenas eu enim condimentum, consectetur ligula vel, tincidunt nisl. Nam laoreet dictum volutpat. Donec at erat venenatis, ultrices lorem ac, vestibulum neque.
	
	\begin{figure}[h!]
	\begin{center}
		\includegraphics[width=10cm]{../FillingPicture.png}
	\end{center}
	\caption{Linear polarization dot338 QD3}
	\label{CrLinPolar}
	\end{figure}
	
	Curabitur eget ipsum egestas dui viverra suscipit. Cras aliquet lacus vitae erat finibus semper. Nulla pharetra eget urna vitae sodales. Nunc faucibus velit lacus, nec ornare eros aliquet quis. Donec a orci nec sem pulvinar ultricies sit amet ut arcu. Nullam id vehicula enim, at tincidunt velit. Duis vestibulum lorem a molestie fringilla. Nullam tincidunt semper placerat. Donec nibh sem, ornare eget cursus ac, luctus sit amet eros. Phasellus eget interdum nisi. Donec mollis risus id lectus fringilla, et commodo risus iaculis. Donec at lacus sed nibh posuere posuere sit amet eget sapien. In dignissim, enim sit amet convallis fermentum, lacus nulla gravida tortor, non facilisis ex nisl sit amet augue. Maecenas eu enim condimentum, consectetur ligula vel, tincidunt nisl. Nam laoreet dictum volutpat. Donec at erat venenatis, ultrices lorem ac, vestibulum neque.
	
	\begin{figure}[h!]
	\begin{center}
		\includegraphics[width=10cm]{../FillingPicture.png}
	\end{center}
	\caption{Overall energy structure (with +/- 2 which doesn't luminesce)}
	\label{CrEnergyStruct}
	\end{figure}
	
	Curabitur eget ipsum egestas dui viverra suscipit. Cras aliquet lacus vitae erat finibus semper. Nulla pharetra eget urna vitae sodales. Nunc faucibus velit lacus, nec ornare eros aliquet quis. Donec a orci nec sem pulvinar ultricies sit amet ut arcu. Nullam id vehicula enim, at tincidunt velit. Duis vestibulum lorem a molestie fringilla. Nullam tincidunt semper placerat. Donec nibh sem, ornare eget cursus ac, luctus sit amet eros. Phasellus eget interdum nisi. Donec mollis risus id lectus fringilla, et commodo risus iaculis. Donec at lacus sed nibh posuere posuere sit amet eget sapien. In dignissim, enim sit amet convallis fermentum, lacus nulla gravida tortor, non facilisis ex nisl sit amet augue. Maecenas eu enim condimentum, consectetur ligula vel, tincidunt nisl. Nam laoreet dictum volutpat. Donec at erat venenatis, ultrices lorem ac, vestibulum neque.
	
	\begin{figure}[h!]
	\begin{center}
		\includegraphics[width=10cm]{../FillingPicture.png}
	\end{center}
	\caption{dot338 QD3 spectra temperature evolution}
	\label{CrTemp}
	\end{figure}
		
		\subsection{Deduction of the quantum dot parameters}
		
	\begin{figure}[h!]
	\begin{center}
		\includegraphics[width=10cm]{../FillingPicture.png}
	\end{center}
	\caption{Magneto-optic of dot334 QD3 and dot334 QD4 (X + X2)}
	\label{CrMagOptExp}
	\end{figure}
		
		Lorem ipsum dolor sit amet, consectetur adipiscing elit. Curabitur tortor quam, imperdiet quis facilisis sed, fringilla a quam. Cras ante odio, hendrerit ac ante nec, cursus imperdiet urna. Mauris convallis ultricies purus, nec condimentum erat bibendum vel. Aliquam erat volutpat. Pellentesque condimentum, eros a consequat accumsan, turpis sem euismod nisi, sed fringilla quam turpis sit amet erat. Mauris dictum odio sed nisi dapibus, et molestie mauris rutrum. Praesent convallis dolor in nibh blandit bibendum. Quisque sit amet arcu consectetur lorem luctus venenatis nec quis dui. Aliquam erat volutpat. Aenean auctor elit nec tristique dignissim. Nulla massa mi, efficitur semper ex id, pretium eleifend massa. Vivamus sit amet orci scelerisque, gravida est ut, vulputate odio.
		
	\begin{figure}[h!]
	\begin{center}
		\includegraphics[width=10cm]{../FillingPicture.png}
	\end{center}
	\caption{Linear PL + magneto-optics with modelization and explanation of anti-crossing}
	\label{CrMagOptMod}
	\end{figure}
	
	Write about here the estimated value of D0.
	\newline

	Curabitur eget ipsum egestas dui viverra suscipit. Cras aliquet lacus vitae erat finibus semper. Nulla pharetra eget urna vitae sodales. Nunc faucibus velit lacus, nec ornare eros aliquet quis. Donec a orci nec sem pulvinar ultricies sit amet ut arcu. Nullam id vehicula enim, at tincidunt velit. Duis vestibulum lorem a molestie fringilla. Nullam tincidunt semper placerat. Donec nibh sem, ornare eget cursus ac, luctus sit amet eros. Phasellus eget interdum nisi. Donec mollis risus id lectus fringilla, et commodo risus iaculis. Donec at lacus sed nibh posuere posuere sit amet eget sapien. In dignissim, enim sit amet convallis fermentum, lacus nulla gravida tortor, non facilisis ex nisl sit amet augue. Maecenas eu enim condimentum, consectetur ligula vel, tincidunt nisl. Nam laoreet dictum volutpat. Donec at erat venenatis, ultrices lorem ac, vestibulum neque.
	
	\begin{figure}[h!]
	\begin{center}
		\includegraphics[width=10cm]{../FillingPicture.png}
	\end{center}
	\caption{Excitation power variations on dot334 QD3}
	\label{CrSpectraPwExp}
	\end{figure}
	
	Curabitur eget ipsum egestas dui viverra suscipit. Cras aliquet lacus vitae erat finibus semper. Nulla pharetra eget urna vitae sodales. Nunc faucibus velit lacus, nec ornare eros aliquet quis. Donec a orci nec sem pulvinar ultricies sit amet ut arcu. Nullam id vehicula enim, at tincidunt velit. Duis vestibulum lorem a molestie fringilla. Nullam tincidunt semper placerat. Donec nibh sem, ornare eget cursus ac, luctus sit amet eros. Phasellus eget interdum nisi. Donec mollis risus id lectus fringilla, et commodo risus iaculis. Donec at lacus sed nibh posuere posuere sit amet eget sapien. In dignissim, enim sit amet convallis fermentum, lacus nulla gravida tortor, non facilisis ex nisl sit amet augue. Maecenas eu enim condimentum, consectetur ligula vel, tincidunt nisl. Nam laoreet dictum volutpat. Donec at erat venenatis, ultrices lorem ac, vestibulum neque.
	
	\begin{figure}[h!]
	\begin{center}
		\includegraphics[width=10cm]{../FillingPicture.png}
	\end{center}
	\caption{Power variation simulation}
	\label{CrSpectraPwMod}
	\end{figure}
	
	Curabitur eget ipsum egestas dui viverra suscipit. Cras aliquet lacus vitae erat finibus semper. Nulla pharetra eget urna vitae sodales. Nunc faucibus velit lacus, nec ornare eros aliquet quis. Donec a orci nec sem pulvinar ultricies sit amet ut arcu. Nullam id vehicula enim, at tincidunt velit. Duis vestibulum lorem a molestie fringilla. Nullam tincidunt semper placerat. Donec nibh sem, ornare eget cursus ac, luctus sit amet eros. Phasellus eget interdum nisi. Donec mollis risus id lectus fringilla, et commodo risus iaculis. Donec at lacus sed nibh posuere posuere sit amet eget sapien. In dignissim, enim sit amet convallis fermentum, lacus nulla gravida tortor, non facilisis ex nisl sit amet augue. Maecenas eu enim condimentum, consectetur ligula vel, tincidunt nisl. Nam laoreet dictum volutpat. Donec at erat venenatis, ultrices lorem ac, vestibulum neque.
	
	\begin{figure}[h!]
	\begin{center}
		\includegraphics[width=10cm]{../FillingPicture.png}
	\end{center}
	\caption{Linear polar and magneto-optics simulation with high E}
	\label{CrHighE}
	\end{figure}
	
	Curabitur eget ipsum egestas dui viverra suscipit. Cras aliquet lacus vitae erat finibus semper. Nulla pharetra eget urna vitae sodales. Nunc faucibus velit lacus, nec ornare eros aliquet quis. Donec a orci nec sem pulvinar ultricies sit amet ut arcu. Nullam id vehicula enim, at tincidunt velit. Duis vestibulum lorem a molestie fringilla. Nullam tincidunt semper placerat. Donec nibh sem, ornare eget cursus ac, luctus sit amet eros. Phasellus eget interdum nisi. Donec mollis risus id lectus fringilla, et commodo risus iaculis. Donec at lacus sed nibh posuere posuere sit amet eget sapien. In dignissim, enim sit amet convallis fermentum, lacus nulla gravida tortor, non facilisis ex nisl sit amet augue. Maecenas eu enim condimentum, consectetur ligula vel, tincidunt nisl. Nam laoreet dictum volutpat. Donec at erat venenatis, ultrices lorem ac, vestibulum neque.
		
				
	
	\section{The case of six peaks dots}
	
	Lorem ipsum dolor sit amet, consectetur adipiscing elit. Curabitur tortor quam, imperdiet quis facilisis sed, fringilla a quam. Cras ante odio, hendrerit ac ante nec, cursus imperdiet urna. Mauris convallis ultricies purus, nec condimentum erat bibendum vel. Aliquam erat volutpat. Pellentesque condimentum, eros a consequat accumsan, turpis sem euismod nisi, sed fringilla quam turpis sit amet erat. Mauris dictum odio sed nisi dapibus, et molestie mauris rutrum. Praesent convallis dolor in nibh blandit bibendum. Quisque sit amet arcu consectetur lorem luctus venenatis nec quis dui. Aliquam erat volutpat. Aenean auctor elit nec tristique dignissim. Nulla massa mi, efficitur semper ex id, pretium eleifend massa. Vivamus sit amet orci scelerisque, gravida est ut, vulputate odio.
	
	\begin{figure}[h!]
	\begin{center}
		\includegraphics[width=10cm]{../FillingPicture.png}
	\end{center}
	\caption{dot334 QD150521/22 linear pol and magneto-optics}
	\label{CrSixPeaksMagOpt}
	\end{figure}
	
	Curabitur eget ipsum egestas dui viverra suscipit. Cras aliquet lacus vitae erat finibus semper. Nulla pharetra eget urna vitae sodales. Nunc faucibus velit lacus, nec ornare eros aliquet quis. Donec a orci nec sem pulvinar ultricies sit amet ut arcu. Nullam id vehicula enim, at tincidunt velit. Duis vestibulum lorem a molestie fringilla. Nullam tincidunt semper placerat. Donec nibh sem, ornare eget cursus ac, luctus sit amet eros. Phasellus eget interdum nisi. Donec mollis risus id lectus fringilla, et commodo risus iaculis. Donec at lacus sed nibh posuere posuere sit amet eget sapien. In dignissim, enim sit amet convallis fermentum, lacus nulla gravida tortor, non facilisis ex nisl sit amet augue. Maecenas eu enim condimentum, consectetur ligula vel, tincidunt nisl. Nam laoreet dictum volutpat. Donec at erat venenatis, ultrices lorem ac, vestibulum neque.
	
	\begin{figure}[h!]
	\begin{center}
		\includegraphics[width=10cm]{../FillingPicture.png}
	\end{center}
	\caption{dot390 QD4 electric field map zoom on X+-Cr}
	\label{CrSixPeaksEFieldX+}
	\end{figure}
	
	Curabitur eget ipsum egestas dui viverra suscipit. Cras aliquet lacus vitae erat finibus semper. Nulla pharetra eget urna vitae sodales. Nunc faucibus velit lacus, nec ornare eros aliquet quis. Donec a orci nec sem pulvinar ultricies sit amet ut arcu. Nullam id vehicula enim, at tincidunt velit. Duis vestibulum lorem a molestie fringilla. Nullam tincidunt semper placerat. Donec nibh sem, ornare eget cursus ac, luctus sit amet eros. Phasellus eget interdum nisi. Donec mollis risus id lectus fringilla, et commodo risus iaculis. Donec at lacus sed nibh posuere posuere sit amet eget sapien. In dignissim, enim sit amet convallis fermentum, lacus nulla gravida tortor, non facilisis ex nisl sit amet augue. Maecenas eu enim condimentum, consectetur ligula vel, tincidunt nisl. Nam laoreet dictum volutpat. Donec at erat venenatis, ultrices lorem ac, vestibulum neque.
	
	\begin{figure}[h!]
	\begin{center}
		\includegraphics[width=10cm]{../FillingPicture.png}
	\end{center}
	\caption{dot390 QD14 map under E field (cut at -4V) + spectra at E = -2.5V and E = 0V + linear polar at E = -2.5V and E = 0V}
	\label{CrSixPeaksComplete}
	\end{figure}
	
	Curabitur eget ipsum egestas dui viverra suscipit. Cras aliquet lacus vitae erat finibus semper. Nulla pharetra eget urna vitae sodales. Nunc faucibus velit lacus, nec ornare eros aliquet quis. Donec a orci nec sem pulvinar ultricies sit amet ut arcu. Nullam id vehicula enim, at tincidunt velit. Duis vestibulum lorem a molestie fringilla. Nullam tincidunt semper placerat. Donec nibh sem, ornare eget cursus ac, luctus sit amet eros. Phasellus eget interdum nisi. Donec mollis risus id lectus fringilla, et commodo risus iaculis. Donec at lacus sed nibh posuere posuere sit amet eget sapien. In dignissim, enim sit amet convallis fermentum, lacus nulla gravida tortor, non facilisis ex nisl sit amet augue. Maecenas eu enim condimentum, consectetur ligula vel, tincidunt nisl. Nam laoreet dictum volutpat. Donec at erat venenatis, ultrices lorem ac, vestibulum neque.

\bibliographystyle{unsrt}
\bibliography{../Bibliography}

\end{document}