\documentclass[a4paper,12pt]{report}

\usepackage[utf8]{inputenc}
\usepackage[T1]{fontenc}
\usepackage{array}
\usepackage{amsmath}
\usepackage[english]{babel}
\usepackage{graphicx}
\usepackage[a4paper]{geometry}
\usepackage[colorlinks=true,urlcolor=blue,linkcolor=blue]{hyperref}
\usepackage{url}
\usepackage[nottoc,numbib]{tocbibind}
\usepackage{color}
\usepackage{epstopdf}
\usepackage{xcolor}

\makeatletter
	\renewcommand{\thechapter}{\Roman{chapter}}
\makeatother

\begin{document}

\chapter{Magneto-optical study of Cr-doped CdTe quantum dots}

	In this chapter, we will study the photoluminescence of a single Chromium atom in a II-VI quantum dot. We saw in the chapter I that the magnetic anisotropy of the spin lead to a zero magnetic field splitting of the $0$, $\pm 1$ and $\pm 2$ states. In a neutral Cr-doped quantum dot, such a magnetic anisotropy is induced by the bi-axial strain in the plane of the dots. Studying the magnetic-field dependence of the quantum dots photoluminescence, we will also show the influence of the quantum dot symmetry on carrier-Cr spin coupling. The proximity of dark lead to a characteristic repartition of intensity on the three luminescence peak of the QD. However, some dots are not explained by this model, and need to take the position of Cr atom in regard of the dot into account.

	\section{A system strongly coupled to strain state at the Cr position}
	
		\subsection{Energy structure of a Cr in a quantum dot}
		
		Using the procedure described in the second chapter, we randomly incorporated Cr atom in a CdTe/ZnTe quantum dots, adjusting the density of the Cr atoms to be roughly equal to the density of dots, in order to get QDs containing 0, 1 or a few Cr atoms. The emission of individual QDs, induced by optical excitation with a dye laser tuned on resonance with an excited state of the dots, is studied in magnetic fields (up to 11 T) by optical micro-spectroscopy in Faraday configuration~\cite{BesombesPumpMnSFD}.
		
	\begin{figure}[h!]
	\begin{center}
		\includegraphics[width=10cm]{../FillingPicture.png}
	\end{center}
	\caption{Dot334 QD4 PLE with highlight on quasi-resonant state, and spectra X and X2, along X-X2 od dot334 QD3.}
	\label{SpectraXX2}
	\end{figure}

The low temperature (T=5K) PL of the neutral exciton (X-Cr) and biexciton (X$^2$-Cr) of an individual Cr-doped QD are reported in Fig.~\ref{SpectraXX2}(b). Four emission lines are observed for the neutral species (X and X$^2$). Scanning with an energy tunable laser, we saw that all this complex share a common quasi-resonant state, where all these peaks are at a maximum intensity, as highlighted in Fig.~\ref{SpectraXX2}(a). This is an indication that they originate from the same dot.

The relative intensities of the lines and their splitting change from dot to dot as illustrated in Fig.~\ref{SpectraXX2} (b-c). A splitting of the central line is observed for X-Cr and X$^2$-Cr and an additional line appears on the low energy side of the X-Cr spectra. All these features result from the exchange coupling of the electron and hole spins with a single Cr spin.

	\begin{figure}[h!]
	\begin{center}
		\includegraphics[width=10cm]{../FillingPicture.png}
	\end{center}
	\caption{Graph in $\pi_x$ and $\pi_y$ above linear polarization of dot338 QD3}
	\label{CrLinPolar}
	\end{figure}
	
	One can observe a dependence of the quantum dot emission in linear polarization, presented in Fig.~\ref{CrLinPolar}. The central line (S$_z$=0) is split and linearly polarized along two orthogonal directions. As in non-magnetic QD, this results from a coupling of the two bright excitons $|\pm1\rangle$ by (i) the short range e-h exchange interaction in the presence of valence band mixing and/or (ii) the long-range e-h exchange interaction in a QD with an in-plane shape anisotropy~\cite{SplitInvTh}. This anisotropic e-h exchange energy mixes the bright exciton associated with the same Cr spin state, inducing an extra splitting between them. The mixing is maximum for the central pair of bright exciton (S$_z$=0) which are initially degenerated. The outer lines are also slightly linearly polarized but the influence of the e-h exchange interaction is attenuated by the initial splitting of the $|\pm1\rangle$ excitons induced by the exchange interaction with the Cr spin S$_z$=$\pm1$.	

	\begin{figure}[h!]
	\begin{center}
		\includegraphics[width=10cm]{../FillingPicture.png}
	\end{center}
	\caption{Decay time on dot338 QD3}
	\label{CrDecay}
	\end{figure}

	Looking at the time resolved photoluminescence, presented in Fig.~\ref{CrDecay}, we see that the line (4) present a decay time about twice as long as the high energy peak. A long recombination time one of the characteristic of a dark exciton emission~\cite{DELongLifetime}. Under normal circumstances, a dark exciton recombination is non-radiative. However, it possible to observe a dark exciton recombination emitting a photon in low symmetry quantum dot~\cite{DELum}. This hypothesis will be confirmed by the magneto-optical study of the dot presented in Fig.~\ref{CrMagOptExp} and \ref{CrMagOptMod}.

	\begin{figure}[h!]
	\begin{center}
		\includegraphics[width=10cm]{../FillingPicture.png}
	\end{center}
	\caption{Overall energy structure (with +/- 2 with no luminescence)}
	\label{CrEnergyStruct}
	\end{figure}

	In a II-VI semiconductor, the orbital momentum of the Cr connects the spin of the atom to its local strain environment through the modification of the crystal field and the spin-orbit coupling. For biaxial strain in the (001) plane, the ground state of a Cr spin is split by a strain induced magnetic anisotropy term ${\cal H}_{Cr,\varepsilon_\parallel}=D_0S^2_z$ (see chap.~I). It was deduced from electron paramagnetic resonance of bulk Cr-doped CdTe that $D_0$ is positive for compressive biaxial strain~\cite{EPRCr}. In a self-assembled CdTe/ZnTe QD with large in-plane strain, the Cr spin energy levels are split with S$_z$=0 at low energy (Fig.~\ref{CrEnergyStruct}). A value of $D_0$ in the 1 meV range can be expected for a CdTe layer strained on a ZnTe substrate, as shown in chap.I.
	
	When an electron-hole (e-h) pair is injected in a Cr-doped QD, the bright excitons are split by the exchange interaction between the spins of Cr and carriers. In flat self-assembled QDs, the heavy-holes and light-holes are separated in energy by the biaxial strain and the confinement. In a first approximation, the ground state in such QD is a pure heavy-hole (J$_z$=$\pm$3/2) exciton and the exchange interaction with the Cr spin S is described by the spin Hamiltonian ${\cal H}_{c-Cr}=I_{eCr}\vec{S}\cdot\vec{\sigma}+I_{hCr}S_zJ_z$, with $\vec{\sigma}$ the electron spin and J$_z$ the hole spin operator. I$_{eCr}$ and I$_{hCr}$ are, respectively, the exchange integrals of the electron and the hole spins with the Cr spin. These exchange energies depend on the exchange constant of the $3d$ electrons of the Cr with the carriers in CdTe and on the overlap of the Cr atom with the confined carriers. The exchange interaction of the Cr spin is ferromagnetic for both electron and hole spins in common II-VI semiconductors and a typical exchange constant 4 to 5 times larger for the holes than for the electrons is also expected in CdTe~\cite{DMSCrExchInt,CdCrSExchInt}.
	
	\begin{figure}[h!]
	\begin{center}
		\includegraphics[width=10cm]{../FillingPicture.png}
	\end{center}
	\caption{dot338 QD3 spectra temperature evolution}
	\label{CrTemp}
	\end{figure}
	
	For highly strained CdTe/ZnTe QDs with a weak hole confinement, the strain induced energy splitting of the Cr spin $D_0S^2_z$ is much larger than the exchange energy with the confined carriers ($D_0\gg |I_{hCr}|>|I_{eCr}|$). The exchange interaction with the exciton acts as an effective magnetic field which further splits the Cr spins states S$_z$=$\pm$1 and S$_z$=$\pm$2. The resulting X-Cr energy levels are presented in Fig.~\ref{CrEnergyStruct}. The exciton recombination does not affect the Cr atom and its spin is conserved during the optical transitions. Consequently, the large strained induced splitting of the Cr spin is not directly observed in the optical spectra. However, at low temperature, the Cr spin thermalize on the low energy states S$_z$=0 and S$_z$=$\pm$1. This leads to a PL dominated by three contributions: A central line corresponding to S$_z$=0 and the two outer lines associated with S$_z$=$\pm$1 split by the exchange interaction with the carriers.
	
	Since the thermal energy do not allow the $\pm$2 level to be populated, we tried to rise the sample temperature and see if we were able to have emission corresponding to them. The results of this experiment are presented in Fig.~\ref{CrTemp}. We were able to go to 40K before the peak broadening and the interaction with phonon completely blurred the quantum dot emission. However, even at those temperatures, no emission of the $\pm$2 level was observed.
	\newline
		
	\begin{figure}[h!]
	\begin{center}
		\includegraphics[width=10cm]{../FillingPicture.png}
	\end{center}
	\caption{PLE of dot338 QD3. Highlight on different interesting part (mail 17/02/07 – 21:12).}
	\label{CrPLE}
	\end{figure}

	The scan map itself present some other point of interest, highlighted in Fig.~\ref{CrPLE}.
	
	The first remarkable feature of this scan is the really long luminescence of the acoustic phonon replica. As shown on the zoom in Fig.~\ref{CrPLE} [FIG LETTER HERE], the peaks still emit with an excitation several meV above the line and the dot emission energy, going until 2004 meV. One can also see to sharp intensity diminution in this emission. Mapping the intensity of this peak emission to the quantum dot spectrum, it is evidenced that these diminution occur when the laser is in resonance with a QD emission line. The absorption the preferentially occur in this resonantly excited state than in the acoustic phonon band.
	
	Scanning the dot from close, one can see that it present several excited state. The first one to appear, the one at lower energy, is around 2018.5 meV. On this excited, each peaks as a slightly different resonant energy. One can see that the order of appearance of the two central peaks seems to be reversed compared to the external ones. This phenomenon was first observed on QDs in GaAs quantum well~\cite{FineStructSplitGaAsdots}. This indicate an inversion of the splitting due to electron-hole exchange interaction~\cite{SplitInvTh}.
	
	Another excited state can be saw at 2025 meV. This excited state is also really large and can be linked back to an excitation to the optical phonon. Looking at the $\sigma$ polarized emission of this state (Fig.~\ref{CrPLE} [FIG LETTER HERE]), we can see that this excitation present a really good spin conservation: as can be seen on the emission on a dot quasi-resonant state, the low and high energy peak are strongly $\sigma$ polarized, while the central peaks do not show dependency other circular polarization.
	
	Finally, a last interesting excited state appear at 2030 meV. This state present an exchange-induced splitting  different from the splitting in the quasi-resonant state. This is due to a difference in the carriers and Cr atom wavefunction overlap. One can also noticed the this state present a stronger luminescence in $\sigma_{cross}$ than in $\sigma_{co}$: [TO REDISCUSS] this indicate a spin flip of the hole before the recombination.
		
		\subsection{Deduction of the quantum dot parameters}
		
	\begin{figure}[h!]
	\begin{center}
		\includegraphics[width=10cm]{../FillingPicture.png}
	\end{center}
	\caption{Magneto-optic of dot334 QD3 and dot334 QD4 (X + X2)}
	\label{CrMagOptExp}
	\end{figure}
		
		The structure of the energy levels in Cr-doped QDs is confirmed by the evolution of the PL spectra in magnetic field, presented in Fig.~\ref{CrMagOptExp}. One can see that the Zeeman energy of the exciton under magnetic field can compensate the exciton splitting induced by the exchange interaction with the Cr~\cite{LegerQDGeomEffect}. For QD3, this results in an anti-crossing of $|+1\rangle$ and $|-1\rangle$ excitons due to the e-h exchange interaction around B$_z$=6 T observed both in $\sigma$+ and $\sigma$- polarizations (anti-crossing (2) and (3) in Fig.~\ref{CrMagOptExp}(a)).
		
		The low energy emission presented as a dark exciton in Fig.\ref{CrDecay} show an anti-crossing with the bright excitons under B$_z$ in $\sigma$- polarization (anti-crossing (4) in Fig.~\ref{CrMagOptExp}). As illustrated in Fig.~\ref{CrMagOptMod}(c) this anti-crossing arises from a mixing of the bright and dark excitons interacting with the same Cr spin state. Observed in $\sigma$- polarization, it corresponds to the mixing of the exciton states $|-1\rangle$ and $|+2\rangle$ coupled to the Cr spin S$_z$=-1. This dark/bright exciton coupling $\delta_{12}$ is induced by the e-h exchange interaction in a confining potential of reduced symmetry (lower than C$_{2v}$) \cite{DERecombTh}. In such symmetry, the dark excitons acquire an in-plane dipole moment which lead to possible optical recombination at zero magnetic field \cite{DELum} as observed in these QDs. The oscillator strength of this "dark exciton" increases as the initial splitting between $|-1\rangle$ and $|+2\rangle$ excitons is reduced by the magnetic field (Fig.\ref{CrMagOptMod}(c)).
		
	\begin{figure}[h!]
	\begin{center}
		\includegraphics[width=10cm]{../FillingPicture.png}
	\end{center}
	\caption{Linear PL + magneto-optics with modelisation and explanation of anti-crossing}
	\label{CrMagOptMod}
	\end{figure}
	
	Write about here the estimated value of D0.
	\newline

	Curabitur eget ipsum egestas dui viverra suscipit. Cras aliquet lacus vitae erat finibus semper. Nulla pharetra eget urna vitae sodales. Nunc faucibus velit lacus, nec ornare eros aliquet quis. Donec a orci nec sem pulvinar ultricies sit amet ut arcu. Nullam id vehicula enim, at tincidunt velit. Duis vestibulum lorem a molestie fringilla. Nullam tincidunt semper placerat. Donec nibh sem, ornare eget cursus ac, luctus sit amet eros. Phasellus eget interdum nisi. Donec mollis risus id lectus fringilla, et commodo risus iaculis. Donec at lacus sed nibh posuere posuere sit amet eget sapien. In dignissim, enim sit amet convallis fermentum, lacus nulla gravida tortor, non facilisis ex nisl sit amet augue. Maecenas eu enim condimentum, consectetur ligula vel, tincidunt nisl. Nam laoreet dictum volutpat. Donec at erat venenatis, ultrices lorem ac, vestibulum neque.
	
	\begin{figure}[h!]
	\begin{center}
		\includegraphics[width=10cm]{../FillingPicture.png}
	\end{center}
	\caption{Excitation power variations on dot334 QD3}
	\label{CrSpectraPwExp}
	\end{figure}
	
	Curabitur eget ipsum egestas dui viverra suscipit. Cras aliquet lacus vitae erat finibus semper. Nulla pharetra eget urna vitae sodales. Nunc faucibus velit lacus, nec ornare eros aliquet quis. Donec a orci nec sem pulvinar ultricies sit amet ut arcu. Nullam id vehicula enim, at tincidunt velit. Duis vestibulum lorem a molestie fringilla. Nullam tincidunt semper placerat. Donec nibh sem, ornare eget cursus ac, luctus sit amet eros. Phasellus eget interdum nisi. Donec mollis risus id lectus fringilla, et commodo risus iaculis. Donec at lacus sed nibh posuere posuere sit amet eget sapien. In dignissim, enim sit amet convallis fermentum, lacus nulla gravida tortor, non facilisis ex nisl sit amet augue. Maecenas eu enim condimentum, consectetur ligula vel, tincidunt nisl. Nam laoreet dictum volutpat. Donec at erat venenatis, ultrices lorem ac, vestibulum neque.
	
	\begin{figure}[h!]
	\begin{center}
		\includegraphics[width=10cm]{../FillingPicture.png}
	\end{center}
	\caption{Power variation simulation}
	\label{CrSpectraPwMod}
	\end{figure}
	
	Curabitur eget ipsum egestas dui viverra suscipit. Cras aliquet lacus vitae erat finibus semper. Nulla pharetra eget urna vitae sodales. Nunc faucibus velit lacus, nec ornare eros aliquet quis. Donec a orci nec sem pulvinar ultricies sit amet ut arcu. Nullam id vehicula enim, at tincidunt velit. Duis vestibulum lorem a molestie fringilla. Nullam tincidunt semper placerat. Donec nibh sem, ornare eget cursus ac, luctus sit amet eros. Phasellus eget interdum nisi. Donec mollis risus id lectus fringilla, et commodo risus iaculis. Donec at lacus sed nibh posuere posuere sit amet eget sapien. In dignissim, enim sit amet convallis fermentum, lacus nulla gravida tortor, non facilisis ex nisl sit amet augue. Maecenas eu enim condimentum, consectetur ligula vel, tincidunt nisl. Nam laoreet dictum volutpat. Donec at erat venenatis, ultrices lorem ac, vestibulum neque.
	
	\begin{figure}[h!]
	\begin{center}
		\includegraphics[width=10cm]{../FillingPicture.png}
	\end{center}
	\caption{Linear polar and magneto-optics simulation with high E}
	\label{CrHighE}
	\end{figure}
	
	Curabitur eget ipsum egestas dui viverra suscipit. Cras aliquet lacus vitae erat finibus semper. Nulla pharetra eget urna vitae sodales. Nunc faucibus velit lacus, nec ornare eros aliquet quis. Donec a orci nec sem pulvinar ultricies sit amet ut arcu. Nullam id vehicula enim, at tincidunt velit. Duis vestibulum lorem a molestie fringilla. Nullam tincidunt semper placerat. Donec nibh sem, ornare eget cursus ac, luctus sit amet eros. Phasellus eget interdum nisi. Donec mollis risus id lectus fringilla, et commodo risus iaculis. Donec at lacus sed nibh posuere posuere sit amet eget sapien. In dignissim, enim sit amet convallis fermentum, lacus nulla gravida tortor, non facilisis ex nisl sit amet augue. Maecenas eu enim condimentum, consectetur ligula vel, tincidunt nisl. Nam laoreet dictum volutpat. Donec at erat venenatis, ultrices lorem ac, vestibulum neque.
		
				
	
	\section{The case of six peaks dots}
	
	Lorem ipsum dolor sit amet, consectetur adipiscing elit. Curabitur tortor quam, imperdiet quis facilisis sed, fringilla a quam. Cras ante odio, hendrerit ac ante nec, cursus imperdiet urna. Mauris convallis ultricies purus, nec condimentum erat bibendum vel. Aliquam erat volutpat. Pellentesque condimentum, eros a consequat accumsan, turpis sem euismod nisi, sed fringilla quam turpis sit amet erat. Mauris dictum odio sed nisi dapibus, et molestie mauris rutrum. Praesent convallis dolor in nibh blandit bibendum. Quisque sit amet arcu consectetur lorem luctus venenatis nec quis dui. Aliquam erat volutpat. Aenean auctor elit nec tristique dignissim. Nulla massa mi, efficitur semper ex id, pretium eleifend massa. Vivamus sit amet orci scelerisque, gravida est ut, vulputate odio.
	
	\begin{figure}[h!]
	\begin{center}
		\includegraphics[width=10cm]{../FillingPicture.png}
	\end{center}
	\caption{dot334 QD150521/22 linear pol and magneto-optics}
	\label{CrSixPeaksMagOpt}
	\end{figure}
	
	Curabitur eget ipsum egestas dui viverra suscipit. Cras aliquet lacus vitae erat finibus semper. Nulla pharetra eget urna vitae sodales. Nunc faucibus velit lacus, nec ornare eros aliquet quis. Donec a orci nec sem pulvinar ultricies sit amet ut arcu. Nullam id vehicula enim, at tincidunt velit. Duis vestibulum lorem a molestie fringilla. Nullam tincidunt semper placerat. Donec nibh sem, ornare eget cursus ac, luctus sit amet eros. Phasellus eget interdum nisi. Donec mollis risus id lectus fringilla, et commodo risus iaculis. Donec at lacus sed nibh posuere posuere sit amet eget sapien. In dignissim, enim sit amet convallis fermentum, lacus nulla gravida tortor, non facilisis ex nisl sit amet augue. Maecenas eu enim condimentum, consectetur ligula vel, tincidunt nisl. Nam laoreet dictum volutpat. Donec at erat venenatis, ultrices lorem ac, vestibulum neque.
	
	\begin{figure}[h!]
	\begin{center}
		\includegraphics[width=10cm]{../FillingPicture.png}
	\end{center}
	\caption{dot390 QD4 electric field map zoom on X+-Cr}
	\label{CrSixPeaksEFieldX+}
	\end{figure}
	
	Curabitur eget ipsum egestas dui viverra suscipit. Cras aliquet lacus vitae erat finibus semper. Nulla pharetra eget urna vitae sodales. Nunc faucibus velit lacus, nec ornare eros aliquet quis. Donec a orci nec sem pulvinar ultricies sit amet ut arcu. Nullam id vehicula enim, at tincidunt velit. Duis vestibulum lorem a molestie fringilla. Nullam tincidunt semper placerat. Donec nibh sem, ornare eget cursus ac, luctus sit amet eros. Phasellus eget interdum nisi. Donec mollis risus id lectus fringilla, et commodo risus iaculis. Donec at lacus sed nibh posuere posuere sit amet eget sapien. In dignissim, enim sit amet convallis fermentum, lacus nulla gravida tortor, non facilisis ex nisl sit amet augue. Maecenas eu enim condimentum, consectetur ligula vel, tincidunt nisl. Nam laoreet dictum volutpat. Donec at erat venenatis, ultrices lorem ac, vestibulum neque.
	
	\begin{figure}[h!]
	\begin{center}
		\includegraphics[width=10cm]{../FillingPicture.png}
	\end{center}
	\caption{dot390 QD14 map under E field (cut at -4V) + spectra at E = -2.5V and E = 0V + linear polar at E = -2.5V and E = 0V}
	\label{CrSixPeaksComplete}
	\end{figure}
	
	Curabitur eget ipsum egestas dui viverra suscipit. Cras aliquet lacus vitae erat finibus semper. Nulla pharetra eget urna vitae sodales. Nunc faucibus velit lacus, nec ornare eros aliquet quis. Donec a orci nec sem pulvinar ultricies sit amet ut arcu. Nullam id vehicula enim, at tincidunt velit. Duis vestibulum lorem a molestie fringilla. Nullam tincidunt semper placerat. Donec nibh sem, ornare eget cursus ac, luctus sit amet eros. Phasellus eget interdum nisi. Donec mollis risus id lectus fringilla, et commodo risus iaculis. Donec at lacus sed nibh posuere posuere sit amet eget sapien. In dignissim, enim sit amet convallis fermentum, lacus nulla gravida tortor, non facilisis ex nisl sit amet augue. Maecenas eu enim condimentum, consectetur ligula vel, tincidunt nisl. Nam laoreet dictum volutpat. Donec at erat venenatis, ultrices lorem ac, vestibulum neque.

\bibliographystyle{unsrt}
\bibliography{../Bibliography}

\end{document}